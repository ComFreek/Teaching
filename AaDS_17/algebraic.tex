\section{Specification}

Algebraic data types are a not clearly delineated family of record types.
Typically one field of the record is a type, and the other fields are operations on that type.

Many of them represent mathematical theories, in which case their elements represent mathematical structures.
Therefore, they tend to come up a lot.
But their high level of abstraction leads to them often being neglected or not understood.

Three classes are of major importance: these are the data types based on one type $U$ and
\begin{itemize}
\item one binary relation $U\times U \to \Bool$ such as in graphs, preorders, orders, and equivalence relations,
\item one binary function $U\times U\to U$ such as in monoids, groups, and semi-lattices,
\item two binary functions $U\times U\to U$ such as in rings, fields, and lattices
\end{itemize}
and several axioms such as transitivity, associativity, or distributivity.

The most important special cases are specified in Sect.~\ref{sec:math:binrel} for a binary relation and in Sect.~\ref{sec:math:binop} for one binary function.
We omit the case for two binary functions.

\section{Data Structures}

Apart from the axioms, we can implement algebraic data types very well as polymorphic abstract classes that take $U$ as the type parameter.
However, the axioms can usually not be implemented elegantly (except in very advanced programming languages) and must be realized as comments.

\subsection{One Binary Relation}

We already implemented some of them when sorting lists.

For example, the type of orders on $U$ can be realized as
\begin{acode}
\aclassA{Relation[A]}{}{}{
  \afun[\Bool]{rel}{x:A,y:A}{\ldots}
}\\
\aclassA{Preorder[A]}{}{Relation[A]}{
  \acomment{$rel$ is reflexive and transitive}
}\\
\aclassA{Order[A]}{}{Preorder[A]}{
  \acomment{$rel$ is anti-symmetric}
}\\
\aclassA{TotalOrder[A]}{}{Order[A]}{
  \acomment{$rel$ is total}
}
\end{acode}

Individual structures can now be implemented as concrete classes that implement the abstract ones.
We have already implemented some concrete orders such as $\leq$ on $\Int$, $|$ on $\Int$, or the lexicographic ordering on $\String$.

\subsection{One Binary Function}

For example, the type of monoids on $U$ can be realized as
\begin{acode}
\aclassA{BinOp[A]}{}{}{
  \afun[A]{op}{x:A,y:A}{\ldots}
}\\
\aclassA{Monoid[A]}{}{BinOp[A]}{
  \afun[A]{e}{}{\ldots}\\
  \acomment{$op$ is associative and has neutral element $e$}
}
\end{acode}

Individual structures can now be implemented as concrete classes that implement the abstract ones.

The following implements some example monoids:
\begin{acode}
\aclass{Addition}{}{Monoid[\Int]}{
  \afunI{op}{x:\Int,y:\Int}{x+y}\\
  \afunI[A]{e}{}{0}
}
\end{acode}

\begin{acode}
\aclass{Multiplication}{}{Monoid[\Int]}{
  \afunI{op}{x:\Int,y:\Int}{x*y}\\
  \afunI[A]{e}{}{1}
}
\end{acode}

\begin{acode}
\aclass{Concatenation}{}{Monoid[\String]}{
  \afunI{op}{x:\String,y:\String}{x+y}\\
  \afunI[A]{e}{}{""}
}
\end{acode}

\begin{acode}
\aclass{Maximum}{}{Monoid[\N]}{
  \afunI{op}{x:\N,y:\N}{\aifelseI{x\leq y}{y}{x}}\\
  \afunI[A]{e}{}{0}
}
\end{acode}
In each case, we have to check that the axioms (associativity and neutrality) actually hold to show the correctness of the implementations.

For a more complex example, consider the monoid of $2\times 2$ matrices under multiplication:
\begin{acode}
\aclass{Matrix22}{a:\Int,b:\Int,c:\Int,d:\Int}{}{}
\\
\aclass{Matrix22Multiplication}{}{Monoid[Matrix22]}{
  \afunI{op}{x:\N,y:\N}{\ldots}\\
  \afunI[A]{e}{}{\anew{Matrix22}{1,0,0,1}}
}
\end{acode}
where $\anew{Mat}{a,b,c,d}$ represents the matrix $\begin{pmatrix}a & b \\ c & d\end{pmatrix}$.

There are many more monoids that come up all the time.
Some examples are
\begin{compactitem}
 \item $\wedge$ and $\true$ yield a $Monoid[\Bool]$.
 \item $\vee$ and $\false$ yield a $Monoid[\Bool]$.
 \item Concatenation and empty list yield a $Monoid[List[A]]$ for every $A$.
 \item Union and empty set yield a $Monoid[Set[A]]$ for every $A$.
 \item If we can implement the set $Full[A]:Set[A]$ containing all elements of $A$, then intersection and $Full[A]$ yield a $Monoid[Set[A]]$ for every $A$.
 \item Minimum and $\infty$ yield a $Monoid[\N^\infty]$.
 \item $gcd$ and $0$ yield a $Monoid[\N]$.
 \item Function composition and identity function yield a $Monoid[A\to A]$.
 \item Concatenation of paths and empty path yield a monoid of paths in a graph. (See Ch.~\ref{sec:ad:graphs} for paths in a graph.)
\end{compactitem}

\subsection{Two Binary Functions}

The types of, e.g., rings and fields are implemented accordingly.

\section{Important Algorithms}

The implementation of algebraic data types and algorithms that
\begin{compactitem}
 \item are used in concrete algebraic structures
 \item compute new algebraic structures from existing ones
\end{compactitem}
are studied in the field of \emph{computer algebra}.
This is essential for mathematical computation.
\medskip

Additionally, many algorithms can be easily generalized to take an arbitrary structure as their input.
For example, sorting can be implemented as a function \[sort(x:List[\Int]):List[\Int].\]
But it is much more general, equally easy, and essentially as fast to implement a function \[sort[A](ord: TotOrd[A], x:List[A]):List[A]\] that sorts with respect to an arbitrary total order.

We give some examples.

\subsection{Folding Lists over a Monoid}\label{sec:ad:monoidfold}

The function \[fold[A,B](x:List[A], f: A\times B\to B): B\] for lists is important but often confusing.

It becomes much simpler if we consider the special case $A=B$:
\[fold[A](x:List[A], e: A, f: A\times A\to A): A\]
If we additionally write $f$ as infix, i.e., write $x\,f\,y$ instead of $f(x,y)$, we have
\[fold[A]([a_1,\ldots,a_n],e,f)=a_1\,f\,(a_2\,\ldots\,(a_{n-1}\,f\,(a_n\,f\, e))\ldots)\]

In general, the bracketing matters.
But if $f$ is associative, the bracketing becomes irrelevant.
If additionally $e$ is a neutral element for $f$, then $f$ and $e$ form a $Monoid[A]$.
Then we can write $fold$ as
\[monoidFold[A](mon:Monoid[A], x:List[A])\]
\[monoidFold[A](mon, [a_1,\ldots,a_n])=a_1\,mon.op\,a_2\,\ldots\,a_{n-1}\,mon.op\,a_n\,mon.op\, mon.e\]
In particular, we have
\[monoidFold[A](mon, [])=mon.e \tb\mand\tb monoidFold[A](mon, [a])=a.\]

For example,
\begin{compactitem}
\item $monoidFold[\Int](\anew{Addition}{}, x)$ is the sum of all elements in $x$,
\item $monoidFold[\Int](\anew{Multiplication}{}, x)$ is the product of all elements in $x$,
\item $monoidFold[\N](\anew{Maximum}{}, x)$ is the greatest element in $x$,
\item $monoidFold[\String](\anew{Concatenation}{}, x)$ is the concatenation of all strings in $x$.
\end{compactitem}

\subsection{Square-and-Multiply}

We can finally give the square-and-multiply algorithm from Sect.~\ref{sec:ad:exp:sqmult} in full generality.

This should be a function
\begin{acode}
\afun[A]{sqmult[A]}{mon:Monoid[A], x:A, n: \N}{\ldots}
\end{acode}
that satisfies the specification
\[sqmult(mon, x, 0)=mon.e\]
\[sqmult(mon, x, n+1)=mon.op(sqmult(mon, x, n), x)\]
and whose time complexity is $\Theta(\log_2 n)$.

% matrix multiplication for any semi-ring? use for flow?
