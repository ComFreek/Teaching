\section{Overview}

Consider a problem that involves multiple successive choices.
A backtracking algorithm always chooses one of the options and proceeds.
If that leads to a situation where no further progress is possible, the algorithm reverts all steps since the most recent choice and chooses a different option.
The reversal is called \textbf{backtracking}.

The typical example is finding the exit of a maze: at any given intersection, we have to choose one of the possible paths.
If we ever reach a dead end, we backtrack all steps since the most recent intersection and choose a different path.

Backtracking algorithms make sense if
\begin{compactitem}
 \item we have little or no information to predict the best choice right away,
 \item after making a choice, we can detect quickly that it was the wrong choice.
\end{compactitem}
If the first condition is not met, we can usually do better by finding the best choice instead of trying out a random choice.
If the second condition is not met, the backtracking algorithm degenerates into DFS.

For the maze example, we cannot detect early on that a path leads to a dead end.
Therefore, we have to walk all the way to the dead end before we can backtrack.
Thus, we end up performing a DFS of the maze (seen as a graph whose nodes are the intersections).

\section{General Structure}

Consider a problem whose solution involves a series of choices among finitely many options.
We can represent this as finding a branch in a tree:
\begin{compactitem}
 \item the root is the starting point,
 \item at every node $n$, we have to choose one out of the children of $n$,
 \item at a leaf, we can test whether we have a solution or not.
\end{compactitem}

Thus, we obtain the following general problem: given a tree $T:Tree[A]$ and a property $solution:List[A]\to\Bool$, find a branch $b$ of $T$ such that $solution(b)$.

Backtracking works well if we additionally have a test $abort:List[A]\to\Bool$ such that $abort(b)$ implies that there is no solution that starts with $b$.

To be efficient, we avoid ever building the entire tree $T$---otherwise, we would waste the cost-saving effect of $abort$.
Then we can give a general backtracking algorithm as
\begin{acode}
\afun[{Option[List[A]]}]{search}{state:List[A]}{
  \aifI{abort(state)}{\areturn{None}}\\
  \aifI{solution(state)}{\areturn{Some(state)}}\\
  foreach(choices(state), \alam{c}{\ablock{
    x := search(state+[c])\\
    \aifI{x \neq None}{\areturn{x}}
   }}\\)\\
   \areturn{None}
}
\end{acode}

Here the argument $state$ represents the path from the root to the current node, and $choices(state)$ yields its children.
The algorithm is essentially a DFS-algorithm in the tree $T$ that
\begin{compactitem}
 \item never builds $T$ as a whole,
 \item stops as soon as the first solution has been found,
 \item uses $abort$ to avoid traversing a subtree.
\end{compactitem}

\begin{example}
A standard example is the $8$-queens problem: place $8$ queens on a chess-board such that none threatens the other.

Clearly, there has to be exactly one queen in each row.
Thus, we can represent every solution as a list $[c_1,\ldots,c_8]$ such that $c_i$ is the column coordinate of the queen in row $i$.

The possible choices are defined by
\[choices(state) \cas{[1,2,3,4,5,6,7,8] \mifc length(state)<8 \\ [] \mifc length(state)=8}\]
i.e., in every row we have to choose one out of $8$ columns, and we reach possible solutions after making $8$ choices.

The function $abort(state)$ return $\true$ if any two queens threaten each other.
Because this can be tested efficiently already if $length(state)<8$, we can use it to backtrack early.
\end{example}

\section{Constraint Satisfaction Problems}

A constraint satisfaction problem (CSP) consists of a set $x_1:A_1,\ldots,x_m:A_m$ of variables and a set $C_1(\vec{x}),\ldots,C_n(\vec{x	})$ of boolean expressions (called the \textbf{constraints}) about the variables.

A typical example is a system of $n$ linear equations in $m$ variables.
But it is not required that the constraints are equations.

Many practical problems can be captured as CSPs.
This includes many games and puzzles:

\begin{example}[Sudoku]
The Sudoku game is a CSP with $m=81$ variables $x_{ij}\in\{1,\ldots,9\}$ for $i,j=1,\ldots,9$.
It has at least the $27$ constraints of the form $\{x_{i1},\ldots,x_{i9}\}=\{1,\ldots,9\}$ for each row $i$ and correspondingly for each column and each of the $9$ $3x3$ squares.

Additionally, each instance of Sudoku has a few constraints of the form $x_{ij}=c$, which fix the values of certain variables.
\end{example}

Some CSPs can be solved efficiently using backtracking algorithms.
Step $i$ tries to fill in a value for $x_i$.
After $i$ steps $state$ is the list $[v_1,\ldots,v_i]$ of currently tried values for $x_1,\ldots,x_i$.
$choices(state)=A_{i+1}$ returns the possible values for the next variable $x_{i+1}$.
$abort(state)$ checks if there is some $C_j$ such that $C_j(v_1,\ldots,v_i,x_{i+1},\ldots,x_n)$ is false no matter which values are used for $x_{i+1},\ldots,x_n$.
$solution(state)$ evaluates the conjunction of all constraints.

\begin{example}[SAT]
The SAT problem is the special case of CSP where $A_i=\Bool$ for all $i$ and all $C_i$ are boolean expressions (i.e., using only $\neg$, $\wedge$, and $\vee$).
Because there are only $2$ choices for each variable and boolean expressions can be evaluated very easily, it can be solved relatively quickly using a backtracking algorithm.

However, even with a good $abort$ function, the fastest known SAT algorithms are exponential.
\end{example}

