	\documentclass[a4paper]{article}

\usepackage[course={Algorithms and Data Structures},number=11,date=2017-05-02,duedate=2017-05-11,duetime=11:00,solutions]{../../myhomeworks}

\newcounter{chapter} % needed for dependencies of mylecturenotes
\usepackage[root=../..]{../../mylecturenotes}
\usepackage{../../macros/algorithm}

\begin{document}

\header

\begin{problem}{Tail Recursion}{6}
Give a tail-recursive definition of the function $map[A](x:List[A],f:A\to B):List[B]$ of lists.

The following partial solution may help:
\begin{acode}
\afun{map[A]}{x:List[A],f:A\to B}{
  mapAux(x,f,\tb\tb)
}\\
\\
\afun{mapAux[A]}{x:List[A],f:A\to B,result:List[B]}{
\\
}
\end{acode}


\begin{solution}
\begin{acode}
\afun{map[A]}{x:List[A],f:A\to B}{
  mapAux(x,f,Nil)
}\\
\\
\afun{mapAux[A]}{x:List[A],f:A\to B,result:List[B]}{
  \amatch{x}{
    \acase{Nil}{reverse(result)},
    \acase{{Cons(hd,tl)}}{mapAux(tl, f, Cons(f(hd), result))}
  }
}
\end{acode}

The additional argument $result$ accumulates the argument, and the base case returns it.

To be linear instead of quadratic, the accumulator collects the result in reverse order, and the base case undoes the reversal.
\end{solution}
\end{problem}

\begin{problem}{Backtracking}{8}
Write a program that finds a solution to the $n$-queens problems (on an $n\times n$ board) using the general backtracking algorithm.
\end{problem}

\begin{problem}{Divide and Conquer}{8}
Implement Karatsuba's divide-and-conquer algorithm for the multiplication of two polynomials of degree $2^n-1$ as described in the notes.
\end{problem}

\begin{problem}{Master Theorem}{6}
Apply the master theorem to derive the $\Theta$-class of the time complexity of
\begin{compactenum}
  \item mergesort
  \item binary search
  \item Karatsuba multiplication of polynomials
\end{compactenum}
Show your work.

\begin{solution}
We use $d$, $r$, and $c$ as in the statement of the Master theorem.
\begin{compactenum}
  \item $d=2$ and $r=2$. We know dividing takes constant and merging liner time, so $c=1$. Then $r=d^c$, and we obtain $\Theta(n\log_2 n)$.
  \item $d=2$ and $r=1$. We know dividing takes constant time and no merging is necessary, so $c=0$. Then $r=d^c$, and we obtain $\Theta(\log_2 n)$.
  \item $d=2$ and $r=3$. We know dividing takes linear time (splitting and adding polynomials) and merging linear time (adding and shifting polynomials), so $c=1$.
  Then $r>d^c$, and we obtain $\Theta(n^{\log_d r})=\Theta(n^{\log_2 3})\approx\Theta(n^{1.58})$.
\end{compactenum}

\end{solution}

\end{problem}

\end{document}
