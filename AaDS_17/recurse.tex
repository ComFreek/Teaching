\section{Overview}

We speak of recursion if a function calls itself.
More generally, a set of functions is recursive if they call each other in a cyclic way.

A recursive algorithm is a good design choice if we can reduce the problem $P(x)$ to the problem $P(x')$ such that $x'$ is (in some sense) smaller than $x$.
Usually, $x'$ is just a little smaller than $x$, e.g., we might have $x\in\N$ and $x'=x-1$.

Recursive algorithms have been used in multiple places throughout these notes.
In the following, we look at some frequent patterns of recursive algorithms.

Here we only consider the case where $P(x)$ is reduced to slightly smaller instances, e.g., $P(x-1)$.
There are other patterns, which we discuss in separate chapters, that may or may not be implemented using recursion.
The following gives an overview:
\begin{compactitem}
  \item backtracking: $P(x)$ is reduced to some smaller instance, but we have to try multiple instances before finding the right one.
  \item divide-and-conquer: $P(x)$ is reduced to multiple much smaller instances, e.g., $P(x/2)$.
  \item dynamic programming: $P(x)$ is computed by first computing all smaller instances, e.g., $P(0),\ldots,P(x-1)$.
  \item parallelization: $P(x)$ is reduced to multiple smaller instances, whose solutions are computed in parallel.
\end{compactitem}

\section{Induction}

Induction is the special case of recursion that comes up in conjunction with inductive data types such as natural numbers or lists.
It is used to define most operations on these types.
We consider some examples.

\subsection{Natural Numbers}

Consider the data type of natural numbers:
\begin{acode}
\adata{nat}{zero,succ(pred: nat)}
\end{acode}

An inductive algorithm for $P(n:nat)$ employs a case split on $n$, usually using pattern-matching:
\begin{compactitem}
 \item if $n=zero$: return $P(zero)$ directly without recursion (the base case),
 \item if $n=succ(n')$: recursively compute $y=P(n')$, then use $y$ to compute $P(succ(n'))$ without further recursion (step case).
\end{compactitem}

As an example, consider the inductive algorithm for the factorial on the left:
\newpage
\begin{multicols}{2}
\begin{acode}
\afun{fact}{n:nat}{
  \amatch{n}{\acase{zero}{1}, \acase{succ(n')}{\ablock{y:=fact(n')\\ y\cdot succ(n')}}}
}
\end{acode}
\columnbreak
\begin{acode}
\afun{fact}{n:\Int}{
  \aifelse{n==0}{1}{y:=fact(n-1)\\ y\cdot n}
}
\end{acode}
\end{multicols}

Essentially the same algorithms can also be used if $nat$ is not defined as an inductive type.
Instead of pattern-matching, this requires an if-statement.
For example, if we use $\Int$ to represent natural numbers, we obtain the algorithm on the right above.
\medskip

Virtually all arithmetic operations can be elegantly defined as recursive algorithms along these lines.

\subsection{Lists}

Consider the inductive type of lists:
\begin{acode}
\adata{IndList[A]}{nil,{cons(head:A,tail:IndList[A])}}
\end{acode}

It is slightly more complicated than $nat$.
But an inductive algorithm for $P(x:IndList[A])$ proceeds in essentially the same way:
\begin{compactitem}
 \item if $x=nil$: return $P(nil)$ directly without recursion (the base case),
 \item if $x=cons(hd,tl)$: recursively compute $y:=P(tl)$, then use $hd$ and $y$ to compute $P(cons(hd,tl))$ without further recursion (step case).
\end{compactitem}

Some examples were already given elsewhere in these notes.
The simplest example is the $length$ function:
\begin{acode}
\afun[\Int]{length}{x:IndList[A]}{\amatch{x}{\acase{nil}{0},\acase{cons(hd,tl)}{\ablock{y:=length(tl)\\y+1}}}}
\end{acode}
Here the $cons$-case ignores the value $hd$.

Like for natural numbers, the corresponding recursive algorithms can also be used if lists are not defined as an inductive type.

\subsection{Binary Trees}

Full binary trees are often defined as the following inductive type.
If all nodes are labeled with values from $A$, we obtain
\begin{acode}
\adata{BinTree[A]}{Leaf(label: A),{Node(label: A, left:BinTree[A], right:BinTree[A])}}
\end{acode}

It is more complicated than $IndList[A]$ because it uses two inductive arguments.
But an inductive algorithm for $P(x:BinTree[A])$ proceeds in essentially the same way:
\begin{compactitem}
 \item if $x=Leaf(a)$: compute the result from $a$ directly without recursion,
 \item if $x=Node(a,l,r)$: recursively compute $y:=P(l)$ and $y'=P(r)$, then use $a$, $y$, and $y'$  without further recursion.
\end{compactitem}

As an example, consider a function that returns the list of labels in DFS-order:
\begin{acode}
\afun[{List[A]}]{labels}{x:BinTree[A]}{
  \amatch{x}{\acase{Leaf(a)}{{cons(a,nil)}},\acase{Node(a,l,r)}{\ablock{y:=labels(l) \\ y':=labels(r)\\ cons(a, concat(y,y'))}}}
}
\end{acode}
Here the $cons$-case ignores the value $hd$.

Like for other types, the corresponding recursive algorithms can also be used if trees are not defined as an inductive type.

% min-max example?

\section{Mutual Recursion}

We speak of \emph{mutual recursion} if multiple functions call each other.

A very simple non-trivial example is the $even$-$odd$ recursion:
\begin{acode}
\afun[\Bool]{even}{n:\N}{
  \aifelse{n==0}{\true}{odd(n-1)}
}\\
\\
\afun[\Bool]{odd}{n:\N}{
  \aifelse{n==0}{\false}{even(n-1)}
}
\end{acode}

Of course, that is very inefficient for large $n$ and only makes sense if the $n\modop 2$ function is not available.

Termination arguments for sets of mutually recursive functions are more difficult than for single recursive functions.
But the basic idea is the same: every recursive call should make the argument smaller in some sense.

\section{Recursion vs. While-Loops}\label{sec:ad:recurse:while}

Technically, recursion is redundant if the programming language allows for while-loops.
But recursion is such a versatile concept that it is part of every practical programming language.

The following example shows how a while-loop can be systematically replaced with a recursive function:

\begin{multicols}{2}
\begin{acode}
x:=x_0\\
y:=y_0\\
z:=z_0\\
\awhile{C(x,y,z)}{
  (x,y,z):=Body(x,y,z)\\
}
\end{acode}
\columnbreak
\begin{acode}
\afun{f}{x,y,z}{
  \aifelse{C(x,y,z)}{
    (x',y',z'):=Body(x,y,z)\\
    f(x',y',z')
  }{(x,y,z)}
}\\
f(x_0,y_0,z_0)
\end{acode}
\end{multicols}

On the left, $x$, $y$, and $z$ are the variables, whose values may change during an iteration of the loop, and $C(x,y,z)$ is the loop condition.
$Body$ is some piece of code that may use the current values of $x$, $y$, and $z$, and assign new values to them.
This is indicated in the line $(x',y',z'):=Body(x,y,z)$, which treats $Body$ as a function that takes the old values, performs arbitrary additional operations, and eventually returns the new values of the variables.

The right side shows the equivalent recursive function.

Above we use three variables $x$, $y$, and $z$.
Any other number works accordingly---the only critical aspect is that all variables whose value may be changed by $Body$ must become argument of the recursive function.

The example below uses two mutable variables $result$ and $i$.
It computes the factorial of $n$ in the variable $result$.
$C(result,i)$ is the formula $i>0$.

\newpage
\begin{multicols}{2}
\begin{acode}
result:=1\\
i:=n\\
\awhile{i > 0}{
  (result,i):=(result\cdot i, i-1)
}
\end{acode}
\columnbreak
\begin{acode}
\afun{f}{result,i}{
  \aifelse{i>0}{
    (result',i'):=(result\cdot i, i-1)\\
    f(result',i')
  }{(result,i)}
}\\
f(1,n)
\end{acode}
\end{multicols}

The example may look a bit odd because $Body$ is written in a way that emphasizes the correspondence between while-loops and recursions.
If we rewrite them separately into more familiar styles, we obtain

\begin{multicols}{2}
\begin{acode}
result:=1\\
i:=n\\
\awhile{i > 0}{
  result := result\cdot i\\
  i:=i-1
}
\end{acode}
\columnbreak
\begin{acode}
\afun{f}{result,i}{
  \aifelse{i>0}{
    f(result\cdot i,i-1)
  }{result}
}\\
f(1,n)
\end{acode}
\end{multicols}

Note that the recursive functions that correspond to while-loops have some special properties:
\begin{compactitem}
 \item There is exactly one recursive call, and no other code is executed after it.
 \item The return value (if any) is not used for further computations.
 \item The base case returns all function arguments.
  In order to return anything useful, at least one of the arguments must have the role of collecting the result.
  In the example, the argument $result$ has that role: it is not used in $f$ except for building the result; therefore, the base case can simply return $result$.
\end{compactitem}

Thus, while-loops correspond to a very small class of recursive functions.
It is also possible to translate \emph{any} recursive function (even any set of mutually recursive functions) into a while-loops.
But the translation is a bit more complicated, and the two translations are not inverse to each other.

\section{Tail-Recursion}\label{sec:ad:recurse:tail}

\subsection{Tail-Recursive Functions}

In Sect.~\ref{sec:ad:recurse:while}, we encountered a class of recursive functions with special properties.

Concretely, a recursive function $f$ is called \textbf{tail-recursive}, if all its recursive calls occur in positions where the result of the recursive call directly becomes the result of the $f$.

For example, the following is a tail-recursive implementation of the factorial of $n$:
\begin{acode}
\afun{f}{result,i}{
  \aifelse{i>0}{
    f(result\cdot i,i-1)
  }{result}
}\\
f(1,n)
\end{acode}

\subsection{Optimization}

If $f$ is tail-recursive, a interpreter/compiler may turn $f$ into the corresponding while-loop, which usually yields much more efficient code.
Note that this is one of the rare situations, where the complexity of an algorithm depends on the interpreter/compiler.
For example, Java does not do it; many C compilers do.

The reason for the efficiency gain is the following.
Both the recursive function and the while-loop must pass information from one iteration to the next.
The while-loop uses assignments to mutable variables; because the mutable variables reside in the same memory location for the current and the next iteration, no physical passing of data is needed.
But in a recursive function, the recursive function call (like any other function call) allocates new memory locations and then copies the function arguments into them.
This overhead does not change the $\Theta$-class but can still be substantial.

Moreover, many interpreters/compilers allocate only a fixed amount of memory for the stack (see Sect.~\ref{sec:ad:callstack}).
For large function arguments, a recursive function may create so many nested function calls that it exhausts the available stack space (causing a \emph{stack overflow error}).
Tail-recursive functions are a special case where this danger can be averted by optimizing them into while-loops.

The exact way in which tail-recursion optimization happens is up to the interpreter/compiler.
Usually no new stack frame is allocated for the recursive call---instead the current stack frame is reused.
That works out because---due to the call being tail-recursive---the variable values stored in the current stack frame will never be used again and can therefore be safely overridden.

\subsection{The Call Stack and Stack Frames}

Most interpreters/compilers use a stack to keep track of the nesting of function calls.
This data structure is usually called the \textbf{call stack} (or just \emph{the stack}), and its elements are called \textbf{stack frames}.

When a function call $f(t_1,\ldots,t_n)$ of a function $f(x_1,\ldots,x_n)$ is processed,
\begin{compactitem}
 \item a new stack frame is created containing at least
   \begin{compactitem}
     \item the variable definitions $x_1:=t_1, \ldots, x_n:=t_n$
     \item the current program counter (i.e., the position of the next statement to be executed)
   \end{compactitem}
 \item the frame is pushed onto the stack
 \item execution continues with the body of $f$.
\end{compactitem}

When $f$ returns, the stack frame of $f$ (which is at the top of the call stack now) is popped from the stack, and execution continues at the position where $f$ was called.
