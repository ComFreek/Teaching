\section{Aspects of Heterogeneous Knowledge}

\section{Overview of This Course}

\subsection{Structure}

Während viele dieser Aspekte und Sprachen in eigenen Vorlesungen detailliert behandelt werden, greift WuV sie universell und vergleichend aus der Perspektive von Wissen auf. Ziel ist die grundlegenden Konzepte und ihre Ausprägungen in den diversen hoch-spezialisierten Sprachen und System zu verstehen. Die Übung vertieft dies im praktischen Umgang mit state-of-the-art Software-Systemen für die jeweiligen Aspekte. Einen besonderen Schwerpunkt legt WuV dabei auf die Gemeinsamkeiten, Unterschiede und Integration der Ansätze sowie die Interoperabilität der verschiedenen Systeme.

 Dazu hat die Informatik mehrere Aspekte von Wissen erkannt und spezielle Wissensrepräsentationssprachen entwickelt, die sich im Laufe der Zeit sehr stark spezialisiert und auseinander-entwickelt haben. Dies beinhaltet insbesondere Ontologiesprachen und Linked Data, Programmiersprachen und Algorithmen, Datenbeschreibungssprachen und Daten-Banken, Logik und Beweise sowie natürliche Sprache und informelle Dokumente.


\subsection{Exercises and Running Example}
