\section{Overview}

Let us assume we have a semantics for our syntax.
We again write $l$ for the syntax, $L$ for the semantic, and $\sem{-}$ for the translation function.

We can now use the semantics to answer questions asked in the syntax.
Here we use the syntax to phrase a question and the semantics to determine the answer.

We call this \emph{querying}.
Contrary to standard practice, we will use that word in a very broad sense that covers all aspects.
It is more common to use the word only for concretized querying, where SQL has been developed, which has shaped many intuitions about querying.

Usually, querying requires the syntax to designate some non-terminals as \emph{propositional}.
A non-terminal is propositional if the semantics can make its words true.
Without a notion of propositions, it is impossible to define what questions and answers even are.

\begin{definition}[Propositions]
A context-free \textbf{syntax with propositions} is a context-free syntax with some designated non-terminal symbols.

A \textbf{semantics with theorems} is one that additionally defines some propositions to be theorems.
We write $\vdash F$ if $F$ is a theorem.
\end{definition}

That definition does not mean that any kind of logic is needed for querying.
Many languages use highly restricted notions of propositions that would not generally be considered as logic.
For example, languages might use equalities between objects or even equalities between certain objects as the only propositions.
The following table gives an overview:

\begin{center}
\begin{tabular}{l|l|l}
aspect & typical propositions & proposition operators\\
\hline
ontology language & assertions, concept equality/subsumption &\\
programming language & equality for some types &  boolean operators\\
database language & equality for base types &  boolean operators \\
logic & equality for all types & boolean operators, quantifiers\\
natural language & sentences & and, or, some, every, \ldots\\
\end{tabular}
\end{center}

Often the development of querying for a language leads to the discovery of omissions in the syntax: certain objects that are helpful to ask questions were omitted from the syntax because they were not needed to describe the data.
Then sometimes the syntax is extended with non-terminals or productions that seem like dead code: they are not needed for or not allowed in the official data.
The following table gives some examples:

\begin{center}
\begin{tabular}{l|l}
aspect & typical extensions\\
\hline
ontology language & conjunction of assertions \\
programming language & quantifiers\\
database language & membership in a table \\
logic & (already tries to capture all possible propositions)\\
natural language & (already captures all possible propositions) \\
\end{tabular}
\end{center}

\begin{example}[Propositions in BOL]
The obvious choice of propositions for BOL are the formulas.

In Rem.~\ref{fig:bol}, we mentioned that the BOL syntax from Fig.~\ref{rem:bol:ass} had some redundant parts that were grayed out.
Assertions are needed for writing ontologies only in such that they behave like axioms, i.e., they are automatically true.
But for querying BOL, we also need them to behave like formulas so that we can use them as questions, i.e., we must allow them to be true or false.

Moreover, it is common to also allow conjunctions.
Therefore, the BOL propositions are the conjunctions of formulas.
\end{example}

In the sequel, we will use each of the four kinds of semantics to 
% add evidence: proof/model, proofs+completeness proof, trace, denn-Satz
\begin{center}
\begin{tabular}{lllll}
Section & builds on Section & query & result\\
\hline
\ref{sec:bolquery:ded}  & \ref{sec:bolsem:ded}  & deduction & proposition & yes/no \\
\ref{sec:bolquery:conc} & \ref{sec:bolsem:conc} & concretization & proposition with free variables & true ground instances \\
\ref{sec:bolquery:comp} & \ref{sec:bolsem:comp} & computation & term & value \\
\ref{sec:bolquery:narr} & \ref{sec:bolsem:narr} & narration & question & answer \\
\end{tabular}
\end{center}


\begin{remark}[Meta-Level Questions]
Finally, any semantics admits a meta-level where additional questions can be asked.
Examples are asking for the consistency of a theory or the equivalence of two theories/programs/queries.
At the next-higher meta-level, we can ask about the completeness of a semantics or the equivalence of two semantics (of which completeness is a special case).
These meta-questions can usually not be expressed in the syntax, and we do not consider them a part of querying here.
But it is worth mentioning that the need to use yet another language (a meta-language) to ask these questions can be annoying, and some advancements in language design are about trying to integrate them into the syntax.
For example, reflection is the process of representing a language in itself so that the language can talk about itself.
That way meta-questions become regular questions.
\end{remark}


%%%%%%%%%%%%%%%%%%%%%%%%%%%%%%%%%%%%%%%%%%%
\section{Deductive Querying}\label{sec:bolquery:ded}

\subsection{Method}

We assume that $l$ is a syntax with propositions and that $L$ is a logic (and thus in particular has propositions) whose semantics has theorems.

It is not guaranteed $\sem{-}$ translates $l$-propositions to $L$-propositions.
If not, we assume there is some operation $\truelift$ in $L$ that we can use to lift the translations of $l$-propositions to $L$-propositions.

A \textbf{deductive query} consists of a proposition.
The answer to the query is yes or no.

Thus, the deductive semantics must determine which propositions are theorems, i.e., whether $\vdash_L\truelift(\sem{F})$.
This is usually done in one of two ways.

\textbf{Proof theory} uses a calculus for $L$ to derive true propositions.
Thus, we can say that an $l$-proposition $F$ is true if the calculus derives $\truelift(\sem{F})$.
Accordingly, if $L$ has a negation operator $\neg$, we can say that $F$ is false if the calculus derives $\truelift(\neg\sem{F})$.

\textbf{Model theory} uses a second deductive semantics, namely a translation from $L$ to an even stronger deductive language $M$, usually some form of set theory.
Then, we can say that $l$-propositions are true if the composition of the two translations yields a true $M$-proposition.

Either way, it is determined whether to answer an $l$-proposition with yes or no.

\subsection{Challenges}

\paragraph{Consistency}
The $L$-calculus might derive both $F$ and $\neg F$.
In that case $L$ is inconsistent and usually every formula.
We usually assume $L$ to be consistent even though we do not always prove that.

\paragraph{Decidability}
Deductive semantics is usually undecidable, i.e., there is no algorithm that takes in $F$
and always returns yes or no in finite time. 

Therefore, deductive querying is very difficult in general.
One has to run heuristics (theorem provers) to see if a proof of $F$ or $\neg F$ can be found.

A common compromise is to allow only a restricted set of propositions as queries for which decision procedures exist.
However, it can be tricky to find good restrictions, especially if the syntax allows for function symbols and equality.

For example, SFOL is undecidable.
But many fragments of SFOL are decidable, such as propositional logic and various fragments in between.

When giving a deductive semantics into SFOL, it is therefore important to check whether the image of $\sem{-}$ falls inside a decidable fragment.
This is typically the case for ontology languages.

\paragraph{Completeness}
Deductive semantics is usually incomplete, i.e., there are unanswered questions.
More precisely, the $L$-calculus typically derives $F$ for some propositions, $\neg F$ for some, but neither for some others.
The third kind of proposition cannot be answered by the semantics.

\begin{remark}
The work ``complete'' is used for two different things in logic.

Firstly, it can be a relation between two semantics, typically proof theory and a model theory.
That is the dominant meaning of the word as in, e.g., the completeness theorem for SFOL and G\"odel's incompleteness theorem.

Secondly, it can mean that a logic proves or disproves every proposition, i.e., there is no $F$ such that neither $F$ nor $\neg F$ are derivable.
That is the sense we use above.
This kind of completeness rarely holds, usually only in very restricted circumstances.
\end{remark}

Decidability and completeness are essentially the same problem.
Specifically, if completeness holds, we already obtain a decision procedure for the logic: to decide the truth of $F$, enumerate all proofs until a proof of $F$ or $\neg F$ is found.
Vice versa, if we have a sound decision procedure, running it on $F$ will prove either $F$ or $\neg F$.

\paragraph{Efficiency}
Independent of whether the semantics is complete/decidable, theorem proving is typically very expensive.

Therefore, in addition to identifying decidable fragments of a logic, it is desirable to identify \emph{efficiently} decidable fragments.
Typically, a semantics meant for efficient practical querying aims for polynomially decidable fragments.
This is the case for very simple ontology languages.
But it can quickly become exponential if the language of propositions becomes more expressive.

%%%%%%%%%%%%%%%%%%%%%%%%%%%%%%%%%%%%%%%%%%%
\section{Concretized Querying}\label{sec:bolquery:conc}
%  SQL, SPARQL

\subsection{Method}

We make the same assumptions as for deductive querying.
We further assume that the semantics is compositional.

We define
\begin{compactitem}
\item a concretized query is an $l$-proposition $p$ in context $\Gamma$
\item a \emph{single} result is a
 \begin{compactitem}
 \item a substitution $\vdash_l \gamma:\Gamma$
 \item such that $\vdash_L \truelift \sem{p[\gamma]}$
 \end{compactitem}
\item the \emph{result set} is the set of all results
\end{compactitem}

\subsection{Challenges}

\paragraph{Open World}

\paragraph{Infinity of Results}

\paragraph{Back-Translation of Results}


%%%%%%%%%%%%%%%%%%%%%%%%%%%%%%%%%%%%%%%%%%%
\section{Computational Querying}\label{sec:bolquery:comp}

\subsection{Method}

\subsection{Challenges}

\paragraph{Infinite Types}

\paragraph{Termination}

\paragraph{Confluence}

%%%%%%%%%%%%%%%%%%%%%%%%%%%%%%%%%%%%%%%%%%%
\section{Narrative Querying}\label{sec:bolquery:narr}

\subsection{Method}

\subsection{Challenges}

\paragraph{Natural Language Understanding}

\paragraph{Implementing Common Sense}

\paragraph{Consistency of Knowledge Base}

%\subsection{Equivalence of Semantics}

% open vs. closed world; BOL and FOL are open; SQL, comp., and Herbrand model close the world


%%% Local Variables:
%%% mode: latex
%%% TeX-master: "WuV_notes"
%%% End:
