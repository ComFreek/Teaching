\section{General Principles}

\paragraph{Motivation}
An ontology is an abstract representation of the main concepts in some domain.
Here \emph{domain} refers to any area of the real world such as mathematics, biology, diseases and medications, human relationships, etc.
Many examples can be found at \url{https://bioportal.bioontology.org/}, including the Gene ontology one of the biggest.

Contrary to the other four aspects, ontological knowledge representations do not aim at capturing the entire semantics of the domain objects.
Instead, they focus on defining unique identifiers for the those objects and describing some of their properties and relations to each other.

We use the word \textbf{ontologization} to refer to the process of organizing the knowledge of a domain in ontologies.

Ontologies are most valuable when they are \emph{standardized} (either sanctioned through a formal body or a quasi-standard because everyone uses it).
A standard ontology allows everybody in the domain to use the identifiers defined by the ontology in a way that avoids misunderstandings.
Thus, in the simplest form, an ontology can be seen as a dictionary defining the technical terms of a domain.
For example, the Gene ontology defines identifier \texttt{GO:0000001} to have the formal name "mitochondrion inheritance" and the informal definition "The distribution of mitochondria, including the mitochondrial genome, into daughter cells after mitosis or meiosis, mediated by interactions between mitochondria and the cytoskeleton.".

\paragraph{Ontology Languages}
An ontology is written in \textbf{ontology language}.
Common ontology languages are
\begin{compactitem}
 \item description logics such as ALC,
 \item the W3C ontology language OWL, which is the standard ontology languages of the semantic web,
 \item the entity-relationship model, which focuses on modeling rather than formal syntax,
 \item modeling languages like UML, which is the main ontology language used in software engineering.
\end{compactitem}

Ontology languages are not committed to a particular domain --- in the Tetrapod model, they correspond to programming languages and logics, which are similarly uncommitted.
Instead, an ontology language is a formal language that standardizes the syntax of how ontologies can be written as well as their semantics.

\paragraph{Ontologies}
The details of the syntax vary between ontology languages.
But as a general rule, every \textbf{ontology} declares
\begin{compactitem}
 \item \textbf{individual} --- concrete objects that exist in the real world, e.g., "Florian Rabe" or "WuV"
 \item \textbf{concept} --- abstract groups of individuals, e.g., "instructor" or "course"
 \item \textbf{relation} --- binary relations between two individuals, e.g., "teaches"
 \item \textbf{properties} --- binary relations between an individuals and a concrete value (a number, a date, etc.), e.g., "has-credits"
 \item \textbf{concept assertions} --- the statement that a particular individual is an instance of a particular concept
 \item \textbf{relation assertions} --- the statement that a particular relation holds about two individuals
 \item \textbf{property assertions} --- the statement that a particular individual has a particular value for a particular property
 \item \textbf{axioms} --- statements about relations between concepts, typically in the form subconcept of statements like "instructor" $\sqsubseteq$ "person"
\end{compactitem}

All assertions can be understood and spoken as subject-predicate-object \textbf{triples} as follows:
\begin{center}
\begin{tabular}{l|lll}
Assertion & \multicolumn{3}{c}{Triple} \\
          & Subject & Predicate & Object \\
\hline
concept assertion  & "Florian Rabe" & \texttt{is-a} & "instructor" \\
relation assertion & "Florian Rabe" & "teaches" & "WuV" \\
property assertion & "WuV" & "has credits" & 7.5 \\
\end{tabular}
\end{center}
This uses a special relation \texttt{is-a} between individuals and concepts.
Some languages group \texttt{is-a} with the other binary relations between individuals for simplicity although it is technically a little different.

The possible values of properties must be fixed by the ontology language.
Typically, it includes at least standard types such as integers, floating point numbers, and strings.
But arbitrary extensions are possible such as dates, RGB-colors, lists, etc.
In advanced languages, it is possible that the ontology even introduces its own basic types and values.

Ontologies are often divided into two parts:
\begin{compactitem}
 \item The \textbf{abstract} part contains everything that holds in general independent of which individuals: concepts, relations, properties, and axioms.
 It describes the general rules how the worlds works without committing to a particular set of inhabitants of the world.
 This part is commonly called the \textbf{TBox} (T for terminological).
 \item The \textbf{concrete} part contains everything that depends on the choice of individuals: individuals and assertions.
 It populates the world with inhabitants.
 This part is commonly called the \textbf{ABox} (A for assertional).
\end{compactitem}

A separate division into two parts is the following:
\begin{compactitem}
 \item The \textbf{signature} part contains everything that introduces a \textbf{named entity}: individuals, concepts, relations, and properties.
 \item The \textbf{theory} part contains everything that describes which statements about the named entities are true: assertions and axioms.
\end{compactitem}


\paragraph{Synonyms}
Because these principles pervade all formal languages, many competing synonyms are used in different domains.
Common synonyms are:
\begin{center}
\begin{tabular}{l|llll|l}
 Here       & OWL      & Description logics & ER model & UML & semantics via logics\\
\hline
 individual & instance & individual & entity & object, instance & constant\\
 concept    & class    & concept &  entity-type & class & unary predicate\\
 relation   & object property & role & role & association & binary predicate \\
 property   & data property   & (not common) & attribute & field of base type & binary predicate\\
\end{tabular}
\end{center}

In particular, the individual-concept relation occurs everywhere and is known under many names:
\begin{center}
\begin{tabular}{l|ll}
 domain & individual & concept \\
\hline
type theory, logic & constant, term & type \\
set theory  & element & set \\
database    & row & table \\
philosophy\footnote{as in \url{https://plato.stanford.edu/entries/object/}} & object & property \\
grammar & proper noun & common noun \\
\end{tabular}
\end{center}

%%%%%%%%%%%%%%%%%%%%%%%%%%%%%%%%%%%%%%%%%%%%%
\section{A Basic Ontology Language}

\begin{figure}[hbt]
\begin{commgrammar}
\gcomment{Ontologies}\\
\gprod{O}{\rep{D}}{}\\
\gcomment{Declarations}\\
\gprod{D}{\kw{individual}\; \ID}{atomic individual}\\
\galtprod{\kw{concept}\; \ID}{atomic concept}\\
\galtprod{\kw{relation}\; \ID}{atomic relation}\\
\galtprod{\kw{property}\; \ID: T}{atomic property}\\
\galtprod{I\; \texttt{is-a}\; C}{concept assertion}\\
\galtprod{I\; R\; I}{relation assertion}\\
\galtprod{I\; P\; V}{property assertion}\\
\galtprod{F}{other axioms}\\
\gcomment{Formulas}\\
\gprod{F}{C \Equiv C}{concept equality}\\
\galtprod{C \sqsubseteq C}{concept subsumption}\\
\gcomment{Individual expressions}\\
\gprod{I}{\ID}{atomic individuals}\\
\gcomment{Concept expressions}\\
\gprod{C}{\ID}{atomic concepts}\\
\galtprod{C \sqcup C}{union of concepts}\\
\galtprod{C \sqcap C}{intersection of concepts}\\
\galtprod{\forall R.C}{universal relativization}\\
\galtprod{\exists R.C}{existential relativization}\\
\galtprod{\dom R}{domain of a relation}\\
\galtprod{\rng R}{range of a relation}\\
\gcomment{Relation expressions}\\
\gprod{R}{\ID}{atomic relations}\\
\galtprod{R \cup R}{union of relations}\\
\galtprod{R \cap R}{intersection of relations}\\
\galtprod{R ; R}{composition of relations}\\
\galtprod{R^*}{transitive closure of a relation}\\
\galtprod{R^{-1}}{dual relation}\\
\galtprod{\Delta_C}{identity relation of a concept}\\
\gcomment{Property expressions}\\
\gprod{P}{\ID}{atomic properties}\\
\gcomment{Identifiers}\\
\gprod{\ID}{\text{alphanumeric string}}{}\\
\gcomment{Basic types and values}\\
\gprod{T}{\itg \alt \float \alt \bool \alt \strg}{types}\\
\gprod{T}{\text{(omitted)}}{values}
\end{commgrammar}
\caption{Grammar of BOL}\label{fig:bol}
\end{figure}

\clearpage

We could study practical ontology languages like ALC or OWL now.
But those feature a lot of other details that can block the view onto the essential parts.
Therefore, we first define a basic ontology language ourselves in order to have full control over the details.

\subsection{Syntax}

\begin{definition}[Syntax of BOL]
A BOL-ontology is given by the grammar in Fig.~\ref{fig:bol}.
It is well-formed if
\begin{compactitem}
 \item no identifier is declared twice,
 \item every property assertion assigns a value of the type required by the property declaration,
 \item every reference to an atomic individual/concept/relation/property is declared as such.
\end{compactitem}
\end{definition}

The above grammar exhibits some general structure that we find throughout formal KR languages.
In particular, an ontology consists of \textbf{named declarations} of four different kinds of entities as well as some assertions and axioms about them.
Each entity declaration clarifies which kind it is (in our case by starting with a keyword) and introduces a new entity identifier.
For each kind, there are complex expressions.
These are anonymous and built inductively; their base cases are references to the corresponding identifiers.
Sometimes (in our case: individuals and properties), the references are the only expressions of the kind.
Sometimes (in our case: concepts and relations), there can be many productions for complex expressions.
The complex expressions are used to build axioms; in our case, these are the three kinds of assertions and other formulas.

\subsection{Deductive Semantics}

We give a semantics of BOL as an example of a semantics by translation.
We fix one language that we have already understood and define an interpretation function that maps all complex expression of the syntax into the semantic language.
Specifically, we give a deductive/logical semantics, i.e., the semantic language is a logic.

For simple ontology languages like BOL, ALC, OWL, etc., it is common to use first-order logic (FOL) as the semantic language.
More specifically, we use SFOL, the typed variant of FOL with
\begin{definition}[Semantic of BOL]
Every BOL-ontology $O$ is interpreted as a FOL-theory $\sem{O}$ according to Fig.~\ref{fig:bolsem} where we assume that $\sem{O}$ contains
\begin{compactitem}
 \item a type $\iota$ (for individuals),
 \item additional types and constants corresponding to base types and values of BOL.
\end{compactitem}
\end{definition}

Like with the syntax, we can observe some general principles.
Every BOL-declaration is translated to a FOL declaration for the same name, and ontologies are translated declaration-wise.
For every kind of complex expression, there is one inductive function mapping BOL-expressions to FOL-expressions.
The base cases of references to declared identifiers are translated to themselves, i.e., to the identifiers of the same name declared in the FOL theory.
The other cases are compositional: every case for a complex expression recurses only into the semantics of the direct subexpressions.

\begin{figure}
\begin{center}
\begin{tabular}{l|l}
BOL Syntax $X$ & Semantics $\sem{X}$ in FOL\\
\hline
\hline
ontology & FOL theory \\
$D_1,\ldots,D_n$ & $\sem{D_1},\ldots,\sem{D_n}$ \\
\hline
BOL declaration & FOL declaration \\
\kw{individual}\,$i$ & nullary function symbol $i:\iota$ \\
\kw{concept}\,$i$  & unary predicate symbol $i\sq\iota$ \\
\kw{relation}\,$i$ & binary predicate symbol $i\sq\iota\times \iota$ \\
\kw{property}\,$i:T$ & binary predicate symbol $i\sq\iota\times T$ \\
$I\; \texttt{is-a}\; C$ & axiom $\sem{C}(\sem{I})$\\
$I_1\; R\; I_2$ & axiom $\sem{R}(\sem{I_1},\sem{I_2})$\\
$I\; P\; V$ & axiom $\sem{P}(\sem{I},\sem{V})$\\
$F$ & axiom $\sem{F}$\\
\hline
Formula & Formula without free variables\\
$C_1 \Equiv C_2$ & $\forall x:\iota.\sem{C_1}(x)\Leftrightarrow \sem{C_2}(x)$\\
$C_1 \sqsubseteq C_2$ & $\forall x:\iota.\sem{C_1}(x)\impl \sem{C_2}(x)$\\
\hline
Individual & Terms of type $\iota$ \\
$i$ & $i$ \\
\hline
Concept & Formula with free variable $x:\iota$\\
$i$ & $i(x)$\\
$C_1 \sqcup C_2$ & $\sem{C_1}(x)\vee\sem{C_2}(x)$\\
$C_1 \sqcap C_2$ & $\sem{C_1}(x)\wedge\sem{C_2}(x)$\\
$\forall R.C$    & $\forall y:\iota.\sem{R}(x,y)\impl \sem{C}(y)$\\
$\exists R.C$    & $\exists y:\iota.\sem{R}(x,y)\wedge \sem{C}(y)$\\
$\dom\, R$ & $\exists y:\iota.\sem{R}(x,y)$\\
$\rng\, R$ & $\exists y:\iota.\sem{R}(y,x)$\\
\hline
Relation & Formula with free variables $x:\iota,y:\iota$\\
$i$ & $i(x,y)$\\
$R_1 \cup R_2$ & $\sem{R_1}(x,y)\vee \sem{R_2}(x,y)$\\
$R_1 \cap R_2$ & $\sem{R_1}(x,y)\wedge \sem{R_2}(x,y)$\\
$R_1 ; R_2$ & $\exists m:\iota.\sem{R_1}(x,m)\wedge \sem{R_2}(m,y)$\\
$R^{-1}$          & $\sem{R}(y,x)$\\
$R^*$          & (tricky, omitted)\\
$\Delta_C$     & $x\doteq y\wedge \sem{C}(x)$\\
\hline
Property of type $T$ & Formula with free variables $x:\iota,y:T$\\
$i$ & $i(x,y)$\\
\end{tabular}
\caption{Interpretation Function for BOL into FOL}\label{fig:bolsem}
\end{center}
\end{figure}

\subsection{Concretized Semantics}

We give an alternative semantics using a semantic language for concrete data.
Specifically we focus on the SQL database language.

Even though this is a very different knowledge aspect, the general principles of the semantics are the same:
Every BOL-declaration is translated to an SQL declaration, and ontologies are translated declaration-wise.
For every kind of complex expression, there is one inductive function mapping BOL-expressions to SQL-expressions.

In SQL, we can nicely see the difference between declarations and expressions: the former are translated to side effect-ful statements, the latter to side effect-free queries.

\begin{definition}[Concretized Semantic of BOL]
Every BOL-ontology $O$ is interpreted as an SQL-theory $\sem{O}$ according to Fig.~\ref{fig:bolsemsql} where we assume that
\begin{compactitem}
 \item the underlying database supports a type $ID$ of identifiers as well as all base types and values of BOL,
 \item $\sem{O}$ starts with CREATE TABLE individuals (id ID, name string), where the id field is unique and automatically generated when inserting values.
\end{compactitem}
\end{definition}

\begin{figure}
\begin{center}
\begin{tabular}{l|l}
BOL Syntax $X$ & Semantics $\sem{X}$ in SQL\\
\hline
\hline
ontology & SQL statements \\
$D_1,\ldots,D_n$ & $\sem{D_1},\ldots,\sem{D_n}$ \\
\hline
BOL declaration ($I$, $C$, $R$ atomic) & SQL statement \\
\kw{individual}\,$i$ & INSERT INTO individuals (name) VALUES ($i$) \\
\kw{concept}\,$i$  & CREATE TABLE $i$ (id ID)\\
\kw{relation}\,$i$ & CREATE TABLE $i$ (subject ID, object ID) \\
\kw{property}\,$i:T$ & CREATE TABLE $i$ (subject ID, object $T$) \\
$I\; \texttt{is-a}\; C$ & INSERT INTO $C$ VALUES ($\sem{I}$)\\
$I_1\; R\; I_2$ & INSERT INTO $R$ (subject, object) VALUES ($\sem{I_1}$, $\sem{I_2}$)\\
$I\; P\; V$ & INSERT INTO $P$ (subject, object) VALUES ($\sem{I}$, $V$)\\
$F$ & consistency check, omitted\\
\hline
Formula & Formula without free variables\\
$C_1 \Equiv C_2$ & check equality of query results\\
$C_1 \sqsubseteq C_2$ & check subset of query results\\
\hline
Individual & an identifier from the table individuals \\
$i$ & SELECT id FROM individuals WHERE name=$i$ \\
\hline
Concept & SQL query for one-column table\\
$i$ & SELECT * FROM $i$\\
$C_1 \sqcup C_2$ & $\sem{C_1}$ UNION $\sem{C_2}$\\
$C_1 \sqcap C_2$ & $\sem{C_1}$ INTERSECT $\sem{C_2}$\\
$\forall R.C$    & $\sem{R}$ ??? $\sem{C}$\\ 
$\exists R.C$    & $\sem{R}$ ??? $\sem{C}$\\ % SELECT DISTINCT subject FROM $\sem{R}$, $\sem{C}$ WHERE object=id
$\dom\, R$ & SELECT subject FROM $\sem{R}$\\
$\rng\, R$ & SELECT object FROM $\sem{R}$\\
\hline
Relation & SQL query for two-column table\\
$i$ & SELECT * FROM $i$\\
$R_1 \cup R_2$ & $\sem{R_1}$ UNION $\sem{R_2}$\\
$R_1 \cap R_2$ & $\sem{R_1}$ INTERSECT $\sem{R_2}$\\
$R_1 ; R_2$ & $\sem{R_1}$ ??? $\sem{R_2}$ \\
%SELECT DISTINCT l.subject, r.object FROM $\sem{R_1}$ AS l, $\sem{R_2}$ AS r \\
%            & \tb\tb WHERE l.object = r.subject\\
$R^{-1}$          & SELECT object, subject FROM $\sem{R}$\\
$R^*$          & ???\\
$\Delta_C$     & SELECT id AS subject, id AS object FROM $\sem{C}$\\
\hline
Property of type $T$ & SQL query for two-column table\\
$i$ & SELECT * FROM $i$\\
\end{tabular}
\caption{Interpretation Function for BOL into SQL}\label{fig:bolsemsql}
\end{center}
\end{figure}


%%%%%%%%%%%%%%%%%%%%%%%%%%%%%%%%%%%%%%%%%%%%%
\section{Representing Ontologies as Triples}

It is common to represent an entire ontology as a set of subject-predicate-object triples.
That makes handling ontologies very simple and efficient.
This is the preferred representation of the semantic web.

However, while, e.g., relation assertions are naturally triples, not all declarations are, and some tricks may be necessary.

\paragraph{Inferring the Entity Declarations}
The entity declarations are not naturally triples.
But we can usually infer them from the assertions as follows: any identifier that occurs in a position where an entity of a certain kind is expected is assumed to be declared as an entity for that kind.

For example, the individuals are what occurs as the subject of a concept, relation, or property assertion or as the object of a relation assertion.
It is conceivable that there are individuals that occur in none of these.
But that is unusual because they would be disconnected from everything in the ontology.

If we give TBox and ABox together, this inference approach usually works well.
But if we only give a TBox, this would often not allow inferring all entities.
The only place where they could occur in the TBox is in the axioms, and it is quite possible to have concept, relation, and property declarations that are not used in the axioms.
In fact, it is not unusual not to have any axioms.

\paragraph{Special Predicates}
To turn declarations into triples, we can use reflection, i.e., the process of talking about our language constructs as if they were data.

Reflection requires introducing some built-in entities that represent the features of the language.
In the semantic web area, this is performed using the following entities:
\begin{compactitem}
 \item "rdfs:Resource": a built-in concept of which all individuals are an instance and thus of which every concept is a subconcept
 \item "rdf:type": a special predicate that relates an entity to its type:
  \begin{compactitem}
   \item an individual to its concept (corresponding to \texttt{is-a} above)
   \item other entities to their special type (see below)
  \end{compactitem}
 \item "rdfs:Class": a special class to be used as the type of classes
 \item "rdf:Property": a special class to be used as the type of properties
 \item "rdfs:subClassOf": a special relation that relates a subconcept to a superconcept
% \item "rdfs:subPropertyOf": a special relation that relates a relation to one that it implies
 \item "rdfs:domain": a special relation that relates a relation to the concepts of its subjects
 \item "rdfs:range": a special relation that relates a relation/property to the concept/type of its objects
\end{compactitem}
Here "rdf" and "rdfs" refer to the RDF (Resource Description Framework) and RDFS (RDF Schema) namespaces, which correspond to W3C standards defining those special entities.

Thus, we can represent many and in particular the most important entity declarations as triples:
\begin{center}
\begin{tabular}{l|lll}
Assertion & \multicolumn{3}{c}{Triple} \\
          & Subject & Predicate & Object \\
\hline
individual & individual & "rdf:type" & "rdfs:Resource" \\
concept  & concept & "rdf:type" & "rdf:Class" \\
relation & relation & "rdf:type" & "rdf:Property" \\
property & property & "rdf:type" & "rdf:Property" \\
concept assertion  & individual & "rdf:type" & concept \\
relation assertion & individual & relation & individual \\
property assertion & individual & property & value \\
\hline
\multicolumn{4}{l}{for special forms of axioms}\\
$c\sqsubseteq d$ & $c$ & "rdfs:subClassOf" & $d$ \\
%$r\sqsubseteq s$ & $r$ & "rdfs:subPropertyOf" & s \\
$\dom\,r\Equiv c$ & $r$ & "rdfs:domain" & $c$ \\
$\rng\, r\Equiv c$ & $r$ & "rdfs:range" & $c$ \\
\end{tabular}
\end{center}

This is subject to the restriction that only atomic concepts and relations can be handled.
For example, only concept assertions can be handled that make an individual an instance of an \emph{atomic} concept.
This is particularly severe for axioms, where complex expressions occur most commonly in practice.
Here, the special relations allow capturing the most common axioms as triples.

\paragraph{Problems}
Reflection is subtle and can easily lead to inconsistencies.
We can see this in how the approach of RDF(S) special entities breaks the semantics via FOL.

For example, it treats classes both as concepts (when they occur as the object of a concept assertion) and as individuals (when they occur as subject or object of a "rdfs:subClassOf" relation assertion).
Similarly, "rdfs:Class" is used both as an individual and as a class.
In fact, the standard prescribes that "rdfs:Class" is an instance of itself.

In practice, this is handled pragmatically by using ontologies that make sense.
A formal way to disentangle this is to assume that there are two variants of "rdfs:Class", one as an individual and one as a class.
The translation must then translate "rdfs:Class" differently depending on how it is used.

It would be better if RDFS were described in a way that is consistent under the implicitly intended FOL semantics.
But the more pragmatic approach has the advantage of being more flexible.
For example, being able to treat every class, relation, or property also as an individual makes it easy to annotate metadata to them.
Metadata is a set of properties such as "rdfs:seeAlso" or "owl:versionInfo", whose subjects can be any entity.