\section{The OWL Language}

\paragraph{Abstract Syntax and Semantics}
Due to their central in knowledge representation, a number of languages for ontology writing exist.
Most importantly, the syntax and semantics of OWL, including several sublanguages, are standardized by the W3C.

OWL includes a number of built-in special entities.
Most importantly, "owl:Thing" corresponds to "rdfs:Resource" as the concept of all individuals.

\paragraph{Concrete Syntax}
Several concrete syntaxes have been defined and are commonly used for OWL.
APIs for OWL implement the abstract syntax along with good support for reading/writing ontologies in any of the concrete syntaxes.

\section{The Protege Tool}

A widely used tool for writing ontologies in OWL is Protege\footnote{\url{https://protege.stanford.edu/}}.

To get started with Protege without getting confused, we need to continue understand how its key terminology maps to other contexts.
\begin{center}
\begin{tabular}{l|ll}
 Here       & Protege & Edited in WebProtege via \\
\hline
 individual & individual & listed in "Individuals" tab\\
 concept    & class   & listed in "Classes" tab  \\
 relation   & object property & listed in "Properties" tab\\
 property   & data property & listed in "Properties" tab\\
 concept assertion & Type & detail area of the individual in "Individuals" tab \\
 relation assertion & Relationship & detail area of the subject in "Individuals" tab \\
 property assertion & Relationship & detail area of the subject in "Individuals" tab \\
\end{tabular}
\end{center}

Protege's interface treats some parts of the ontology specially:
\begin{compactitem}
 \item The "Classes" tab organizes concepts using a tree view based on the subconcept relationship.
 Superclasses of a class can also be edited directed by listing parents.
 \item The "Properties" tab organizes properties using a tree view based on the subproperty (i.e., implication, subset) relationship.
 \item Axioms describing the domain and range of a property can be given directly in its details view.
\end{compactitem}

Note that classes can be in relationships with other classes as well even though that was not considered in the course so far.

\section{Exercise}

The topic of Exercise 1 is to use Protege to write an OWL ontology for a university.

Protege is a graphical editor for the abstract syntax of OWL.
Familiarize yourself with the various concrete syntaxes of OWL by writing an ontology that uses every feature once, downloading it in all available concrete syntaxes, and comparing those.

The minimal goal of the exercise session is to get a Hello World example going, at which point the task transitions into homework.
There will be no homework submission, but you will use your ontology throughout the course.

You should make sure you understand and setup the process in a way that supports you when you revisit and change your ontology many times throughout the semester.

Other than that, the task is deliberately unconstrained to mimic the typical situation at the beginning of a big project, where it is unclear what the ultimate requirements will be.