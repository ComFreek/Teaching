\section{Context-Free Syntax}

Abstractly, context-free syntax is specified using grammars.
Concretely, it is implemented using inductive types.

In the sequel, we will start with the standard definitions and then make a series of variation to each of these definitions until they become equivalent.
The intended equivalence is as follows:
\begin{center}
\begin{tabular}{l|l}
CFG & IDT \\
\hline
non-terminal & type \\
production & constructor \\
non-terminal on left of production & return type of constructor \\
non-terminals on right of production & arguments types of constructor \\
terminals on right of production & notation of constructor
\end{tabular}
\end{center}

%%%%%%%%%%%%%%%%%%%%%%%%%%%%%%
\subsection{Context-Free Grammars}

We start with the usual definition:

\begin{definition}[Context-Free Grammar]
Given a set $\Sigma$ of characters (containing the terminal symbols), a \textbf{context-free grammar} consists of
\begin{compactitem}
\item a set $N$ of names called \textbf{non-terminal symbols}
\item a set of \textbf{productions} each consisting of
 \begin{compactitem}
  \item an element of $N$, called the \textbf{left-hand side}\\
  \item a word over $\Sigma\cup N$, called the \textbf{right-hand side}
 \end{compactitem}
\end{compactitem}
\end{definition}

\begin{example}
Let $\Sigma=\{0,1,+,\cdot,\doteq,\leq\}$.
We give a grammar for arithmetic expressions and formulas about them:
\begin{commgrammar}
\gprod{E}{0}{}\\
\gprod{E}{1}{}\\
\galtprod{E+E}{}\\
\galtprod{E\cdot E}{}\\
\gprod{F}{E\doteq E}{}\\
\galtprod{E\leq E}{}\\
\end{commgrammar}
Here we use the BNF style of writing grammars, where the productions are grouped by their left-hand side and written with $\bbc$ and $\bnfalt$.
We have $N=\{E,F\}$.
\end{example}

First, we give a name to each production of a CFG:

\begin{definition}[Context-Free Grammar with Named Productions]
Given a set $\Sigma$ of characters (containing the terminal symbols), a \textbf{context-free grammar} consists of
\begin{compactitem}
\item a set $N$ of names called \emph{non-terminal symbols}
\item a set of \emph{productions} each consisting of
 \begin{compactitem}
  \item a name
  \item an element of $N$, called the \textbf{left-hand side}
  \item a word over $\Sigma\cup N$, called the \textbf{right-hand side}
 \end{compactitem}
\end{compactitem}
\end{definition}

\begin{example}
The grammar from above with names written to the right of each production
\begin{commgrammar}
\gprod{E}{0}{zero}\\
\gprod{E}{1}{one}\\
\galtprod{E+E}{sum}\\
\galtprod{E\cdot E}{product}\\
\gprod{F}{E\doteq E}{equality}\\
\galtprod{E\leq E}{lessOrEqual}\\
\end{commgrammar}
This is not common BNF anymore.
\end{example}

Then we add base types to the productions:

\begin{definition}[Context-Free Grammar with Named Productions and Base Types]
Given a set $\Sigma$ of characters (containing the terminal symbols) and a set $T$ of names (containing the base types allowed in productions), a \textbf{context-free grammar} consists of
\begin{compactitem}
\item a set $N$ of names called \emph{non-terminal symbols}
\item a set of \emph{productions} each consisting of
 \begin{compactitem}
  \item a name
  \item an element of $N$, called the \textbf{left-hand side}
  \item a word over $\Sigma\cup T\cup N$, called the \textbf{right-hand side}
 \end{compactitem}
\end{compactitem}
\end{definition}

The intuition behind base types is that we commonly like to delegate some primitive parts of the grammar to be defined elsewhere.
A typical example are literals such as numbers $0, 1, 2,\ldots$: We could give regular expression syntax for digit-strings.
Instead, it is nicer to just assume we have a set of base types that we can use to insert an infinite set of literals into the grammar.

\begin{example}
Let $Nat$ be the type of natural numbers and let $T=\{Nat\}$.
Then we can improve the grammar from above as follows:
\begin{commgrammar}
\gprod{E}{Nat}{literal}\\
\galtprod{E+E}{sum}\\
\galtprod{E*E}{product}\\
\gprod{F}{E\doteq E}{equality}\\
\galtprod{E\leq E}{lessOrEqual}\\
\end{commgrammar}
\end{example}

%%%%%%%%%%%%%%%%%%%%%%%%%%%%%%
\subsection{Inductive Data Types}

We start with the usual definition:

\begin{definition}[Inductive Data Type]
Given a set of names $T$ (containing the types known in the current context), An \emph{inductive data type} consists of
\begin{compactitem}
 \item a name, called the \textbf{type},
 \item a set of \textbf{constructors} each consisting of
 \begin{compactitem}
  \item a name
  \item a list of elements of $T$, called the \textbf{argument} types
 \end{compactitem} 
\end{compactitem}
\end{definition}

\newcommand{\cons}[2]{\mathtt{#1}\,\kw{of}\,\fold{*}{#2}}
\newcommand{\indtype}[2]{#1\,=\,\fold{\;\tb|\tb}{#2}}
\newcommand{\consnot}[3]{\mathtt{#1}\,\kw{of}\,\fold{*}{#2}\,\#\,#3}
\newcommand{\consn}[2]{\mathtt{#1}\,\#\,#2}

\begin{example}
Let $Nat$ be the type of natural numbers and $T=\{Nat\}$.
We give an inductive type for arithmetic expressions:
\[
\indtype{E}{\cons{literal}{Nat}, \cons{sum}{E,E}, \cons{product}{E,E}} \\
\]
Here we use ML-style notation for inductive data types, which separates constructors by $|$ and writes them as \texttt{name of argument-type-product}.
\end{example}

First we generalize to mutually inductive types:

\begin{definition}[Mutually Inductive Data Types]
Given a set $T$ of names (containing the types known in the current context), a family of \textbf{mutually inductive data type} consists of
\begin{compactitem}
 \item a set $N$ of names, called the \textbf{types},
 \item a set of \emph{constructors} each consisting of
 \begin{compactitem}
  \item a name
  \item an element of $I$, called the \textbf{return type}
  \item a list of elements of $N\cup I$, called the \textbf{argument} types
 \end{compactitem} 
\end{compactitem}
\end{definition}

\begin{example}
We extend the type definition from above by adding a second type for formulas.
Thus, $N=\{E,F\}$.
\[\mathll{
\indtype{E}{\cons{literal}{Nat}, \cons{sum}{E,E}, \cons{product}{E,E}} \\
\indtype{F}{\cons{equality}{E,E}, \cons{lessOrEqual}{E,E}}
}\]
\end{example}

%It may look $T$ and $\Sigma$ correspond to each other.
%But that is not true, we need to add them to each definition.

Then we add notations to the constructors:

\begin{definition}[Mutually Inductive Data Types with Notations]
Given a set $\Sigma$ of characters (containing the terminal symbols) and a set $T$ of names (containing the types known in the current context), a family of \textbf{mutually inductive data type with notations} consists of
\begin{compactitem}
 \item a set $N$ of names, called the \textbf{types},
 \item a set of \emph{constructors} each consisting of
 \begin{compactitem}
  \item a name
  \item an element of $N$, called the \textbf{return type}
  \item a list of elements of $T\cup N$, called the \textbf{argument} types
  \item a word over the alphabet $\Sigma\cup T\cup N$ containing the argument types in order and only elements from $\Sigma$ otherwise, called the \textbf{notation} of the constructor
 \end{compactitem} 
\end{compactitem}
\end{definition}

The intuition behind notations is that it can get cumbersome to write all constructor applications as $Name(arguments)$.
It is more convenient to attach a notation to  such as 

\begin{example}
We extend the type definitions from above by adding notations to each constructor.
We use the set $\Sigma=\{+,\cdot,\doteq,\leq\}$ as terminals in the notations.
\[\mathll{
\indtype{E}{\consnot{literal}{Nat}{Nat}, \consnot{sum}{E,E}{E+E}, \consnot{product}{E,E}{E\cdot E}} \\
\indtype{F}{\consnot{equality}{E,E}{E\doteq E}, \consnot{lessOrEqual}{E,E}{E\leq E}}
}\]
Here we write the constructors as \texttt{name of argument-type-product \# notation}.
It is easy to see that this has introduced redundancy: we can infer the argument types from the notation.
So we can just drop the argument types:
\[\mathll{
\indtype{E}{\consn{literal}{Nat}, \consn{sum}{E+E}, \consn{product}{E\cdot E}} \\
\indtype{F}{\consn{equality}{E\doteq E}, \consn{lessOrEqual}{E\leq E}}
}\]
\end{example}


\subsection{Merged Definition}

With the variation from above we have arrived at the following equivalence:

\begin{theorem}
Given a set $\Sigma$ of characters and a set $T$ of names, the following notions are equivalent:
\begin{compactitem}
\item a family of mutually inductive data types in the context of types $T$ with notations using characters from $\Sigma$,
\item a context-free grammar with named productions, terminal symbols from $\Sigma$, and base types $T$.
\end{compactitem}
\end{theorem}
\begin{proof}
The key idea is that
\begin{compactitem}
 \item the types and constructors of the former correspond to the non-terminals and productions of the latter
 \item for each constructor-production pair
  \begin{compactitem}
   \item the right-hand side of the latter corresponds to the notation of the former,
   \item the argument types of the former correspond to the non-terminals occurring on the right-hand side of the latter.
  \end{compactitem}
\end{compactitem}
\end{proof}

In implementations in programming languages, we often drop the notations.
Instead, those are handled, if needed, by special parsing and serialization functions.

However, in an implementation, it is often helpful to additionally give names to each argument of a production/constructor.
That yields the following definition:

\begin{definition}[Context-Free Syntax]
Given a set $\Sigma$ of characters and a set $T$ of names, a context-free syntax consists of
\begin{compactitem}
 \item a set $N$ of names, called the \textbf{non-terminals/types},
 \item a set of \emph{productions/constructors} each consisting of
 \begin{compactitem}
  \item a name
  \item an element of $N$, called the \textbf{left-hand side/return type}
  \item a sequence of objects, called the \textbf{right-hand side/arguments} which are one of the following
   \begin{compactitem}
    \item an element of $\Sigma$
    \item a pair written $(n:t)$ of a name $n$, called the \textbf{argument name}, and an element $t\in T\cup N$ called the \textbf{argument type}.
   \end{compactitem}
 \end{compactitem}
\end{compactitem}
\end{definition}

\begin{example}
Our example from above as a context-free syntax finally looks as follows:
\[\mathll{
\indtype{E}{\consn{literal}{(value:Nat)}, \consn{sum}{(left:E)+(right:E)}, \consn{product}{(left:E)\cdot (right:E)}} \\
\indtype{F}{\consn{equality}{(left:E)\doteq (right:E)}, \consn{lessOrEqual}{(left:E)\leq (right:E)}}
}\]
\end{example}


\section{Implementation}

Context-free syntax can be implemented systematically in all programming languages.
But, depending on the style of the language, they make drastically different.
We give the two most important paradigms as examples.

\subsection{Functional Programming Languages}

In a function programming language, inductive data types are a primitive feature.
However, notations and named arguments are not available.
So helper functions must be used.

The basic recipe is as follows:
\begin{compactitem}
\item The types and constructors (without the notations and named arguments) are implemented as family of mutually inductive data types.
\item For each argument of each constructor, a partial projective function is defined.
\item A set of mutually recursive string rendering functions are define, one for each constructor, that implement the notations.
\end{compactitem}

\begin{example}
We define our example syntax in ML.

First the inductive types (assuming a type $Nat$ already exists in the context):
\[\mathll{
\kw{data}\, \indtype{E}{\cons{literal}{Nat}, \cons{sum}{E,E}, \cons{product}{E,E}} \\
\kw{and}\,\indtype{F}{\cons{equality}{E,E}, \cons{lessOrEqual}{E,E}}
}\]

Now the projection functions:
\[\mathll{
 \kw{fun}\; \mathtt{literal\_value}(\mathtt{literal}(v)) = SOME\; v \\
 |\tb \mathtt{literal\_value}(\_)= NONE \\
 \kw{fun}\; \mathtt{sum\_left}(\mathtt{sum}(x,\_)) = SOME\; x \\
 |\tb \mathtt{sum\_left}(\_)= NONE\\
 \kw{fun}\; \mathtt{sum\_right}(\mathtt{sum}(\_,x)) = SOME\; x \\
 |\tb \mathtt{sum\_right}(\_)= NONE
}\]
and so on for each constructor argument.

Finally, the string rendering functions (assuming a function $natToString$ already exists in the context):
\[\mathll{
 \kw{fun}\; \mathtt{E\_toString}(\mathtt{literal}(v)) = natToString\;v \\
 |\tb \mathtt{E\_toString}(\mathtt{sum}(x,y))= \mathtt{E\_toString}(x) + "+" + \mathtt{E\_toString}(y) \\
 |\tb \mathtt{E\_toString}(\mathtt{product}(x,y))= \mathtt{E\_toString}(x) + "\cdot" + \mathtt{E\_toString}(y) \\
 \kw{and}\; \mathtt{F\_toString}(\mathtt{equality}(x,y)) = \mathtt{E\_toString}(x) + "\doteq" + \mathtt{E\_toString}(y)\\
 |\tb \mathtt{F\_toString}(\mathtt{lessOrEqual}(x,y)) = \mathtt{E\_toString}(x) + "\leq" + \mathtt{E\_toString}(y)
}\]
\end{example}

Because ML has inductive data types as primitives, pattern-matching on our syntax comes for free.
We will get back to that when defining the semantics.

\subsection{Object-Oriented Programming Languages}

In a object-oriented programming language, inductive data types are not available.
Therefore, they must be mimicked using classes.
On the positive side, this supports arguments names, and notations are a bit easier.

The basic recipe is as follows:
\begin{compactitem}
\item Each types is implemented as an abstract class.
\item Each constructor of type $t$ is implemented as a concrete class that extends the abstract class $t$.
\item The arguments names and type of each constructor $c$ are exactly the argument names and types of the class $c$.
The constructor arguments are stored as fields in the class.
\item The abstract classes require a \texttt{toString} method, which is implemented in every concrete class according to its notation.
\end{compactitem}


\begin{example}
We define our example syntax in a generic OO-language somewhat similar to Scala.\footnote{We could use Java or C++ here. But their concrete syntax makes this less clear than it could be. It is straightforward to refine the syntax into that of any specific OO-language.}

In particular, we assume that the sy

\begin{lstlisting}
abstract class E {
  def toString: String
}
class literal extends E {
  field value: Nat
  constructor (value: Nat) {
    this.value = value
  }
  def toString = value.toString
}
class sum extends E {
  field left: Nat
  field right: Nat
  constructor (left: E, right: E) {
    this.left = left
    this.right = right
  }
  def toString = left.toString + "+" + right.toString
}
class product extends E {
  field left: Nat
  field right: Nat
  constructor (left: E, right: E) {
    this.left = left
    this.right = right
  }
  def toString = left.toString + "\cdot" + right.toString
}

abstract class F {
  def toString: String
}
class equality extends E {
  field left: Nat
  field right: Nat
  constructor (left: E, right: E) {
    this.left = left
    this.right = right
  }
  def toString = left.toString + "\doteq" + right.toString
}
class product extends E {
  field left: Nat
  field right: Nat
  constructor (left: E, right: E) {
    this.left = left
    this.right = right
  }
  def toString = left.toString + "\leq" + right.toString
}
\end{lstlisting}
\end{example}

Because OO-languages do not have inductive data types as primitives, pattern-matching on our syntax requires awkward switch statements.
We will get back to that when defining the semantics.

\subsection{Combining Paradigms}

The Scala language combines ideas from functional and OO-programming.
That makes its representation of context-free syntax particularly elegant.

In Scala, the constructor arguments are listed right after the class name.
These are automatically fields of the class, and a default constructor always exists that defines those fields.
That gets rid of a lot of boilerplate.

If we want to make those fields public (and we do because those are the projection functions, we add the keyword \texttt{val} in front of them.
But even if that is too much boilerplate. So Scala defines a convenience modifier: if we put \texttt{case} in front of the classes corresponding to constructors of our syntax, Scala puts in the \texttt{val} automatically.
It also generates a default implementation of \texttt{toString}, which we have to override if we want to implement notations, too.
Finally, Scala also generates pattern-matching functions so that we can pattern-match in the same way as in ML.

Then our example becomes (as usual, assuming a class \texttt{Nat} already exists):

\begin{lstlisting}
abstract class E {
  def toString: String
}
case class literal(value: Nat) extends E {
  override def toString = value.toString
}
case class sum(left: Nat, right: Nat) extends E {
  override def toString = left.toString + "+" + right.toString
}
case class product(left: Nat, right: Nat) extends E {
  override def toString = left.toString + "$\cdot$" + right.toString
}

abstract class F {
  def toString: String
}
case class equality(left: Nat, right: Nat) extends E {
  override def toString = left.toString + "$\doteq$" + right.toString
}
case class lessOrEqual(left: Nat, right: Nat) extends E {
  override def toString = left.toString + "$\leq$" + right.toString
}
\end{lstlisting}

\section{Semantics as a Recursive Function}

%\section{Context-Sensitive Syntax}

