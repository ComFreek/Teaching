\documentclass[12pt]{article}
\usepackage[top=2cm,bottom=3cm,left=3cm,right=3cm]{geometry} %define borders

\usepackage{url}
\usepackage{paralist}
\usepackage{xcolor}

\usepackage[bookmarksnumbered,bookmarksopen,colorlinks,urlcolor=gray,linkcolor=blue,citecolor=blue]{hyperref}
\usepackage{basics}
\setlength\parindent{0pt}

\begin{document}

\title{Advice for Students}
\author{Florian Rabe}
\date{\today}
\maketitle{}

\begin{abstract}
This is an incomplete but growing list of things, which are interesting or important for a student to know, but which sometimes nobody tells them.
Most points apply in general; some are specific to computer science; some are specific to working with me.

This is an extremely valuable resource for students.
Make sure you do not underestimate it, e.g., reread it every year and think about how it relates to your experiences.
\footnote{This document can be freely distributed as long as it is unchanged.
I am happy to receive suggestions for additions or changes.
References and links to this document should include my name and the title to make sure that it can be found via search engines even if my homepage moves.}
\end{abstract}

\tableofcontents

\section{General}

There is a natural synergy between us. I want to (in the beginning) educate you so that you can contribute to research (usually later). You want guidance and opportunities to do independent research. That usually works out very well, and you should always remember that we are working together.

Therefore:
\begin{compactitem}
\item Do not try to impress me by pretending you understand something when you do not.
\item Do not hide problems from me.
\item Tell me when you do not have time to work on something.
\item Come to me when you need help---specific to our joint work or general academic help.
\item Tell me when you do not like to do something---there are always lots of other things you can do instead.
\item Do not pretend you are interested when really you are not committed. Then we are both better off working with someone else.
\item Do not think that you know what to do---just thinking that is proof enough that you have no clue.
Ask for advice as much and as often as you can and try to learn as much as you can.
\end{compactitem}

\section{Meetings with Me}

I like meeting with students, and I will usually offer you some kind of regular meeting. But you should be aware of some things:
\begin{compactitem}
\item I have very little time, and the time I offer you is valuable. If I feel, I am not getting anything out of it, I will stop advising you.
\item I will arrange all my meeting slots to optimize the use of my time. Thus, even when you arrive on time for a meeting, I might still be in another meeting, in which case you may have to wait a bit. Always come in in that case \item otherwise, I might not notice or forget that you're waiting.
\item If I reserve a meeting slot for you, I expect you to show up. If you have a good reason to skip a meeting, you usually know about it well in advance.
\end{compactitem}
\medskip

It is a common mistake of students to skip meetings with their advisor because they feel they have not done or understood enough.
In fact, those are the most important points at which to meet.

Students generally need more advising/supervising than they think they do---the most important purpose of advising is to tell you the things that you do not know that you need to know.
Even the strong students can become even stronger with more advising.
\medskip

It is very useful to bring a voice recorder and a camera to meetings so that you can record what we say and draw on the board. That gives you more time to think during the meeting and maximizes how much you get out of it.

\section{Communication with Me}

My preferred means for quick communication is skype. My user name is florian.rabe. When you contact me, start by stating your question or problem. I will usually attend to it on the same day. Sometimes I don't have time, and I will forget to answer; you should wait around one day before reminding me. However, if you come to me with questions that could be asked using Google or reading lecture notes, I will tend to ignore your message.
\medskip

If I write rather curt emails to you, that (usually) does not mean I'm mad at you---it's just an efficient way to communicate.
\medskip

As you make progress in your research, your work will become more and more challenging and independent. You will notice that more and more things do not work---that's because you're reaching the end of current research where some things just aren't done yet. Talking to me (or someone else) will often be the only way to get the information, understanding, or help you need.
\medskip

On the other hand, I have very little time to talk to you. You will be most successful if you learn to optimize the benefit you get out of conversations/emails with me:
\begin{compactitem}
\item Try to ask questions in a short and precise way so that I can answer fast and conveniently while giving you the important information.
\item Write emails with informative subject lines: for example, include the course you ask about or add the deadline for my response.
\item The time it takes me to reply to an email or IM is exponential in the length of the message. (quote from Alan Bundy)
\item If we have a meeting, try to write your problem down before, and take a print out to the meeting.
\item If you think you know what I should do to help you, don't be too shy to give me an exact list of instructions.
\end{compactitem}
\medskip

I (or other advisors) will occasionally tell you briefly that certain things may be interesting to you. Cherish these ideas and pointers even if their purpose is not immediately obvious to you. They are almost always valuable to you, but often you need time to understand why and how.
Write them down so that you can reread them when you have learned enough to understand them.
\medskip

Talk as much as possible to your advisors and your peers about your and their research.

If you don't talk to your advisor about either your results or the problems you encountered, you probably did not do anything valuable (and your advisor will assume that!).

From your advisor's perspective, there are two problems when communication is not frequent:
\begin{enumerate}
\item I get worried. Many students have difficulty coping with stress (course work, jobs, research projects; family, relationship, performance pressure, etc.). This leads to phases of psychological exhaustion at least occasionally. Then I need to support them. Moreover, students are often in denial about that; so it's difficult to judge what it means when they tell me they're doing fine. Keep in mind that a lack of communication is my only indicator of such situations.
\item Doing any kind of useful research work requires a tight feedback loop between student and advisor: The quality of a student's work is exponential in the amount of interaction, and crossing the usefulness threshold is hard. I will usually not even want to look at what a non-tightly-supervised student did.
\end{enumerate}
I can't force when I have ideas for your work. Often they come a day after we talked about a problem that occurred in your work---so if you have an office space close to me, it's good to be around after a meeting.
\medskip

You should seek to form cross-connections with other students in the research group.
Often you will be able to solve problems faster if you don't have to wait for an answer from me.
In some cases, this can lead to collaborations that produce joint papers. (Writing two papers together can be easier than one paper alone.)
Or it can lead to older students coadvising younger students (which is very helpful for both).
In any case, you will benefit a lot from broadening your knowledge.
Being able to give a 1-minute summary of someone else's research is often the difference between a good and a very good student.
\medskip

Do not wait for people to tell you what to do.
Actively seek them out and ask for advice what you should do.
Think of your studies (especially graduate studies) as a transition towards independence.
The further along you are, the more proactive you should be.

To find out what to do, ask for advice and observe what other people (who provide or negative examples, depending on how successful they are) do.

\section{Bachelor's, Master's theses, PhD theses, informal research projects, and student jobs}

I am always interested in advising students, for a thesis, just for fun, or to work towards a paper.
Some research topics are on my homepage, but the lists are usually outdated.
I prefer fitting a topic to your interest and abilities.
Contact me for an appointment to discuss possible topics.
\medskip

I maintain a list of previous theses on my homepage.
\medskip

I am also always interested in hiring students.
As a general rule, paid work is simpler, less fun, and more time-sensitive than unpaid work.
Also to pay you, we need to be confident that you will have a good ratio of results produced per time spent supervising.
So think about whether you prefer a student job or an informal research project.
\medskip

I will always try to write papers with you.
But that is always quite some extra work for both of us.
Also we will be bound by conference deadlines.
So tell me as early as possible if you would like to eventually write a paper.

\section{Collaboration}

If we are collaborating on something, you will usually have access to some SVN or git repository. You should get familiar with both.
\medskip

I usually give rather generous write-access. So watch out not to mess things up. However, if you do, that's no problem - just tell me, and we can easily restore the old version.
\medskip

Enter commit-log messages (at least) when you changed something that affects other people's parts.
\medskip

You should commit your work regularly, at least at the end of the day. This has three reasons:
\begin{compactitem}
\item It protects yourself from hardware failures, which could destroy all your work.
\item If I see no commits in the log, I might assume that you have done nothing at all.
\item I will occasionally and spontaneously make changes to your parts of the project (namely whenever I have ideas and time). If your part is not up to date, we will get conflicts.
\end{compactitem}

\section{Paper writing tips}

\subsection{Structure}

A typical research paper or thesis in computer science has the following sections:
\begin{compactitem}
\item Abstract: A short, high-level summary (2-3 paragraphs) that tells people who do not have your paper handy what is in it.
It should help people decide whether your paper is relevant to their work.
Do not duplicate sentences between abstract and paper.
\item Introduction: Concise (= short and precise) answers to the following questions
 \begin{compactitem}
  \item What is the problem?
  \item Why is it important?
  \item What solutions have been tried in the past?
  \item What is the key idea behind your solution?
  \item What does your solution contribute to the field?
 \end{compactitem}
\item Preliminaries: Summary of all previously existing work that is needed to understand what follows.
  A paper should be self-contained, i.e., understandable to an expert without looking up any references.
\item Possibly a section that develops some new basic concepts that are used later on.
Make sure the paper does not get boring in this section, i.e., foreshadow how the new concepts will be used.
\item A section about the main thing you did. Depending on the result, this may have the key definitions, algorithms, and theorems.
\item Possibly another section with secondary results that build on the previous results.
\item Possibly a section about the implementation and practical results (timing measurements, case studies, etc.).
\item Conclusion: A short (much shorter than the introduction) summary of the paper from the perspective of someone who has read it already.
Depending on the paper, it may answer the questions
 \begin{compactitem}
  \item How did your solution work out?
  \item What can still be improved?
  \item What impact do you expect your work to have in the near future?
  \item What future work is now possible or recommended?
 \end{compactitem}
\end{compactitem}
Moreover, a section Related Work is needed, but it may occur in different places:
 \begin{compactitem}
  \item If the problem is very technical and comparing it to other solutions requires knowing your solution, Related Work can be just before the conclusion.
  \item If related work is structurally very different from your solution and helps explain what the problem is, Related Work can come in or after the introduction.
 \end{compactitem}

\subsection{Separations}

There are some areas where bad writing conflates things that should be strictly separated.

\paragraph{Preliminaries vs. Contribution}
Always clearly separate pre-existing work and your new work.
Usually they go into separate sections, in which the title or the first sentence clarifies their nature.
This is important because, otherwise, many people will not be able to assess what existed before and what you did.

Preliminaries should be written in such a way that
\begin{compactitem}
 \item experts can efficiently catch up on which definitions and notations are needed to understand your contribution without having to read anything else,
 \item non-experts find out which generally known concepts they need to read up on and where they can do that
\end{compactitem}

\paragraph{Conceptualization vs. Application vs. Implementation}
The conceptual parts presents all new ideas, methods, results, etc.
It is the most important part.
It should be described in a way that it is entirely independent of your application or implementation.

The application shows that the results are useful in practice.
Depending on the result, this may take many different forms.
It is often a separate section or an example in the same section as the conceptualization.
It can also occur as a brief description in the introduction or conclusion.
The application often involves an implementation that is presented via, e.g., screenshots.
But it can also be theoretical (e.g., instances of your new theorem) or hypothetical (e.g., examples of using your new unimplemented algorithm).
If an implementation exists, the application should be described in a way that is entirely independent of the implementation.

The implementation serves two purposes: it demonstrates the feasibility of your results, and provides a service to users who can apply it.
The implementation is usually barely mentioned at all: it is usually enough to say it exists and reference the sources and documentation.
If any kind of benchmark is available (speed, expressivity, user-friendliness, etc.), the merits of the implementation should be discussed.
Beyond that, only the critical but non-obvious details should be mentioned.
For example, the choice of programming language usually does not matter except if, e.g., it was made specifically so that a certain indispensable library could be used; 
Only in a thesis or a technical report should you add a section that describes the implementation in more detail, e.g., with some code listings.

\paragraph{Definition vs. Example vs. Theorem/Application}
It is not always necessary or even reasonable to dogmatically follow the definition-example-theorem style.
However, even if this style is not used explicitly, the material should still be structured in this way.
It should be clear to readers
\begin{compactitem}
 \item whether a word is used in its general or common sense or whether it is a technical concept introduced in your paper,
 \item where a new concept is
  \begin{compactitem}
    \item motivated or discussed,
    \item defined precisely (and this should be done somewhere for each new concept),
    \item exemplified by considering a special case,
  \end{compactitem}
 \item where a result is presented that is enabled by the previous work,
\end{compactitem}

\subsection{General Advice}

The most important things about your work should always be the most conspicuous.
Use section structure, definition/theorem/notation environments, display math, figures, tables, boldface fonts, etc. to structure your text.
This is important because many people will not actually read but only skim the paper.
So make sure your paper is - at some abstract level - understandable even for a very cursory reader.
\medskip

Use the hyperref package to make sure a good table of contents is generated for your pdf.
For example, use
\begin{verbatim}
\setcounter{tocdepth}{3}
\usepackage[bookmarks,bookmarksopen,bookmarksnumbered,colorlinks,
   linkcolor=blue,citecolor=red,urlcolor=gray,breaklinks]{hyperref}
\end{verbatim}
\medskip

There are certain formulations that have important literal meanings.
For example,``we would/could/should do'' means you cannot do it (yet).
``we can do'' means you have not tried it (yet).
``we will do'' means you plan to do it but have not done it.
``we do'' means you did it, and this paper presents the result.
Always unambiguously express what you did.
\medskip

Each paper needs a story, a reason why your work is worthy of being made available to others.
Often people do not care about your particular motivation, background, or application.
They only care about the general ideas and lessons that they can take from your work. That is your story.
Usually you need an advisor to explain to you what the story is because you have no idea what other researchers will care about.
You can publish multiple papers with different stories about the same research. But it's difficult to publish a paper that just contains the raw research without a story.
A good combination is often to write up the raw research separately (That is called a technical report.), put it on your homepage, and then refer to it in the paper.
Sometimes the technical sections of a paper and the story can be written independently. Sometimes the technical part is modified substantially (simplified, summarized, different notation, reordered) to fit the specific story.
\medskip

An important rule of thumb is ``One result, one paper.''
Throwing too many ideas into a paper makes it confusing.
\medskip

A paper should not describe what you did.
It should describe what you wish in retrospect you had done.
Setbacks, mistakes, and detours that were necessary to find the result do not go into the paper.
Only the direct route to the result does.
\medskip

Sometimes it is possible -- and then it is very strong -- to develop a little ontology of general concepts related to the problem you work on.
For example, it can describe desirable but mutually exclusive properties where trade-offs are needed.
Or it can describe fundamentally different approaches to a problem with their advantages and disadvantages.
Based on such descriptions, you can concisely position your work within the field and relate it to other papers.

\section{Thesis Writing Tips}

This is mostly written for PhD theses.
It partially applies to BSc and MSc theses as well.

Depending on your field, topic, and personality, these tips are neither necessary not sufficient criteria for writing a good thesis.

\subsection{Initial Phase}

You should know the general area of your topic and have a well-defined goal.
This can be unrealistically visionary as long as it steers you in a clear direction.
\medskip

You should write a short summary of your envisioned thesis and keep it around.
You should revise and update it frequently, gradually closing in on your true topic.
\medskip

Within the confines of your general goal, you should stay open to other ideas and not limit yourself early on.

\subsection{Middle Phase}

Talk to your advisor frequently, about once a week for one hour.
\medskip

Ask your advisor to recommend you to about two secondary advisors, possibly external ones.
Talk to them irregularly, e.g., give them a presentation twice a year or send them major papers you wrote.
\medskip

Collect as much information as you can: read papers, talk to people.
You can safely assume that virtually every idea you have in the first years has been done before in some way.
Find out how it was done and understand what you want to do differently.
\medskip

You can and usually should engage in a few side projects as long as they are somewhat related to your goal.
At the very least, these will generate ideas and create a network of people to get feedback from.
Often these projects become part of your thesis topic before you realize it.
\medskip

Write up partial results early on.
Publish them if possible.

\subsection{Final Phase}

At some point your topic emerges and you understand what you really want to do.
Start writing your thesis document: update your thesis summary, draft chapter titles, and copy-paste your previous write-ups into your thesis.
This will be a mess.
Tidying it up will be the main step towards your thesis.
Often tidying up will trigger additional research questions to resolve.
\medskip

A PhD. thesis is different from a paper:
\begin{compactitem}
\item It is more fundamental: It does not solve a small preexisting problem but identifies, analyze, and partially solves a new problem.
\item It is more self-contained: It includes all details and (where feasible) preliminaries.
\end{compactitem}
\medskip

It is hard but important now to isolate yourself from distractions.
Finishing the thesis is hard and will require your full attention.
\begin{compactitem}
\item Major academic measures
\begin{compactitem}
\item Discuss with yourself and your advisor which aspects are crucial for your thesis and which are optional.
Be prepared to cut your thesis topic down drastically to make it realistic.
\item Focus on getting the simplest possible thesis written fast.
To do that, classify as many of your problems as possible as future-work.
If you have time left afterwards (and often you do not), you can attack additional problems to increase the mass of your thesis.
\item Do not put other people's work on your critical path.
\item Create a mind map or similar of all thesis-related concepts. Keep it around and update it.
\end{compactitem}
\item Expect \ldots
\begin{compactitem}
\item \ldots setbacks:
Discovering major flaws and gaps in your thesis is a normal part of writing up.
\item \ldots delays:
Write-up will take much longer than you expect.
\item \ldots cuts: You cannot solve all problems in one thesis and have to focus on one problem.
\item \ldots mental exhaustion: Casual thesis writing is nearly impossible -- if you are doing it right, you will almost hate your thesis towards the end.
\end{compactitem}
\item Private life
\begin{compactitem}
\item You will (have to) be more and more consumed by your thesis, almost completely so in the last few months.
\item Tell your friends/family/partner that you are writing up. Ask them to support you by helping you focus your time and mental energy.
\item Decide which aspects of your private life, you can sacrifice for half a year.
\item Pick one or two of your favorite private activities and schedule them consciously to recharge.
Cut everything else.
\item Minimize the time spent on eating and chores to create large blocks of distraction-free time.
\item Eat/drink and sleep well. Work according to your body's schedule.
\end{compactitem}
\item Minor measures
\begin{compactitem}
\item Keep a time log to identify where you procrastinate.
\item Skip meetings by default.
\item Ask for a quiet desk or office.
\item Clean up your desk. Remove anything that is not thesis-relevant. Put up thesis-related printouts (table of contents, open problems, to-do list, blurbs on related work items, etc.).
\item Do not implement anything but the system needed for your thesis.
\item Uninstall software that tends to distract you, e.g., your programming language's IDE.
\item Use browser add-ons to block websites where you procrastinate.
\item Filter mailing-lists and stop reading them.
\item Limit email and other communication to certain parts of the day (e.g., one hour each evening).
Disable your social media accounts.
\end{compactitem}
\end{compactitem}

\section{Presentation Tips}

\subsection{Content}

A presentation should not summarize the research you did or the paper/thesis you wrote (typical beginner's mistake).
Instead, it should convey an interesting insight to the audience and explain how it was done.
Details that the audience will forget anyway should be left out or confined to a short segment at the end.
(Of course, the talk must back up the advertisement appropriately.)
\medskip

A presentation should be roughly divided into three thirds: understandable to everyone present, understandable to people from the field, understandable to experts.
\medskip

You can use slide titles and subtitles and subslide numbering to indicate the structure of your talk.
It is usually a mistake to have a table-of-contents slide: they are boring and useless.
At the very least, do not spend more than a few seconds on such a slide.
Alternatives to a table-of-contents slide can be:
\begin{compactitem}
\item Intuitively explain the words in your title in such a way that they end up summarizing the result.
During the talk, you can refer back to this summary to explain which part comes next.
\item If you have a diagram that summarizes your results, you can show it multiple times, each time highlighting the part that you present next.
You can show the diagram again at the end to summarize.
\end{compactitem}
\medskip

It is acceptable and often necessary to change details of your work when presenting it.
Usually it has to be simplified a lot to be understandable by your audience.
Often new, simpler definitions or notations have to be introduced specifically for the talk.
Sometimes it is acceptable to omit details or adapt statements in such a way that the presented results are technically false, e.g., one can omit a technical precondition for a theorem; to be safe, this should be done with a footnote-like disclaimer.
\medskip

Every slide should only assume minimal knowledge about the previous slides.
Introducing a technical definition on one slide and then using it on the next almost never works well.
\medskip

If in doubt, give examples instead of definitions.
\medskip

If a slide has to be technical, protect your audience from getting confused or even lost.
For example, say explicitly that the slide provides non-essential technical details.
\medskip

Make sure the main ideas can be understood by someone who does not listen at all and only occasionally looks at the slides.
For example, point out which slide presents the main result.
\medskip

Generally, you should assume the audience
\begin{compactitem}
\item knows much less
\item will understand much less
\item will remember much less
\item will listen much less
\item will pay much less attention
\end{compactitem}
than you think they do.
Make an active effort to explain your results in terms the audience can understand (even if you have to omit a lot of technical aspects).
\medskip

\subsection{Form}

\paragraph{Structure}
Start by thanking the person who introduced you.
Begin the presentation by briefly introducing yourself and your background as needed for the audience to place you.
When presenting work by multiple people, list all of them as authors on the title slide and indicate (e.g., by underlining) which one you are.
\medskip

Understand in advance what the most important transitions and slides are.
Spend more time on those.
Repeat things, or pause to give the audience a chance to think about it.
\medskip

Use a final slide that summarizes your main results.
This should be such that someone who only sees that last slide gets a good (high-level) impression of what you did.
Many people will only remember this slide anyway.
To get back everybody's attention introduce this slide with something like ``And now for my last slide, which summarizes the main result again.''
Sometimes it works to explicitly tell the audience that if they only want to remember one slide, it should be this one.
Leave this slide on during the following discussion -- do not jump to, e.g., a useless ``Thank You'' slide.

\paragraph{Style}
Use consistent style/formatting.
\medskip

Slides should contain concise text.
Full sentences should usually be avoided.
On average, at least every second slide should contain some non-text information (table, diagram, example, screenshot, etc.).
Often it works very well to use half the slide for text and the other half for an example.
\medskip

Itemizations should be introduced by a line that describes what is in the itemization.
All items should be similar in type, e.g., all items are parts of one whole, or all items are pros or cons of an idea.
A second level of nested itemizations often helps clarity.
A third level can easily be too messy though.
\medskip

Visually distinguish the main points from secondary aspects. For example, by using my \verb|\lec| macro.
\medskip

Overlays, animations, and colored highlighting are a lot of work.
(Avoid colors like bright green fonts or black backgrounds that are hard to read when projected.)
But they make the presentation much better and should be used where possible.
But they should always be used precisely and with a clear goal in mind.
For example, you can use colors to indicate which part of the text corresponds to which part of the example. 
\medskip

Slides should be numbered so that your audience can refer to them in the discussion.
\medskip

Citations in slides are totally different from citations in a paper because your audience cannot look up a citation.
It works best if a citation manually (i.e., without using \verb|\cite|) lists the authors by last name, the year, and possibly the acronym of the journal/conference.
Do not show a bibliography during a presentation; you can have a bibliography slide (especially, if you put your slides on the web later) but do not show it during the presentation.

\subsection{Technology}

If your battery is modern enough, you should be able to trust it without using a power adapter.
Make sure it is charged.
\medskip

Check if your computer works with the provided projector, e.g., in one of the breaks before your session.
When your talk starts, you should be able to get going within 1 minute.
\medskip

Make sure your slides are available online or on a memory stick so that you can easily switch computers if yours does not work.
Convert your slides to pdf unless you absolutely need something else.
\medskip

Make sure your phone is switched to silent.
\medskip

Make sure your screen saver is off and none of your running applications can pop up windows (e.g., anti-virus software or notifications from instant messaging applications).
\medskip

Modern operating systems like Windows 10 update automatically and then restart.
Make sure no restart is scheduled during your presentation.
\medskip

You should use a laser pointer and a remote slide-switcher, which you have to bring yourself.
Watch out not to rush slide transitions when using a slide switcher.
\medskip

If you plan a live demo (and if you can do one, you should), prepare a backup in case it does not work (especially for demos that require network access, especially if your server uses non-standard ports).
As a backup, put screenshots or videos into your slides.
These backups can also be helpful if you run out of time.

\subsection{Interaction with Audience}

Face the audience except possibly when you point at something.
You may occasionally say the same thing as written on your slides (typically very important parts that you want to emphasize), but generally you should not read off your slides.
Do not look at your computer except possibly when switching slides.
\medskip

Give the audience the feeling that you are telling them something interesting.
Make yourself aware that you are giving the talk for the audience: you want to explain something to them.
Show a little excitement and enthusiasm.
Look at them, ask them (rhetorical) questions, adjust your examples to their background, adjust your speed based on their facial expressions.
\medskip

Record yourself and check your body language to see if you actually appear the way you want to.
\medskip

Some people find it helpful to pick one audience member and imagine that they are talking to that person.
\medskip

Find out how loud and fast you should speak to be understood well.
\medskip

Jokes can be very useful to connect with the audience.
Be careful with jokes when you are already nervous: if not made confidently, they can give away your nervousness.
Do not make self-deprecating jokes that could give the audience a bad impression of your work.
\medskip

Always appreciate questions---people are trying to understand more of your work.
If no one asks questions, probably no one understood anything.

\subsection{Timing}

1 minute per slide should be a lower bound.
Depending on the presentation and your style, even 2 minutes per slide can be reasonable.
Time yourself in each talk and find out how many minutes per slide you usually need.
\medskip

Practice your talk to check the timing.
Be aware that you may speak much faster (less often: much slower), when you are nervous during the actual presentation.
\medskip

Check in advance if the room has a clock or make sure you have a clock that you can look at discreetly.
\medskip

Identify a few points (e.g., one every 10 minutes) at which you check your time.
Identify in advance which slides you can skip or skip partially if you are running long.
(The dual advice also applies, but it is rare that speakers run short.)

\subsection{Multi-Talk Sessions}

Very often presentations are given as a succession of talks in multiple sessions.
Often each session takes about $90$ minutes and contains about $3$ presentations.
Usually a session chair will introduce the presentations.
\medskip

Conference talks usually give you 30 minute slots.
In that case, allow for about 5 minutes of discussion.
\medskip

If you give the first talk in a session, you can set up during the break.
\medskip

The session chair may seek you out to make sure you are present, find out which author is presenting, and how to pronounce your name.

She may also inform you about time signals.
A typical signal is 5 fingers to indicate that you have 5 minutes left.
Make sure you know whether that is 5 minutes including or excluding the discussion phase.
If you get the $1$-minute-left signal, jump directly to your slide and use it to summarize your main result.
Make sure the last part of your presentation is non-critical in the sense that you can skip over it if you have to.
\medskip

Do not use other people's time.
Check at what time you started so that you know when your slot will end.
If the session chair stands up, that can mean you are running late.

\subsection{Thesis Defense Talks}

\paragraph{General Aspects}
A defense talk addresses a wider audience than a conference talk, especially if senior researchers from outside your area (e.g., a second reviewer) are present.
As a rule of thumb, try to be understandable to graduate students who have not read the thesis.
\medskip

If you did a good job, you and your adviser are the only experts on your topic in the room.
Do not focus too much on you being evaluated by her --- instead, think of the talk as you teaching the audience about an interesting topic.
If your reviewers learn something interesting, they will also evaluate you positively. 
\medskip

Often other professors are present.
Be aware that it is not only you who is evaluated but only your professor: he gains in reputation if his students give impressive defenses.
That also means he should be interested to help you prepare. 
\medskip

Try to attend other defenses before yours to get a feeling for what the procedures and formalities are like.

\paragraph{Rough Structure}
Use the first third to describe the general area of your thesis:
\begin{compactitem}
 \item What are the key developments (historical and current)?
 \item What are the key problems and why are they important and difficult?
 \item Why are the (partially) unsolved problems and what approaches have been tried by the community?
\end{compactitem}
Presenting the answers to these questions should gradually lead to talking about your thesis.
This part should be understandable to every computer scientist, including graduate students.
\medskip

The second third should give a good overview of your research:
\begin{compactitem}
 \item the ideas behind your solution and how they differ from other approaches,
 \item an overview of how you realized these ideas,
 \item the strengths and weaknesses of your solution,
 \item the contribution of your work to the overall field.
\end{compactitem}
This part should be more technical but still try to cover your entire thesis, i.e., you should present a highly abstracted view of your entire work.
This part should be understandable to all your reviewers except possibly for secondary reviewers from outside your area.
\medskip

The third third gives technical details.
It is acceptable if this part is only understood by experts, e.g., your adviser and other members of his research group.
Here you usually have to restrict yourself to one or two important parts that are presented in detail.
\medskip

There should be enough time (around $10\%$ of your time) left at the end to summarize everything.
This part should be understandable to everybody present again.
It should repeat the key contribution and describe future work.
\medskip

The above structure does not always fit well.
For example, if your work can be nicely split into 2-3 parts, it makes sense to present them separately.
Then you can describe general and technical aspects of each part separately.

\paragraph{Content}
The most important rule, especially for a PhD defense, is to omit details.
You most likely underestimate how much expertise you acquired while working on your thesis and overestimate how much your audience understands about your work.
Do not try to talk about everything you did, and skip details whenever you are not sure if they will be understood.
\medskip

However, make sure to present some technical details at least once or twice.
The purpose here is indirect:
\begin{compactitem}
 \item give your audience a feeling for what the details are like (and it is OK to introduce the details by pointing this out),
 \item make sure no one mistakenly assumes your work does not include any technical details.
\end{compactitem}
\medskip

Your research likely has parts that you have not understood well yet.
Simply skip those (except for the future work section).
Give a very good talk about the things you have understood rather than a good one that covers more topics.

If necessary to be understandable, present your work differently than in your thesis (e.g., simplified notation, restriction to a special case, new examples).

\paragraph{Structuring Advice}
For long defense talks like PhD defenses, the structure is very important.
Make sure it is clear and present it very clearly.
This is critical to guide your audience.
\begin{compactitem}
\item Structure the technical parts of your talk (e.g., the last two thirds) into a few sections.
 Ideally, present the structure early on as a part of giving an overview of your solution.
 (Avoid functional section titles like ``introduction'' or ``example''---instead, use topical ones that help explain your work.)
\item Use slide titles and subtitles to clearly indicate where you are in the talk. 
\item Use section-separating slides. Use these slides to recall the overall structure and explain how the previous and the upcoming section relate to each other and/or to the whole.
\item You may lose some audience members during technical slides. Use the section breaks to offer those people reentry points to your talk.
\end{compactitem}
\medskip

If you have a long title, it may make sense to structure the initial part as an explanation of all the words in the title.
For example, you can use 1-2 slides on the 3 main words of the title.
\medskip

Use a running example throughout the talk.
Ideally, it appears at the beginning to motivate the problem, along the way to illustrate everything you say, and again at the end to show how your solution works.
Choose the example carefully: it should be as easy as possible but only solvable with your work.
\medskip

If possible, use a repeating slide that highlights which part of your work you are currently talking about.
(This is similar to and can be combined with using section-separating slides.)

For example, your initial slide with the running example can contain the grayed-out solution already; along the way you reveal different parts of the solution; at the end the full solution is shown.

Alternatively, you can use a diagram of your system or of how various aspects of your work interrelate; in the beginning you use it to give an overview; along the way you highlight the part you are talking about next; at the end, the audience has understood all parts.
\medskip

It may help to use a second screen or a whiteboard to keep one part of your talk (e.g., the running example) visible all the time.
A whiteboard is especially helpful if you have a large example that does not fit on one slide.

If you use a whiteboard in this way, do not plan to write on it during the talk.
Carefully prepare the entire content beforehand and test it on the whiteboard. Make sure it is easily readable from the back row.
On the defense day, enter the room well in advance to put up the content.

\paragraph{Timing}
A defense talk is usually the longest talk you have ever given.
Make sure you have a good feeling for your timing.
Practice your talk live, ideally in front of fellow students or advisers.
Pay particular attention to the structure of your talk and your timing.

\paragraph{Rite of Passage}
Your defense talk marks a transition in your education and career.
If you graduate, the talk is your opportunity to demonstrate for the first time your independence as a professional.
Be aware of this function and show appropriate respect (e.g., through your clothes, your body language, and your attitude towards your audience and reviewers).
\medskip

In a defense talk, your audience includes people who would normally not listen to a talk by you.
It is your responsibility to demonstrate that you are worthy of their time and attention.
\medskip

This applies particularly to a PhD thesis defense.
Think of it as transitioning from being a student (= child) to a researcher (= grown-up).

\paragraph{Administrative Responsibility for a PhD Defense}
As a grown-up, you are now blamed for everything that goes wrong even if it is not your fault.
Therefore, unless your school has different regulations or customs, you are responsible for all administrative aspects:
\begin{compactitem}
 \item You schedule your defense yourself. Contact all your reviewers early to find a good time.
 \item Book an appropriate room for the defense. Make sure the room is open or that someone has a key.
 \item Think about what equipment (second screen, whiteboard, flip-chart, etc.) you want to use.
   Make sure your room has it and make sure everything works.
 \item Check with your advisor if there is anybody else you should invite, e.g., local researchers who are not automatically invited and might be interested in your results.
 \item If an external reviewer attends remotely, double-check the connection on the day before the defense. (Skype is a good choice these days, but multiple alternatives have popped up.)
   Make sure you know where to place the camera to provide a good view.
   Sent your slides to your reviewer before your talk so that she can follow that way if the video quality is not good.
   Make sure you know her phone number and that a phone is available as a backup in case the internet is out.
 \item If your school has unusual regulations, make sure your external reviewers know about them.
 \item Find out if you are expected to provide a reception after your talk. Ask your spouse or a fellow student to help you with the organization on the defense day so that you can focus on your talk.
\end{compactitem}

\section{Wisdom about Research}

A researcher is never happy with a paper when it gets submitted.
\medskip

Research never goes as fast as planned. Be honest about that, and people will understand.
\medskip

When you have a thesis/project deadline, do not put other people's work on your critical path.
\medskip

Making new mistakes is progress.
\medskip

Identifying what you or other people should have done or should do is progress.
\medskip

Research often occurs in phases:
\begin{compactenum}
\item have an idea
\item work out the idea properly
\end{compactenum}

Then it continues along one of the following two branches.

Theory:
\begin{compactenum}
\item develop the theorems about your worked-out idea and prove them
\item write a paper about your theorems
\item use your theorems to solve problems
\end{compactenum}

Practice:
\begin{compactenum}
\item implement the worked-out idea
\item make the implementation usable for others
\item use the implementation to do a case study
\end{compactenum}
\medskip

Be aware of the following problem
\begin{compactitem}
\item each phase takes orders of magnitude more work than the previous one,
  therefore: moving to phase n+1 without fully understanding phase n is an expensive mistake,
\item each phase discloses problems that were invisible in the previous phase,
  therefore: you can't understand phase n without moving to phase n+1,
\item good research constantly loops between the phases so that experiences from the later phases feed back into earlier phases.
\end{compactitem}
In a PhD thesis, it's quite normal that half your time is spent redoing earlier phases (including renewing your understanding of what the idea actually is).
\medskip

Always do the simplest possible thing all the way through.
Then try to do more general problems all the way through.
If you try to do the first step for a general case, you will usually get stuck; even if you don't get stuck, your solution for the first step will usually be wrong.

Two metaphors related to this lesson:
\begin{compactitem}
\item The tunnel metaphor: To build a tunnel through a mountain, first walk around the mountain (~ manual input-output example) to see whether you want to be there at all; then build the simplest, smallest possible tunnel that you can crawl through (~ prototype), then widen the tunnel (~ development).
\item The pyramid metaphor: To construct a hiearchic building (~ computer system), start with the smallest base area such that the tip of the pyramid over it reaches the intended height; then widen the base area turning the pyramid into a frustum with a widening plateau area (~ the provided interface layer).
\end{compactitem}

And an opposite advice: Sometimes foresight commands substantial changes right at the beginning.
Metaphorically, we might foresee a much larger eventual pyramid frustum, too large to be supported by the chosen construction site; then we should start building elsewhere right away (or be aware that we first build a small practice pyramid before starting with the real one).
Applying this foresight in the right situations requires a lot of intuition, expertise, and experience.

\section{Mentoring}

These are usually not relevant for students directly until they are senior enough to mentor a student yourself.
But they can help you judge who is a good mentor for you.\footnote{Several points in here are adapted from a list given at \url{http://http://www.askamanager.org/}.}

\begin{itemize}
\item Invite them to sit in while you do things. Discuss with them afterward how it went and why it went that way.
\item Discuss your situation and problems with them. Ask them what they would do (not to get advice but to train their decision making).
\item Let them mentor someone on their own. Talk to them regularly about how it goes and what they can do.
\item Increase their responsibilities regularly. Give them tasks that go slightly beyond their current skills. Discuss how it went with them.
\item Discuss their goals with them. Try to help them.
\item Increase their confidence with honest feedback about what they are good at.
\end{itemize}

\section{Etiquette}

Senior researchers know a lot more than you and have very little time.
Always be aware of that when you interact with them.

\paragraph{Asking for Feedback}
When you ask someone for feedback on your work (and you should do that often!), make sure you give her a readable text.
In general, try to make the job of helping you as easy and quick as possible.

Tips:
\begin{compactitem}
\item Print it out with wide margins and give it to her (preferably before she goes on a trip).
\item Mark (or comment out) the passages that are immature to save her from trying to understand them.
\item Leave notes about the things you still want to write so that she does not have to tell you to write it.
\item When you ask for a second round of feedback (and you should do that!), make sure all the feedback from the first round is worked in already. Also try to mark all the big changes so that she does not have to read things twice.
\end{compactitem}

\paragraph{Acknowledging Contributions}
If a key idea of your work arose in discussion with someone else or from an anonymous review, you should acknowledge that person's contribution in a paper or thesis. (To do that, you can add a paragraph ``Acknowledgments'' somewhere, e.g., after the conclusion.)
If you cannot describe the specific contribution, use a phrase like ``This work benefited from discussions with''.
If that person has already published a paper related to that idea, also cite the paper.
This does not apply to feedback you get from your advisors for a thesis---that is taken care of already by stating your thesis advisors in the title page.

If you write a paper, it is allowed to reuse parts of it for a project proposal or thesis.
But you have to mention that and cite the paper.
If the paper has authors besides you, your thesis has to describe (usually about one or two sentences) who did what.
This is usually done in the introduction or in a paragraph titled ``Acknowledgments'' at the end.

\paragraph{Different Opinions}
Do not tell a peer, let alone a more senior researcher, to use method X even if you think you are right.
Instead, ask them how their work is related to/different from X or why they chose not to use X.

Do not recommend related work to a more senior researcher: If it is really related, she probably knows it already.
If you really think she is unaware of something she should know, ask her what she thinks of that work.
If she really does not know, she will usually ask to explain what it is.

Similarly, do not dismiss their ideas even if you think they are wrong.
Instead, try to understand why they are right.

As a rule of thumb, if it turns out that your advisor was wrong and your were right, you are ready to write-up your PhD thesis about it.

\paragraph{Communicating with More Important People}
In any meeting (from a 1-hour administrative meeting to a 1-week conference), time is a scarce resource.
It is important that it is used well by communicating only the most important information.

Therefore, estimate your own importance, try to talk proportionately much, and try to talk only if your point is the most important thing to say right now.
In particular, do not interrupt or otherwise disturb when someone more important is talking.

When you say something, make sure it is concise and relevant ot the others.
Do not spend a lot of time talking about what you but no-one else care about.
\medskip

Usually, seniority is a good measure of people's importance
(Unfortunately, the opposite effect also exists: If objective truth cannot be established easily, people can be mistaken to be important or correct just because they talk a lot.)
For example, an undergraduate student should not interrupt a discussion in which a PhD student is involved.
\medskip

This emphasis on seniority may appear counter-intuitive.
After all, science is about neutral discussion and truth and not authority or personal merits.
Indeed, many scientific breakthroughs have to overcome resistance in the beginning.
However, scientists usually need to be highly immersed in a subject before they can make valuable contributions, and beginners and outsiders are usually unaware of how much they do not understand yet.

Therefore, it is best to ask a series of questions with the goal of filling the gaps in your understanding.
More senior researchers will often be happy to explain things to you if you make it easy for them by learning quickly.
If applicable, they will also help you understand under what specific circumstances your idea has merit even if it is false in general.
But they lose interest in you if you keep making statements they know to be naive or irrelevant.
\medskip

While seniority is a good indicator, importance also depends on the context.
For example,
\begin{itemize}
\item If you are the most knowledgeable at the current topic, your opinion is important.
(But make sure you understand what the topic actually is---maybe your area of expertise is not as relevant to the topic as you think.)
\item If someone else is resting an important argument on a mistaken assumption, it is important that you correct them.
(But you do not have to correct trivial mistakes that everybody else notices also.)
\item If two researchers are engaged in a scientific discussion in a public place (e.g., in a hallway or during a conference coffee break), it is usually OK to listen in.
But do not interrupt or change the subject unless you are certain it helps their discussion.
\item If you are giving a talk and a senior researcher derails it with tangential questions, you may point out that you are happy to answer questions later but need to continue with your talk now.
(But make sure the questions really are tangential---maybe you are misunderstanding a crucial point.)
\item If you are a member of a hiring committee, you are more important than the candidates, no matter how senior.
(But remember that the other members of the hiring committee may be even more important.)
\end{itemize}

\section{Procrastination and Productivity}

First of all, understand that \textbf{creation is hard}.
Especially writing of a thesis or a paper usually takes a lot of mental energy and a certain state of mind.
Even the best researchers often find themselves blocked or unmotivated.

Procrastination (doing anything else but the thing you have decided is most important to do next) happens to everyone.
Find your personal way to deal with it.


\subsection{Avoiding Procrastination}

These sites may help you
\begin{compactitem}
\item \url{http://alifeofproductivity.com/}
\item \url{http://www.structuredprocrastination.com/}
\end{compactitem}
\medskip

An annoying task that can be done in a few minutes, should usually be done right away.
\medskip

Do not try to force yourself to do a task you dread.
Do not remain in denial about why you are not doing it.
Instead, be honest to yourself and evaluate why you want to do it and why you are not doing it.
Try to reorganize your environment and your work in such a way that you can do the task happily.
That is very difficult though, and you may have to find your own way to deal with it.

\subsection{Structuring Your Work}

General tips:
\begin{compactitem}
\item have a good room temperature (about 22 degrees Celsius)
\item air the room every few hours
\item eat small snacks every few hours---e.g., a piece of fruit or a small slice of bread, nothing unhealthy though
\item use music, walks, resting furniture, mild exercise, or healthy snacks to have short creative breaks every few hours
\item spend breaks away from the computer, avoid sitting in breaks
\end{compactitem}
\medskip

Organize your work into categories for different levels of motivation and excitement
\begin{compactitem}
 \item easy routine work: filling out administrative forms, prepare diagrams for a paper, re-read previous work
 \item previously difficult problems that you have understood by now and ``only'' have to work out: usually write-up or implementation
 \item free interaction: brainstorm new ideas, future developments
 \item targeted interaction: talk to colleagues/advisers about specific problems
 \item challenging creative work: think about a problem, try out complex ideas that are difficult and may not work
\end{compactitem}
Make sure you always have enough tasks in each category and prioritize by importance and personal enjoyment.
Then choose tasks based on environment and mood.
\medskip

The brain needs an alternation between two modes:
\begin{compactitem}
\item Focused: on the desk, highly concentrated, goal-oriented, usually on your own
\item Mind-wandering: open thinking, not necessarily directly related to your work, maybe talking to colleagues, could be in a break or on the way to work or even in the shower
\end{compactitem}
Use mind-wandering phases to get ideas and focused phases to work them out.
Switch to mind-wandering or a break when focused work tires you out.
Switch to focused work when mind-wandering stops making progress

\subsection{Happiness While Working}

Good work comes from the following cycle: motivation - work - success - motivation.
\medskip

People need four kinds of circumstances to be happy working:
\begin{compactitem}
\item physical: through opportunities to regularly renew and recharge at work
\item emotional: by feeling valued and appreciated for their contributions
\item mental: by having the opportunity to focus in an absorbed way on their most important tasks and define when and where they get their work done
\item spiritual: by doing more of what they do best and enjoy most, and by feeling connected to a higher purpose at work
\end{compactitem}
\medskip

People need the following factors to work well:\footnote{taken from \url{http://www.nytimes.com/2014/06/01/opinion/sunday/why-you-hate-work.html}}
\begin{compactitem}
\item Renewal: Employees who take a break every 90 minutes report a 30 percent higher level of focus than those who take no breaks or just one during the day. They also report a nearly 50 percent greater capacity to think creatively and a 46 percent higher level of health and well-being. The more hours people work beyond 40 — and the more continuously they work — the worse they feel, and the less engaged they become. By contrast, feeling encouraged by one’s supervisor to take breaks increases by nearly 100 percent people’s likelihood to stay with any given company, and also doubles their sense of health and well-being.
\item Value: Feeling cared for by one's supervisor has a more significant impact on people's sense of trust and safety than any other behavior by a leader. Employees who say they have more supportive supervisors are 1.3 times as likely to stay with the organization and are 67 percent more engaged.
\item Focus: Only 20 percent of respondents said they were able to focus on one task at a time at work, but those who could were 50 percent more engaged. Similarly, only one-third of respondents said they were able to effectively prioritize their tasks, but those who did were 1.6 times better able to focus on one thing at a time.
\item Purpose: Employees who derive meaning and significance from their work were more than three times as likely to stay with their organizations — the highest single impact of any variable in our survey. These employees also reported 1.7 times higher job satisfaction and they were 1.4 times more engaged at work.
\end{compactitem}
\medskip

There are different kinds of writer's block, and it is possible to overcome them with specific strategies.
One study\footnote{Adapted from \url{http://www.newyorker.com/science/maria-konnikova/how-to-beat-writers-block}} about creative writers found that blocked individuals showed low levels of positive and constructive mental imagery: they were less able to form pictures in their minds, and the pictures they did form were less vivid.
They were less likely to daydream in constructive fashion—or to dream.
Blocked individual could be grouped according to their predominant emotional reaction.
\begin{compactitem}
\item Deep emotional anxiety and distress that sapped the joy out of writing.
 Individuals are unmotivated because of excessive self-criticism—nothing they produced was good enough—even though their imaginative capacity remained relatively unimpaired. Although they could still generate images, they tended to ruminate, replaying scenes over and over, unable to move on to something new.
\item Unhappiness expressed interpersonally through anger and irritation at others.
  Individuals are unmotivated because they did not want their work compared to the work of others.
  Not everyone was afraid of criticism; some said that they did not want to be ``object[s] of envy.''
  Although their daydreaming capacity was largely intact, they tended to use it to imagine future interactions with others. 
\item Apathetic and disengaged. Individuals could not daydream.
 They lacked originality; and they felt that the ``rules'' they were subjected to were too constrictive.
 Their motivation was also all but nonexistent. 
\item Angry, hostile, and disappointed, strongly negative emotions as opposed to merely sad.
  Individuals tended to look for external motivation; they were driven by the need for attention and extrinsic reward.
  They were more narcissistic, and that narcissism shaped their work as writers.
  They did not want to share their mental imagery, preferring that it stay private.
\end{compactitem}

\section{Cognitive Biases}

Read about cognitive biases\footnote{e.g. , at \url{http://en.wikipedia.org/wiki/List_of_cognitive_biases}}.
Learn to recognize them in yourself and others. For example, very often found in research are:
\begin{compactitem}
\item Bias blind spot
\item Confirmation bias
\item Congruence bias
\item Dunning-Kruger effect
\item Expectation bias
\item Irrational escalation
\item Mere exposure effect
\item Overconfidence effect
\item Planning fallacy
\item Reactance
\item Selective perception
\item Semmelweis reflex
\item Well travelled road effect
\end{compactitem}
\medskip

A somewhat research-specific bias is the following:
Describing X in terms of Y, where Y is already understood, is a great way to understand X.
Often researchers become experts at X and then use it to understand many other things. (This is one reason why knowledge gain can be exponential.)
But sometimes we become too good at X and lose the ability to understand Y independent of X.
That can make us miss interesting insights about Y.
\medskip

A bias I notice again and again in students is that they overestimate their level of expertise.
They are so excited about knowing \emph{some}thing that they try to convince or educate someone else who knows a lot more.
Thus, they sabotage their own learning opportunity at best and alienate a potential ally or mentor at worst.

\section{Miscellaneous}

I (and most other senior researchers) do not keep track of deadlines for submitting your time sheets, contract renewals, thesis proposals, theses, or recommendation letters. That's your responsibility; remind us if necessary.
\medskip

Alan Bundy has some very helpful How To guides on his web site: \url{http://homepages.inf.ed.ac.uk/bundy/} Read in particular, ``How to be my Student'' and ``The Researcher's Bible''.

\paragraph{Mailing lists}
The following mailing lists are very useful to be informed about upcoming calls for papers and current discussions:
\begin{compactitem}
\item \url{http://lists.seas.upenn.edu/mailman/listinfo/types-announce}
\item \url{http://lists.seas.upenn.edu/mailman/listinfo/types-list}
\item \url{http://www.mta.ca/~cat-dist/}
\item \url{http://cs.nyu.edu/mailman/listinfo/fom}
\item \url{http://lists.jacobs-university.de/mailman/listinfo/projects-mkm-ig}
\end{compactitem}

Also consider this web site with conference links
\url{http://www.logic.at/staff/gramlich/conferences.html}

\paragraph{VG-Wort}
In Germany, if you have published a paper, you can report it to the VG-Wort and get money for it. Here are some details:
\begin{compactitem}
\item You can report papers if they satisfy certain visibility conditions, e.g., LNCS or journals.
\item Registration and reporting is done via the VG-Wort website (in German).
\item You need to count how many characters the paper has as they pay per normed pages of 1500 characters.
\item The value of a page varies per year, maybe a few euros, divided among the authors.
\item You have to pay tax on the money you get.
\end{compactitem}

\section{Some Recurring LaTeX Issues}

TeXstudio seems to be the best tex editor at the moment.
\medskip

Familiarize yourself with the following general LaTeX packages:
\begin{compactitem}
\item beamer (slides)
\item listings (source code)
\item tikz (diagrams)
\item hyperref (pdf links and table of contents)
\item ed (margin comments during authoring)
\end{compactitem}
and these custom LaTeX packages by Florian, which you find at \url{https://svn.kwarc.info/repos/kwarc/doc/macros}:
\begin{compactitem}
\item basics (various macros I use for myself)
\item basics-slides (various macros I use when using beamer)
\item mytikz (various macros I use when using tikz)
\item crossref (pdf cross-references)
\item twelf-math (MMT-style theories)
\end{compactitem}
\medskip


bibtex is a part of LaTeX for bibliographies.
The following are important sources of bib entries:
\begin{compactitem}
\item my bib files in the above SVN repos
\item \url{http://citeseer.ist.psu.edu}
\item \url{http://liinwww.ira.uka.de/bibliography/}
\end{compactitem}
When using bibtex, note that
\begin{compactitem}
\item capital letters in the title fields of bibtex entries are lost unless {W}rapped in braces.
\item special characters have to be wrapped as well as in \verb|{\"a}|.
\item your bib entries should use the same format, e.g., contain the same field using the same abbreviations
\end{compactitem}

%\section{Information Specific to Working with Our Kwarc Research Group}
%
%I will assume that you are subscribed to the project-kwarc@jacobs-university.de (all Kwarc) and mmt-dev@jacobs-university.de (MMT) mailing lists.
%You should subscribe yourself if we forget to add you.
%
%We use SVN and git (use the TortoiseSVN and TortoiseGit clients on Windows).
\end{document}