As for FOL, for the proof theoretic semantics of HOL, we add a judgment of provability:
\[\iscons{\Sigma}{\Gamma}{\Delta}{F}.\]
The proof rules that define this judgment are given in Fig.~\ref{fig:hol:proofs}, where the less relevant parts are kept in gray. We use the abbreviations from Sect.~\ref{hol:syntax:abbrevs} and only give rules for $\doteq_A$; the usual rules for the other symbols are derived in Lem.~\ref{lem:hol:proofs}. Note that the rules are such that $\iscons{\Sigma}{\Gamma}{\Delta}{F}$ implies $\oftype{\Sigma}{\Gamma}{F}{o}$.

\begin{fignd}{hol:proofs}{Proof Rules}
%\ibnc{\iscons{\Sigma}{\Gamma}{\Delta,F}{G}}
%     {\oftype{\Sigma}{\Gamma}{F}{o}}
%     {\iscons{\Sigma}{\Gamma}{\Delta}{F\impl G}}
%     {\impl I}
%\tb\tb
%\ibnc{\iscons{\Sigma}{\Gamma}{\Delta}{F\impl G}}
%     {\iscons{\Sigma}{\Gamma}{\Delta}{F}}
%     {\iscons{\Sigma}{\Gamma}{\Delta}{G}}
%     {\impl E}
%\\
\ianc{\oftype{\gSigma}{\gGamma}{t}{A}}
     {\isconsgSGD{t\doteq_A t}}
     {eqrefl}
\tb\tb
\ianc{\isconsgSGD{s\doteq_A t}}
     {\isconsgSGD{t\doteq_A s}}
     {eqsym}
\\
\ibnc{\isconsgSGD{f\doteq_{A\arr B}f'}}
     {\isconsgSGD{t\doteq_A t'}}
     {\isconsgSGD{(f\;t)\doteq_B (f'\;t')}}
     {eqapp}
\tb\tb
\ianc{\iscons{\gSigma}{\gGamma,\;x:A}{\gDelta}{t\doteq_B t'}}
     {\isconsgSGD{(\lam{x}{A}t)\doteq_{A\arr B} (\lam{x}{A}t')}}
     {eqlam}
\\
\ibnc{\oftype{\gSigma}{\gGamma,\;x:A}{t}{B}}
     {\oftype{\gSigma}{\gGamma}{s}{A}}
     {\isconsgSGD{(\lam{x}{A}t)\;s \doteq_B t[x/s]}}
     {beta}
\tb\tb
\ibnc{\oftype{\gSigma}{\gGamma}{f}{A\arr B}}
     {x\mnot\minn \Gamma}
     {\isconsgSGD{f\doteq_{A\arr B} \lam{x}{A}(f\; x)}}
     {eta}
\\
\ibnc{\iscons{\gSigma}{\gGamma}{\gDelta,F}{G}}
     {\iscons{\gSigma}{\gGamma}{\gDelta,G}{F}}
     {\isconsgSGD{F\doteq_o G}}
     {\doteq_o I}
\tb\tb
\ibnc{\isconsgSGD{F\doteq_o G}}
     {\isconsgSGD{F}}
     {\isconsgSGD{G}}
     {\doteq_o E}
\\
\ibnc{F\in\Delta}
     {\oftype{\gSigma}{\gGamma}{F}{o}}
     {\isconsgSGD{F}}
     {axiom}
\tb\tb
\ianc{{\color{gray}\iscont{\Sigma}{\Gamma}}}
     {\isconsgSGD{\forall x:o.(x\doteq_o\true\vee x\doteq_o \false)}}
     {tnd}
\\
\ibnc{\iscons{\gSigma}{\gGamma,x:A}{\gDelta}{F}}
     {x\nin\Gamma,\gDelta,F}
     {\isconsgSGD{F}}
     {nonempty}
\tb\tb
\ibnc{\isconsgSGD{F}}
     {\iscons{\gSigma}{\gGamma}{\gDelta,F}{G}}
     {\isconsgSGD{G}}
     {cut}
\end{fignd}

The intuitions behind the rules are as follows.
\begin{itemize}
 %\item $\impl I$, $\impl E$: implication introduction and elimination as for FOL.
 \item $eqrefl$, $eqsym$: These rules are reflexivity and symmetry and make equality an equivalence relation on terms. (Transitivity is derivable -- exercise.)
 \item $eqapp$, $eqlam$: These rules make equality a congruence relation. $eqapp$ says that applying equal functions to equal arguments yields equal results. $eqlam$ says that $\lambda$-abstraction over equal terms yields equal results.
 \item $beta$, $eta$: These rules axiomatize the meaning of functions. $beta$ says that function application means to substitute the argument for the $\lambda$-abstracted variable.
 \item $\doteq_o I$, $\doteq_o E$: These rules express the intuition that equality between formulas is logical equivalence, i.e., $F\doteq_o G$ is the same as $(F\impl G) \wedge (G\impl F)$.
 \item $tnd$: tertium non datur as for FOL.
 \item $axiom$: axiom rule as for FOL.
 \item $nonempty$: Similar to FOL, this rule makes sure that all types are non-empty.
 \item $cut$: cut rule as for FOL.
\end{itemize}
\medskip

As for FOL, we define proof-theoretic consequence.
\begin{definition}
We say that $F\in\Sen^{HOL}(\Sigma)$ is a proof-theoretical \emph{consequence} of the set of assumptions $\Delta\sq\Sen^{HOL}(\Sigma)$ if there is a derivation of $\es;\Delta\der F$ in the calculus above.
\end{definition}

\begin{lemma}\label{lem:hol:proofs}
Using the abbreviations from Sect.~\ref{hol:syntax:abbrevs}, the natural deduction rules for introduction and elimination of truth, falsity, conjunction, disjunction, implication, and negation as given for FOL are derivable in HOL.

Similarly, the following rules for universal and existential quantification are derivable in HOL:
\[\rul{\gdseqa{\forall x:A.F}}{\gdseqc{x:A}{F}\tb x\nin\Gamma,\Delta}{\forall_A I}
\tb\tb
\rul{\gdseqa{F[\sub{x}{t}]}}{\gdseqa{\forall x:A.F \tb \oftype{\gSigma}{\gGamma}{t}{A}}}{\forall_A E}
\]

\[
\rul{\gdseqa{\exists x:A.F}}{\gdseqa{F[\sub{x}{t}]} \tb \oftype{\gSigma}{\gGamma}{t}{A}}{\exists_A I}
\tb\tb
\rul{\gdseqa{C}}{\gdseqa{\exists x:A.F}\tb\gdseqbc{x:A}{F}{C}\tb x\nin\Gamma,\Delta,C}{\exists_A E}
\]
\end{lemma}
\begin{proof}
We show the case of negation introduction as an example and leave the rest as an exercise.

Using the definition $\neg:=\lam{x}{o}(x\doteq_o \false)$, the rule for negation introduction is
\[\rul{\iscons{\Sigma}{\Gamma}{\Delta}{(\lam{x}{o}(x\doteq_o \false))\; F}}{\iscons{\Sigma}{\Gamma}{\Delta,F}{\false}}{\neg I}\]
To show that it is a derivable rule means to show that there is a derivation of $\iscons{\Sigma}{\Gamma}{\Delta}{(\lam{x}{o}(x\doteq_o \false))\;F}$ using an assumption $\iscons{\Sigma}{\Gamma}{\Delta,F}{\false}$. Such a derivation is this where we assume that we have derived the rule $\false E$ already:

\[\hspace{-1cm}
  \rul{\iscons{\Sigma}{\Gamma}{\Delta}{(\lam{x}{o}(x\doteq_o \false))\;F}}
  {
    \rul{\iscons{\Sigma}{\Gamma}{\Delta}{(F\doteq_o \false) \doteq_o ((\lam{x}{o}(x\doteq_o \false))\;F)}}
    {
      \rul{\iscons{\Sigma}{\Gamma}{\Delta}{((\lam{x}{o}(x\doteq_o \false))\;F) \doteq_o (F\doteq_o \false)}}
      {}
      {beta}
    }
    {eqsym}
  \tb
    \rul{\iscons{\Sigma}{\Gamma}{\Delta}{F\doteq_o \false}}
    {
      \iscons{\Sigma}{\Gamma}{\Delta,F}{\false}
    \tb
      \rul{\iscons{\Sigma}{\Gamma}{\Delta,\false}{F}}
      {}
      {\false E}
    }
    {\doteq_o I}
  }
  {\doteq_o E}
\]
\end{proof}

