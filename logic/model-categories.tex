In this chapter we investigate general methods for studying the collection of models of a given theory.
The word ``universal'' refers to the insight that we can establish many deep definitions and theorems generically for an arbitrary theory.

This section will focus on $\FOLEQ$.
But many results are even more general -- they can be established for an arbitrary theory of an arbitrary logic.
Unless mentioned otherwise, we work with a fixed $\FOLEQ$-signature $\Sigma$.

The basic idea that $\Sigma$-models are sets with structure (namely the structure given by the interpretations of the $\Sigma$-symbols).
Now we want to generalize well-known operations on sets to operations on models.
For the special case where $\Sigma$ is the empty signature, we obtain the analogous operation on sets.

\section{Model Morphisms}\label{sec:fol:modelmorphisms}\label{sec:univ:morph}

\subsection{The Analog for Sets: Functions}

For any two sets $M,N$, we can form the set of functions $\phi:M\to N$.
Functions are essentially the most important concept of mathematics and computer science, representing structure and algorithms.

If $M,N$ are $\Sigma$-models, the morphisms are functions $\phi:\univ^M\to \univ^N$ that preserve the structure.

\begin{example}[Monoid Morphisms]
A morphism from a monoid $M_1=(U_1,*_1,e_1)$ to a monoid $M_2=(U_2,*_2,e_1)$ is a map $\phi:U_1\arr U_2$ such that
\begin{itemize}
  \item $\phi(u *_1 v)=\phi(u) *_2 \phi(v)$,
	\item $\phi(e_1)=e_2$.
\end{itemize}
$\phi$ preserves the operations: It maps composition to composition and unit to unit.
\end{example}

\begin{example}[Monoid Morphisms]
Consider $M_1=(\N,+,0)$ and $M_2=(\Z,+,0)$ and let $\phi:\N\arr\Z$ be the inclusion.
It is easy to see that this inclusion is a monoid morphism.

For a less trivial example, consider $M_1$ and $M_2$ as before but let now $\phi(u)=k u$ for some fixed $k\in\Z$.
This is a monoid morphism for every $k$ (even for $k=0$).

For an example that relates very different monoids, consider $M_1=(\Sigma^*,\cdot,\epsilon)$ and $M_2=(\N,+,0)$ where $\Sigma^*$ denotes the set of words over some fixed alphabet (i.e., Scala-strings), $+$ denotes the binary function that concatenates two words, and $\epsilon$ denotes the empty word. Then ($M_1$ is a monoid and) the length of a word is a monoid morphism from $M_1$ to $M_2$.
\end{example}

The above examples already indicate how powerful the concept of morphisms is: Relations between very different objects can be expressed succinctly.
Furthermore, the definition of morphism does not depend so much on the particular structure that is supposed to be preserved.
For monoids, the preserved structure consists of composition and unit.

%%%%%%%%%%%%%%%%%%%%%%%%%%%%%%%%%%%%%%%%%%%%
\subsection{Formal Definition}

\begin{definition}[Model Morphisms]
For two $\Sigma$-models $M,N$, a $\Sigma$-\textbf{morphism} $\phi:M\to N$ is a mapping from $\univ^M$ to $\univ^N$ such that
\begin{itemize}
  \item for every function symbol $f$ or arity $n$ and for all $u_1,\ldots,u_n\in\univ^M$
      \[\phi(f^M(u_1,\ldots,u_n))=f^N(\phi(u_1),\ldots,\phi(u_n))\]
  \item for every predicate symbol $p$ or arity $n$ and for all $u_1,\ldots,u_n\in\univ^M$
      \[p^M(u_1,\ldots,u_n)\leq p^N(\phi(u_1),\ldots,\phi(u_n))\]
\end{itemize}
$\phi$ is called \textbf{strong} if we have $=$ instead of $\leq$ in the cases for predicate symbol.
\end{definition}

\begin{example}[Morphisms]
Morphisms come up virtually everywhere in mathematics.
\begin{compactitem}
\item The models and morphisms of the empty theory are simply the sets and mappings.
\item All inclusion maps between the number sets $\N$, $\Z$, $\Q$, $\R$, and $\C$ are morphisms for various signatures.
For example, $(\Z,+,0,-,\cdot,1)\to(\C,+,0,-,\cdot,1)$ is a ring morphism.
\item For any number set $U$ and every $k\in U$, the mapping $M_k:x\mapsto kx$ (which multiplies by $k$) is a monoid morphism $(U,+,0)\to(U,+,0)$ because
\[k(x+y)=kx+ky\tb\mand\tb k0=0.\]
\item The exponential function $\exp:x\mapsto e^x$ is a group morphism $(\R,+,0,-)\to(\R^+,\cdot,1,^{-1})$ where $\R^+$ is the set of positive real numbers because
\[e^{x+y}=e^x\cdot e^y\tb\mand\tb e^0=1\tb\mand\tb e^{-x}=1/(e^x).\]
Its inverse $\ln$ is a group morphism in the opposite direction.
\item Let $\modulus_k:\Z\to\Z_k$ be the mapping $n\mapsto n\modop k$.
Then $\modulus_k$ is a ring morphism $(\Z,+,0,-,\cdot,1)\to(\Z_k,+_k,0,-_k,\cdot_k,1)$.
Moreover, if $k|l$, the mapping $\modulus_k:\Z_l\to\Z_k$ is a ring morphism between the corresponding rings.
(The former is a special case of the latter using $l=0$, $k|0$, and $\Z_0=\Z$.)
\item The real and imaginary part functions $\mathrm{Re,Im}$ are homomorphisms $(\C,+,0,-)\to(\R,+,0,-)$ because
\[\mathrm{Re}(x+y)=\mathrm{Re}x+\mathrm{Re}y\tb\mand\tb \mathrm{Re}0=0\tb\mand\tb\mathrm{Re}(-x)=-\mathrm{Re}x\]
and accordingly for $\mathrm{Im}$.
\item The conjugation $x+yi\mapsto x-yi$ is a field isomorphism from $(\C,+,0,-,\cdot,1,^{-1})$ to itself.
\item The order morphisms $\phi:(U,\leq)\to(U',\leq')$ are the monotonous mappings, i.e., the mappings for which $u\leq v$ implies $\phi(u)\leq'\phi(v)$.
\item The complement $-^C:\pwr(S)\to\pwr(S)$ is a lattice morphism $(\pwr(S),\cup,\es,\cap,S)\to(\pwr(S),\cap,S,\cup,\es)$.
It is also a strong order morphism $(\pwr(S),\sq)\to(\pwr(S),\supseteq)$.
\item The morphisms between vector spaces are the linear maps.
\item The length-function $|-|$ is a monoid morphism $(\Sigma^*,\cdot,\epsilon)\to(\N,+,0)$ because
\[|vw|=|v|+|w|\tb\mand\tb |\epsilon|=0.\]
\item Let $U$ be a sufficiently large set of finite sets that contains $0:=\es$, $1:=\{\es\}$, disjoint unions $A\sqcup B:=\{0\}\times A\cup \{1\}\times B$, and function sets $A\to B$.
Then the cardinality-function $|-|$ is a morphism $(U,\sqcup,\es,\times,\{\es\},\to)\to(\N,+,0,\cdot,1,(x,y)\mapsto y^x)$ because
\[|\es|=0 \tb |\{\es\}|=1 \tb |A\sqcup B|=|A|+|B| \tb |A\times B|=|A|\cdot |B| \tb |A\to B|=|B|^{|A|}.\]
\end{compactitem}
\end{example}

%%%%%%%%%%%%%%%%%%%%%%%%%%%%%%%%%%%%%%%%%%%%
\subsection{Morphisms as Property-Preserving Maps}

Intuitively, morphisms preserve properties.
In fact, the main reason why morphisms are interesting is that they preserve properties -- many important mathematical theorems can be stated as the preservation of some property under some operation.
The following definition makes precise what we mean by that:

\begin{definition}[Preservation and Reflection]
Consider a morphism $\phi:M\to N$ and a $\Gamma$-formula $F$.

If $\alpha$ is an assignment from $\Gamma$-into $M$, we write $\phi(\alpha)$ for the assignment from $\Gamma$ into $N$ that maps $\phi(\alpha)(x)=\phi(\alpha(x))$.

Then we say that
\begin{compactitem}
\item $\phi$ \textbf{preserves} $F$ if for every assignment $\alpha$
\[\semm{F}{M,\alpha}\leq \semm{F}{N,\phi(\alpha)}\]
\item $\phi$ \textbf{reflects} $F$ if for every assignment $\alpha$
\[\semm{F}{M,\alpha}\geq \semm{F}{N,\phi(\alpha)}\]
\end{compactitem}
For a $\Gamma$-term $t$, we say that $\phi$ preserves/reflects $t$ if it preserves/reflects the formula $y=t$, where $y$ is an additional variable that is not in $\Gamma$.

We say that $\phi$ preserves/reflects the predicate symbol $p$ if it preserves/reflects the formula $p(x_1,\ldots,x_n)$.
We say that $\phi$ preserves/reflects the function symbol $f$ if it preserves/reflects the term $f(x_1,\ldots,x_n)$.
\end{definition}

%Intuitively, preservation of terms means that $\phi$ can be pulled into (or reversely: out of) and term or atomic formula.
%Therefore, other words for morphisms are \defemph{structural}, \defemph{recursive}, or \defemph{compositional} mapping because the result of applying $\phi$ to the interpretation of a term is obtained by recursively applying $\phi$ to the interpretations of the subterms and composing the results using the same term structure.

Note that the definition of the preservation of $F$ can be rephrased more intuitively as: if $F$ holds in $M$ under $\alpha$, then it holds in $N$ under $\phi(\alpha)$.
Even more intuitively, $F$ stays true when going from $M$ to $N$ via $\phi$.
That is why we call it \emph{preservation}.
The definition of reflection can be rephrased accordingly.

Moreover, we can rephrase the definition of morphisms as follows: A mapping $\phi$ is a morphism iff it preserves function and predicate symbols.
A morphism reflects function symbols if it is injective and reflects predicate symbols iff it is strong.
In fact, morphisms preserve even more:

\begin{theorem}[Preservation]
A morphism preserves all terms and all atomic formulas.
\end{theorem}
\begin{proof}
Exercise.
\end{proof}

Morphisms do \emph{not} in general preserve \emph{all} formulas.
The preservation of all formulas is usually only possible if the involved models are very similar.
But morphisms are most interesting and valuable if they allow connecting very different models, i.e., models that satisfy very different formulas.
Therefore, morphisms are only required to preserve atomic formulas.

\begin{theorem}[Preservation/Reflection of Complex Formulas]\label{thm:univ:preserve}
Under the following conditions, a morphism $\phi$ preserves/reflects the respective formula:
\begin{center}
\begin{tabular}{|l|l|l|}
\hline
The formula & is preserved if & is reflected if \\
\hline
$s\doteq t$ & always & $\phi$ injective \\
$p(t_1,\ldots,t_n)$ & always & $\phi$ strong \\
$\true$     & always & always \\
$\false$    & always & always \\
$\neg F$    & $F$ reflected & $F$ preserved \\
$F\wedge G$ & $F,G$ preserved & $F,G$ reflected \\
$F\vee G$   & $F,G$ preserved & $F,G$ reflected \\
$F\impl G$  & $F$ reflected, $G$ preserved & $F$ preserved, $G$ reflected\\
$\forall x.\;F$ & $F$ preserved, $\phi$ surjective & $F$ reflected \\
$\exists x.\;F$ & $F$ preserved & $F$ reflected, $\phi$ surjective \\
\hline
\end{tabular}
\end{center}
\end{theorem}
\begin{proof}
Exercise.
\end{proof}

For example, we can see from the table that
\begin{compactitem}
 \item morphisms preserve all formulas that do not use $\forall$, $\neg$, or $\impl$, and $\forall$ is allowed in the surjective case,
 \item injective strong morphisms reflect all formulas that do not use $\exists$, $\neg$, or $\impl$,
 \item bijective strong morphisms preserve and reflect all formulas.
\end{compactitem}

%%%%%%%%%%%%%%%%%%%%%%%%%%%%%%%%%%%%%%%%%%%%
\subsection{Special Morphisms}

\begin{theorem}[Identity Morphism]
For every theory and every model $M$, the identity $\id{M}:M\to M$, which maps $\id{M}(x)=x$, is a morphism.
\end{theorem}
\begin{proof}
Exercise.
\end{proof}

\begin{theorem}[Composition]
For every theory and morphisms $\phi_1:M_1\to M_2$ and $\phi_2:M_2\to M_3$, the composition mapping $\phi_1;\phi_2:M_1\to M_3$, which maps $(\phi_1;\phi_2)(x)=\phi_2(\phi_1(x))$, is a morphism.
\end{theorem}
\begin{proof}
Exercise.
\end{proof}

\begin{definition}[Powers]
Given a model morphism $\phi:M\to M$, we define $\phi^n:M\to M$ by
\[\phi^0=\id{M} \tb\mand\tb \phi^{n+1}=\phi^n;\phi\]
\end{definition}

In particular, $\phi^1=\phi$ and $\phi^{n+1}(x)=\phi(\phi^n(x))$.

\begin{theorem}[Kinds of Morphisms]
Consider a morphism $\phi:M\to N$.
Then:
\begin{compactitem}
\item The following are equivalent:
	\begin{compactitem}
	\item $\phi$ is an injective mapping
	\item $\phi$ can be canceled in terminal position: $\psi;\phi=\psi';\phi$ implies $\psi=\psi'$
	\end{compactitem}
	In that case, $\phi$ is called a \textbf{mono(morphism)}.
\item The following are equivalent:
	\begin{compactitem}
	\item $\phi$ is a surjective mapping
	\item $\phi$ can be canceled in initial position: $\phi;\psi=\phi;\psi'$ implies $\psi=\psi'$
	\end{compactitem}
	In that case, $\phi$ is called an \textbf{epi(morphism)}.
\item The following are equivalent:
	\begin{compactitem}
	\item $\phi$ is strong and a bijective mapping
	\item $\phi$ is strong, mono, and epi
	\item $\phi$ can be inverted: there is a morphism $\phi^{-1}$ such that $\phi;\phi^{-1}=\id{M}$ and $\phi^{-1};\phi=\id{N}$
	\end{compactitem}
	In that case, $\phi$ is called an \textbf{iso(morphism)}.
\end{compactitem}
\end{theorem}
\begin{proof}
Exercise.
For isomorphisms, note that the mapping $\phi^{-1}$ exists uniquely if the mapping $\phi$ is bijective.
\end{proof}

\begin{theorem}
The identity is iso and thus in particular strong, mono, and epi.

If two morphisms are strong, mono, epi, or iso, then so is their composition.

The inverse of an iso is also iso.
\end{theorem}
\begin{proof}
Exercise.
\end{proof}

\begin{terminology}
Morphism from $M\to M$ are sometimes called \textbf{endomorphisms} of $M$.
Iso-Endo-Morphisms of $M$ are sometimes called \textbf{automorphisms} of $M$.
\end{terminology}

%%%%%%%%%%%%%%%%%%%%%%%%%%%%%%%%%%%%%%%%%%%%
\section{Closed Subsets and Submodels}\label{sec:univ:sub}

\subsection{The Analog for Sets: Subsets}

If for every $x\in M$, we also have $x\in N$, we write $M\sq N$ and call $M$ a subset of $N$.
In that case, we have the inclusion function $M\harr N$, which maps every $x\in M$ to itself.

\subsection{Formal Definition}

\subsubsection{Closed Subsets}

\begin{definition}[Closed Set]
Consider a $\Sigma$-model $N$.
A subset $S\sq\univ^N$ is called \textbf{closed} if
\begin{compactitem}
 \item for every $n$-ary function symbol $f$
  \[\mif \tb u_1\in S, \ldots,u_n\in S\tb\mthen\tb f^M(u_1,\ldots,u_n)\in S.\]
\end{compactitem}
\end{definition}

\subsubsection{Images of Morphisms}

\begin{definition}[Image]
Consider a $\Sigma$-morphism $\phi:M\to N$.
The \textbf{image} $\image{\phi}$ of $\phi$ is the subset of $N$ defined by $\{\phi(m):m\in\univ^M\}$.
\end{definition}

\begin{theorem}
The image of a morphism is closed.
\end{theorem}
\begin{proof}
Exercise.
\end{proof}

\subsubsection{Submodels}

\begin{definition}[Submodel]
Consider a $\Sigma$-model $N$ and a closed set $S$.
We define the submodel $M=N|S$ by
\begin{compactitem}
\item $\univ^{M}=S$,
\item for every $n$-ary function symbol $f$
 \[f^{M}(u_1,\ldots,u_1)=f^{N}(u_1,\ldots,u_n),\]
\item for every $n$-ary predicate symbol $p$
 \[p^{M}(u_1,\ldots,u_1)=p^{M}(u_1,\ldots,u_n).\]
\end{compactitem}

Well-definedness: $f^M$ is well-defined (i.e., returns an element of $\univ^M$) because $S$ is closed.
\end{definition}

Intuitively, the submodels of $N$ are the subsets of $\univ^N$ that are closed.

\begin{theorem}
The submodels of $N$ are exactly those models $M$ with $\univ^M\sq \univ^N$ for which the inclusion mapping is a morphism.
\end{theorem}
\begin{proof}
Consider a subset $S$ of $\univ^N$.
We have the inclusion mapping $S\to \univ^N$, which maps $x\mapsto x$.
We have to show that the following are equivalent:
\begin{compactenum}
 \item $S$ is closed.
 \item The inclusion is a morphism $N|S\to N$
\end{compactenum}
This is an exercise.
\end{proof}

\begin{example}[Submodels]
Almost every model has a rich structure of different submodels.
\begin{compactitem}
\item The submodels of the empty theory are simply the subsets.
\item Every model $M$ is its own submodel $M\leq M$.
Of course, this is the greatest possible submodel of $M$.
\item For every model $M$, we can form the set $Defin_M := \{u\in \univ^M\,|\,\mexists t \msuchthat \semm{t}{M}=u\}$.
Intuitively, it contains all definable elements, i.e., all elements that can be named by a term $t$ in the syntax.
$Defin_M$ is the smallest submodel of $M$.
For example, $Defin_O=\es$ for every order $O$, and $Defin_R=\{0\}$ for every ring $R$.
\item All inclusion maps between the number sets $\N$, $\Z$, $\Q$, $\R$, and $\C$ yield submodels for various signatures.
For example, $(\Z,+,0,-,*,1)$ is a subring of $(\C,+,0,-,*,1)$.
And the rational numbers are a subfield of the real numbers, which are a subfield of the complex numbers.
\item The positive integers with $0$ form a subsemiring of $(\Z,+,0,\cdot,1)$.
They do not form a subring because they are not closed under negation.
The according result applies to rational and real numbers.
\item The positive real numbers without $0$ form a subgroup of $(\R,\times,1,^{-1})$.
The according result applies to rational numbers.
\item For $k\in\N$, let $k\Z=\{kz : z\in\Z\} = \{z\in\Z\;|\;k|z\}$ be the set of integers that are divisible by $k$.
(Note that $1\Z=\Z$ and $0\Z=\{0\}$.)
This set is the image of the endomorphism $z\mapsto kz$.
Thus $k\Z$ is closed and yields a subring of the integers.
Moreover, we have $k\Z\sq l\Z$ iff $l|k$.
\item The universal relation $\top_M=\{(x,y)\,:\,x,y\in\univ^M\}=\univ^M\times\univ^M$ is an equivalence relation( and thus a model of the theory of equivalence relations).
Its submodels are the equivalence relations on $\univ^M$.
\end{compactitem}
\end{example}

\subsection{Interaction with Theories}

Note that if $M\leq N$ and $N$ is a $(\Sigma;\Theta)$-model, then $M$ may or not be a $(\Sigma;\Theta)$-model.
Thus, $M$ has fewer submodels if $\Theta$ has more axioms.

However, we know that the inclusion morphism $M\to N$ is strong and mono.
Therefore, Thm.~\ref{thm:univ:preserve} tells us that a propositional formula (i.e., a formula without $\forall$, $\exists$) holds in $M$ for some assignment iff it holds in $N$ for the same assignment.
Universally quantified propositional formulas that hold in a model also hold in any submodel.
And existentially quantified propositional formulas that hold in a model also hold in any supermodel.

%%%%%%%%%%%%%%%%%%%%%%%%%%%%%%%%%%%%%%%%%%%%
\section{Product Models}\label{sec:univ:prod}

\subsection{The Analog for Sets: Cartesian Product}

For any two sets $M_1,M_2$, we can form the set $M_1\times M_2$.
It comes with projection functions $(-)_i:M_1\times M_2\to M_i$ such that $(u,v)_1=u$ and $(u,v)_2=v$.

\subsection{Formal Definition}

The product of two models uses the cartesian product of the universes as its universe and interprets every function and predicate symbol component-wise:

\begin{definition}[Product Model]
Consider two $\Sigma$-models $M_1, M_2$.
We define the $\Sigma$-model $M_1\times M_2$ as follows:
\begin{compactitem}
\item $\univ^{M_1\times M_2}=\univ^{M_1}\times\univ^{M_2}$,
\item for every $n$-ary function symbol $f$
 \[f^{M_1\times M_2}(u^1,\ldots,u^n)=(f^{M_1}(u^1_1,\ldots,u^n_1),f^{M_2}(u^1_2,\ldots,u^n_2))\]
\item for every $n$-ary predicate symbol $p$
 \[p^{M_1\times M_2}(u^1,\ldots,u^n)=1 \tb\miff\tb p^{M_1}(u^1_1,\ldots,u^n_1)=1 \mand p^{M_2}(u^1_2,\ldots,u^n_2)=1\]
\end{compactitem}
\end{definition}

For $i=1,2$, we obtain projection morphisms $(-)_i:M_1\times M_2\to M$ that map $(u,v)_1=u$ and $(u,v)_2=v$.

\begin{example}
Products are one the most important ways to obtain to build larger models from smaller ones. (Submodels and quotients only produce smaller models.)
\begin{compactitem}
\item The commutative group $(\C,+,0,-)$ can be seen as the product of $(\R,+,0,-)$ with itself because
\[(x+yi)+(x'+y'i)=(x+x')+(y+y')i\tb 0=0+0i\tb -(x+yi)=-x+(-y)i\]
The two projections are $\mathrm{Re}$ and $\mathrm{Im}$.

However, complex multiplication cannot be obtained in this way.
\item Every $k$-dimensional $F$-vector space is isomorphic to the $n$-th power $P$ (i.e., the $n$-fold product with itself) of $(F,+,0,-)$.
The operations in $P$ are
\[\vect{u_1\\\vdots\\u_n}+^P\vect{v_1\\\vdots\\v_n}=\vect{u_1+v_1\\\vdots\\u_n+v_n} \tb
  0^P=\vect{0\\\vdots\\0}\tb
  -^P\vect{u_1\\\vdots\\u_n}=\vect{-u_1\\\vdots\\-u_n}
\]
\item The product of two rings is again a ring.
\item The product $P=F\times G$ of two fields is only a ring and not a field.
The problem is that elements like $(0^F,1^G)$ are different from $0^P=(0^F,0^G)$ but do not have a multiplicative inverse.
\end{compactitem}
\end{example}

\subsection{Interaction with Theories}

If $M_1$ and $M_2$ are $(\Sigma;\Theta)$-models, then $M_1\times M_2$ may or may not be a $(\Sigma;\Theta)$-model.
Therefore, if we restrict attention to $(\Sigma;\Theta)$-models we may not always have product models.

However, we can apply Thm.~\ref{thm:univ:preserve} using that the projection morphisms are epi (but not necessarily strong) if both factors are non-empty.
Moreover, we know that $(u,v)=(u',v')$ iff $u=u'$ and $v=v'$.
Therefore, if both factors are non-empty, then any formula that does not use $\neg$, $\vee$, or $\impl$ holds in a product iff it holds in both factors.
Similarly, any Horn formula holds in a product if it holds in both factors.

Thus, many important theories admit product models.
An important exception is the theory of fields (which uses an axiom with $\neg$). 

%%%%%%%%%%%%%%%%%%%%%%%%%%%%%%%%%%%%%%%%%%%%
\section{Power Models}\label{sec:univ:exp}

\subsection{The Analog for Sets: Families of Elements}

For any two sets $M,I$, we can form the set $M^I$ of functions from $M$ to $I$.
Such a function can be seen as a family $(u_i)_{i\in I}$ of values $u_i\in M$.

\subsection{Formal Definition}

We obtain a simple generalization to morphisms if we use a $\Sigma$-model $M$ and a set $I$.
We call $I$ the index set.

\begin{definition}[Power Model]
Consider a $\Sigma$-model $M$ and a set $I$.
We define the $\Sigma$-model $P=M^I$ as follows:
\begin{compactitem}
\item $\univ^{P}=(\univ^{M})^I$,
\item for every $n$-ary function symbol $f$
 \[f^{P}(u^1,\ldots,u^n)=(f^{M}(u^1_i,\ldots,u^n_i))_{i\in I}\]
\item for every $n$-ary predicate symbol $p$
 \[p^{P}(u^1,\ldots,u^n)=1 \tb\miff\tb p^{M}(u^1_i,\ldots,u^n_i)=1 \mforall i\in I\]
\end{compactitem}
\end{definition}

For every $i\in I$, we obtain a projection morphism $M^I\to M$ that maps $u\mapsto u_i$.

\begin{example}
Power models have some important cases if $I$ is finite:
\begin{compactitem}
\item If $I$ is the set $\{0,\ldots,k-1\}$, then $M^I$ is isomorphic to the $k$-th power of, i.e., $M\times\ldots\times M$.
In particular, $M^{\{0,1\}}$ is isomorphic to $M\times M$, and the two definitions of the projection morphisms coincide.
\item $M^{\{0\}}$ is isomorphic to $M$ itself, and the projection is that isomorphism.
\item $M^\es$ is isomorphic to the trivial model, and the projection is the unique morphism into the terminal model (see Sect.~\ref{sec:univ:term}).
\end{compactitem}
\end{example}

\subsection{Interaction with Theories}

If $M$ is a $(\Sigma;\Theta)$-model, then $M^I$ may or may not be a $(\Sigma;\Theta)$-model.
Therefore, if we restrict attention to $(\Sigma;\Theta)$-models we may not always have power models.

However, we can use the same arguments as for product models to show that $M^I$ is a $(\Sigma;\Theta)$-model for special cases of $\Theta$.

%%%%%%%%%%%%%%%%%%%%%%%%%%%%%%%%%%%%%%%%%%%%
\section{Terminal Models}\label{sec:univ:term}

Terminal models do not always exist and are not always useful.
But they are the dual of the crucially important concept of initial models (see Sect.~\ref{sec:univ:init}).

\subsection{The Analog for Sets: The Trivial Sets}

Singleton sets (i.e., sets of size $1$) have a special property.
If $T=\{t\}$ for some $t$, there is a unique function $M\to T$ for every set $M$: It maps every $x\in M$ to $t$.

\subsection{Formal Definition}

\begin{definition}[Terminal Model]
A $(\Sigma;\Theta)$-model $T$ is called \textbf{terminal} if for every $(\Sigma;\Theta)$-model $M$ there is a unique morphism $M\to T$.
\end{definition}

\begin{theorem}[Trivial Model]
For $\Theta=\es$, we obtain a terminal $\Sigma$-model $T$ as follows:
\begin{compactitem}
 \item $\univ^T$ is some singleton set, e.g., $\{\ast\}$,
 \item $f^T$ for function symbols $T$ is the uniquely determined map that always returns $\ast$,
 \item $p^T$ for predicate symbols $p$ is the map that always returns $1$.
\end{compactitem}
Moreover, all terminal $\Sigma$-models are of this form.
\end{theorem}
\begin{proof}
Exercise.
\end{proof}

\subsection{Interaction with Theories}

If $\Theta\neq\es$, the terminal $\Sigma$-model may or may not be a $(\Sigma;\Theta)$-model.
If it is a $(\Sigma;\Theta)$-model, it is also a terminal $(\Sigma;\Theta)$-model.
If not, there may or not be other $(\Sigma;\Theta)$-models that are terminal.

However, it is easy to see that the terminal $\Sigma$-model satisfies all formulas that do not contain $\false$ or $\neg$.
Therefore, it is terminal for many theories.

%%%%%%%%%%%%%%%%%%%%%%%%%%%%%%%%%%%%%%%%%%%%
\section{Initial Models}\label{sec:univ:init}

\subsection{The Analog for Sets: The Smallest Set}

The empty set $\es$ is the smallest set in the sense that $\es\sq M$ for every set $M$.
Moreover, for every set $M$, there is a unique map $\es\to M$, namely the empty map $\es:\es\harr M$.

\subsection{Formal Definition}

We do not always have a smallest $\Sigma$-model.
But we do have the dual of the terminal model:

\begin{definition}[Initial Model]
A $\Sigma$-model $T$ is called \textbf{initial} if for every $\Sigma$-model $M$ there is a unique morphism $I\to M$.
\end{definition}

\begin{theorem}[Term Model]\label{thm:univ:termmodel}
For $\Theta=\es$, we obtain an initial $\Sigma$-model $I$ as follows:
\begin{compactitem}
 \item $\univ^I$ is the set of all closed $\Sigma$-terms,
 \item $f^I$ for function symbols $T$ maps $f^I(t_1,\ldots,t_n)=f(t_1,\ldots,t_n)$,
 \item $p^I$ for predicate symbols $p$ always returns $0$.
\end{compactitem}
\end{theorem}
\begin{proof}
The unique morphism $u:I\to M$ maps $u(t)=\semm{t}{M}$.
The rest is left as an exercise.
\end{proof}

$\FOL$ does not have an initial model for every theory.
The most important theories for which initial models exists are the Horn theories:

\begin{theorem}[Term Model for Horn Theories]\label{thm:univ:termmodel2}
If $\Theta$ contains only Horn formulas, we obtain an initial $\Sigma$-model $I$ as follows:
\begin{compactitem}
 \item $\univ^I$ is the set of all closed $\Sigma$-terms quotiented by the equivalence relation $\Equiv$ defined by
   \[s\Equiv t  \tb\miff\tb \iscons{\Sigma}{}{\Theta}{s\doteq t}.\]
 \item $f^I$ for function symbols $T$ maps \[f^I([t_1],\ldots,[t_n])=[f(t_1,\ldots,t_n)],\]
 \item $p^I$ for predicate symbols $p$ maps \[p^I([t_1],\ldots,[t_n])=\cas{1\mifc \iscons{\Sigma}{}{\Theta}{p(t_1,\ldots,t_n)}\\0\mothw}\]
\end{compactitem}
\end{theorem}
\begin{proof}
The model is well-defined because $\Equiv$ is a congruence relation (see Sect.~\ref{sec:univ:quot}).
Moreover, $I$ is the quotient of the term model constructed in Thm.~\ref{thm:univ:termmodel}.

The unique morphism $u:I\to M$ maps $u([t])=\semm{t}{M}$.
The rest is left as an exercise.
\end{proof}

Note that the term model of Thm.~\ref{thm:univ:termmodel} is isomorphic to the one from Thm.~\ref{thm:univ:termmodel2} for $\Theta=\es$.

\begin{example}[Initial Models]
Many data types can be specified by Horn theories, and we obtain may initial models:
\begin{compactitem}
\item The initial set is the empty set.
\item More generally, the empty model is initial whenever it is a model at all (i.e., whenever it satisfies all axioms in $\Theta$), e.g., for the theory of orders.
\item The trivial model is initial for monoids, groups, rings without 1, and vector spaces.
Indeed, we can check that all closed terms over these signatures are equal to the neutral element.
\item The integers are an initial ring with 1.
Indeed, the smallest set of not-provably-equal terms over the theory of rings with 1 is $\{0,1,-1,1+1,-(1+1),\ldots\}$.
\item Peano arithmetic from Ex.~\ref{ex:nat:succ} is not a Horn theory. However, the term model is still initial.
It is isomorphic to the natural numbers.
Indeed, the smallest set of not--provably equal terms over Peano arithmetic is $\{0,\suc(0),\suc(\suc(0)),\ldots\}$.
\item The theory of fields is not a Horn theory and has no initial model.
\end{compactitem}
\end{example}

\subsection{Interaction with Theories}

The initial $(\Sigma;\Theta)$-model $I$ from Thm.~\ref{thm:univ:termmodel2} has a very special property --- it satisfies exactly the theorems:
\[\semm{F}{I}=1\tb\miff\tb \iscons{\Sigma}{}{\Theta}{F}\]
For atomic formulas, that holds by construction.
For the remaining formulas, we can show it by induction using the fact all axioms are Horn formulas.

If $\Theta$ contains non-Horn formulas, $I$ is usually not a $(\Sigma;\Theta)$-model and therefore also not an initial $(\Sigma;\Theta)$-model.
However, there may be other $(\Sigma;\Theta)$-models that are initial.

%%%%%%%%%%%%%%%%%%%%%%%%%%%%%%%%%%%%%%%%%%%%
\section{Congruence Relations and Quotient Models}\label{sec:univ:quot}

As we will see in Sect.~\ref{sec:univ:meta:subquot}, quotients are the dual construction to submodels.

\subsection{The Analog for Sets: Equivalence Relations and Quotient Sets}

An equivalence relation on a set $M$ is a binary relation $\Equiv$ on $M$ that is reflexive, symmetric, and transitive.
In that case, we can form the equivalence class $[m]:=\{x\in M\,|\,m\Equiv x\}$ for every $m\in M$.

The set $M/\Equiv:=\{[m]:m\in M\}$ of all equivalence classes is called the quotient of $M$ by $\Equiv$.

The classification mapping $[-]:M\to M/\Equiv$ maps every $m$ to its equivalence class.
It is surjective, and we have $m\Equiv n\iff [m]=[n]$.

\subsection{Formal Definition}

\subsubsection{Congruences}

It is easy to obtain equivalence relations on models.
But not every equivalence relation interacts well with the symbols.
We are interested in those relations that are closed under the symbols:

\begin{definition}[Congruence Relation]
Consider a $\Sigma$-model $M$ and a binary relation $r$ on $M$.
$r$ is called a \textbf{pre-congruence} if
\begin{compactitem}
\item for every $n$-ary function symbol $f$
  \[\mif\tb (u_1,v_1)\in r,\;\ldots,(u_n,v_n)\in r\tb\mthen\tb (f^M(u_1,\ldots,u_n),f^M(v_1,\ldots,v_n))\in r\]
\item for every $n$-ary predicate symbol $f$
  \[\mif\tb (u_1,v_1)\in r,\;\ldots,(u_n,v_n)\in r\tb\mthen\tb p^M(u_1,\ldots,u_n)=p^M(v_1,\ldots,v_n)\]
\end{compactitem}
$r$ is called a \textbf{congruence} if it is additionally an equivalence relation.
\end{definition}

\subsubsection{Kernels of Morphisms}

\begin{definition}[Kernels]
Consider a $\Sigma$-morphism $\phi:M\to N$.
The kernel $\kernel{\phi}$ of $\phi$ is the binary relation on $M$ defined by $\{(x,y)\in \univ^M\times \univ^M\;|\;\phi(x)=\phi(y)\}$.
\end{definition}

\begin{theorem}
The kernel of a morphism is a congruence.
\end{theorem}
\begin{proof}
Exercise.
\end{proof}

\subsubsection{Quotients}

\begin{definition}[Quotient Model]
Consider a $\Sigma$-model $M$ and a congruence $\Equiv$ on it.
We define the $\Sigma$-model $Q=M/\Equiv$ as follows:
\begin{compactitem}
\item $\univ^{Q}=\univ^{M}/\Equiv$,
\item for every $n$-ary function symbol $f$
 \[f^{Q}([u_1],\ldots,[u_1])=[f^{M}(u_1,\ldots,u_n)],\]
\item for every $n$-ary predicate symbol $p$
 \[p^{Q}([u_1],\ldots,[u_1])=p^{M}(u_1,\ldots,u_n).\]
\end{compactitem}

Well-definedness:
Here the arguments of $f^Q$ are equivalence classes, and we define the result of $f^Q$ by applying $f^M$ to arbitrary elements $u_i$ of these equivalence classes.
This is well-defined because the properties of pre-congruences guarantee that all values $f^M(u_1,\ldots,u_n)$ are in the same equivalence class, no matter which $u_i$ we pick.
The argument that $p^Q$ is well-defined proceeds accordingly.
\end{definition}

\begin{theorem}
The quotients of $M$ are exactly those models with universe $\univ^M/\Equiv$ for which the classification mapping is a morphism.
\end{theorem}
\begin{proof}
Consider an equivalence relation $\Equiv$ on $\univ^M$.
We have the mapping $[-]:\univ^M\to \univ^M/\Equiv$, which maps $x\mapsto [x]$.
We have to show that the following are equivalent:
\begin{compactenum}
 \item $\Equiv$ is a congruence.
 \item $[-]:M\to M/\Equiv$ is a morphism.
\end{compactenum}
This is an exercise.
\end{proof}

\begin{example}
Almost every model has a rich structure of quotient models:
\begin{compactitem}
\item Modulo $k$: The relation $m\Equiv_k n$ on $\Z$ is defined by $m-n\in k\Z$. It is a congruence on the ring with 1 of integers.
It is the kernel of the morphism $\modulus_k:\Z\to\Z_k$.
$\Z_k$ is isomorphic to the corresponding quotient.
\item A group is called simple if it has no non-trivial quotient groups. In a huge collaborative undertaking, mathematicians have been able to find all finite simple groups \cite{finitegroups1}, one of the biggest results ever in mathematics.
\item The diagonal relation $\Delta_M=\{(x,x)\,:\,x\in \univ^M\}$ is a congruence.
The quotient $M/\Delta_M$ is isomorphic to $M$.
\item The universal relation $\top_M$ is a congruence.
The quotient $M/\top_M$ is a trivial model.
\end{compactitem}
\end{example}


\subsection{Interaction with Theories}

Note that if $M$ is $(\Sigma;\Theta)$-model, then $M/\Equiv$ may or not be a $(\Sigma;\Theta)$-model.
Thus, $M$ has fewer quotients if $\Theta$ has more axioms.

However, we know that the classification morphism is strong and epi.
Therefore, Thm.~\ref{thm:univ:preserve} tells us that any formula that does not use $\impl$ or $\neg$ that holds in a model also holds in every quotient.

%%%%%%%%%%%%%%%%%%%%%%%%%%%%%%%%%%%%%%%%%%%%
%\section{Factoring Models}
%
% theory with null elements
% congruence relations 
%  [0] closed
%  x\Equiv y \miff x-y\in [0]
%  $[0]$ is submodel of $M$.
% Let $R$ be a system of representatives for $\Equiv$ with canonical form $C$
% Then every x can be uniquely written as r+z with $r\in R$ and $z\in[0]$.
% $M=(M/\Equiv)\times (M|[0])
% $R$ has $\Sigma$-structure by defining $f^R(r_1,\ldots,r_n)=C(f^M(r_1,\ldots,r_n))
%  isomorphic to $M/\Equiv$

% converse: does every submodel yield congruence relation?
% not quite

%%%%%%%%%%%%%%%%%%%%%%%%%%%%%%%%%%%%%%%%%%%%
\section{Meta-Mathematics}\label{sec:univ:meta}

\subsection{Isomorphism as an Equivalence Relation}

The relation $M\cong N$ defined by ``there is an isomorphism from $M$ to $N$'' is an equivalence relation.
Thus, we can quotient $\Mod(\Sigma;\Theta)$ by $\cong$.

For a property $P$, we say ``$P$ holds up to isomorphism'' if the following holds: If $P$ holds for $M$ and $M\cong N$, then $P$ also holds for $N$.
In other words, $P$ holds in the quotient $\Mod(\Sigma;\Theta)/\cong$.

\subsection{Models Containing Morphisms}

Given a model $M$, the endomorphisms of $M$ can be composed with each other.
Thus, they form a magma $(Endo(M),\circ)$.

Moreover,
\begin{itemize}
\item $(Endo(M),\circ, \id{M})$ is a monoid.
\item The set of automorphisms of $M$ is a closed subset of $(Endo(M),\circ, \id{M})$.
In other words, the identity is an automorphism, and the composition of automorphisms is an automorphism.)
$(Auto(M),\circ, \id{M},\inv)$ is a group.
\end{itemize}

\subsection{Product as a Monoid}

The product model operation $M\times N$ is a magma on $\Mod(\Sigma;\Theta)$.
$\cong$ is a congruence with respect to $\times$ and thus the quotient $\Mod(\Sigma;\Theta)/\cong$ is also a magma.
This quotient has several nice properties:
\begin{itemize}
 \item It is commutative: $M\times N\cong N\times M$.
 \item It is associative.
 \item The trivial $\Sigma$-model (if it is $\Theta$-model) is a neutral element.
 \item The empty model (if it is a $\Theta$-model) is an absorbing element.
\end{itemize}

Here it is necessary to work ``up to isomorphism'', i.e., in the quotient by $\cong$.
For example, we have $M\times N \cong N\times M$ but not $M\times N=N\times M$.
Similarly, we have $A\times \{\ast\}\cong A$ but not $A\times \{\ast\}=A$.
Thus, the axioms do not hold exactly --- they only hold if we identify isomorphic models.
An example is the absorbing element: We do have $A\times \es=\es$.

\subsection{Duality}\label{sec:univ:meta:dual}

\subsubsection{Terminal and Initial Models}\label{sec:univ:meta:terminit}

\subsubsection{Submodels and Quotients}\label{sec:univ:meta:subquot}

Given a model $M$, we have the following correspondence of dual concepts:

\begin{center}
\begin{tabular}{|l||l|l|}
\hline
Concept & for submodels & for quotients \\
\hline
\hline
New models & submodels $L$ of $M$ & quotients $Q$ of $M$ \\
Constructed from & closed subsets of $\univ^M$ & congruence relations on $\univ^M$ \\
Characterized by & inclusion monomorphism $L\to M$ & classification epimorphism $M\to Q$ \\
Induced by & images of morphisms into $M$ & kernels of morphisms out of $M$ \\
Biggest model & $M$ & $M/\Delta_M\cong M$ \\
Least model & image of initial model & kernel of terminal model \\
Least model for $\Theta=\es$ & $Defin_M$ & $M/\top_M$ \\
\hline
\end{tabular}
\end{center}

We can even unify closed subsets and congruence relations as follows:
\begin{definition}
An $n$-ary logical relation on $M$ is a closed subset of $M^n$.
\end{definition}

Now we have:
\begin{theorem}\label{thm:univ:closedcong}
The closed subsets of $M$ are the unary logical relations on $M$.
The pre-congruences on $M$ are the binary logical relations on $M$.
\end{theorem}
\begin{proof}
The first result holds by definition.

For the second result, the key idea is that $M\times M$ is a $\Sigma$-model.
The universe of a submodel of $M\times M$ is a subset of $\univ^M\times\univ^M$ (i.e., a binary relation on $M$) with an inclusion morphism into $M\times M$.
Spelling out the morphism properties yields exactly the conditions on pre-congruences.
\end{proof}

Now a crucial observation is the following closure properties:

\begin{theorem}\label{thm:univ:closedinter}
The closed subsets of $M$ are closed under intersection.
\end{theorem}
\begin{proof}
Exercise.
\end{proof}

\begin{theorem}
The congruence relations of $M$ are closed under intersection.
\end{theorem}
\begin{proof}
Due to Thm.~\ref{thm:univ:closedinter} and~\ref{thm:univ:closedcong}, the pre-congruences are closed under intersection.
Moreover, we know that equivalence relations on $\univ^M$ are just the submodels of $\top^M$ and thus by Thm.~\ref{thm:univ:closedinter} closed under intersection.
Therefore, the congruence relations are as well.
\end{proof}

Thus, the closed subsets and the congruence relations of $M$ each form a closure system and thus a lattice.

Concretely, the lattice relation between submodels $N,N'$ of $M$ is given by the subset relation of their universes.
Equivalently, $N\leq N'$ iff there is a monomorphism $N\harr N'$.
The greatest element is $M$ itself.
The least element is $Defin_M$. This is also the image of the unique morphism $I\to M$.
The join of $N$ and $N'$ is the smallest submodel that is bigger than both $N$ and $N'$; we can construct it by starting with $N\cup N'$ and iteratively adding elements until we obtain a closed subset.

The lattice relation between quotients $Q,Q'$ of $M$ is given by the subset relation between their congruence relations.
However, we have to watch out for the following: the bigger the congruence relation, the smaller the quotient.
We will write $Q\prec Q'$ if $Q$ is smaller than $Q'$, i.e., if the congruence of $Q'$ is bigger than that of $Q$.
Equivalently, $Q\prec Q'$ iff there is an epimorphism $Q'\arr Q$.
The greatest quotient is essentially $M$ itself: $M/\Delta_M$, which uses the least congruence relation $\Delta_M$ (which identifies nothing) and is isomorphic to $M$.
The least quotient is $M/\top_M$, which uses the greatest congruence relation (which identifies everything).
This is also the kernel of the unique morphism $M\to T$.

Note that contrary to the least quotient, the least submodel can have rich structure.
For example, the least subring of the real numbers is the ring of integers.

\begin{example}\quad
\begin{compactitem}
\item The mapping $k\mapsto (k\Z,+,0,-,1,\cdot)$ is a lattice morphism from $(\N,|^d,1,0)$ to the lattice of subrings of $\Z$.\\
In particular, if $k|l$, then $l\Z\sq k\Z$ and the inclusion is a monomorphism.\\
The least subring is $0\Z$; its universe is $\{0\}$.\\
The greatest subring is $1\Z$; its universe is $\Z$.
\item The mapping $k\mapsto (\Z_k,+_k,0,-_k,1,\cdot_k)$ is a lattice morphism from $(\N,|,1,0)$ to the lattice of quotients of $\Z$.
(Technically, $\Z_k$ is not a quotient set of $\Z$. But it is isomorphic to the quotient $\Z/\Equiv_k$.)\\
In particular, if $k|l$, then $\Z_k\sq \Z_l$ and $\modulus_k:\Z_l\to\Z_k$ is an epimorphism.\\
The greatest quotient is $\Z/\Equiv_0$, its universe is $\{\{z\}:z\in \Z\}$, which is isomorphic to $\Z_0=\Z$.\\
The least quotient is $\Z/\Equiv_1$; its universe is $\{\Z\}$, which is isomorphic to $\Z_1=\{0\}$.
\end{compactitem}
\end{example}

\subsection{Factorization of Morphisms}

\begin{theorem}\label{thm:univ:factorize}
For every model morphism $\phi:M\to N$, we have $M/(\kernel{\phi})\cong N|(\image{\phi})$.
\end{theorem}
\begin{proof}
The isomorphism maps \[M/(\kernel{\phi}) \ni[m] \mapsto \phi(m) \in  N|(\image{\phi}).\]
Its inverse maps \[N|(\image{\phi})\ni n \mapsto \{m\in\univ^M\;|\;\phi(m)=n\}\in M/(\kernel{\phi}).\]
\end{proof}

\begin{theorem}\label{thm:univ:factorize2}
Every morphism $\phi:M\to N$ is of the form $\phi=e;m$ where $e$ is epi and $m$ is mono.

Every morphism $\phi:M\to N$ is uniquely of the form $\phi=e;i;m$ where $e$ is a classification, $i$ is iso, and $m$ is an inclusion.
\end{theorem}
\begin{proof}
The first statement follows easily from the second.

To prove the second statement, we pick the classification $e:M\to M/(\kernel{\phi})$ and the inclusion $m:N|(\image{\phi})\to N$.
The isomorphism $i$ is then (uniquely) the one from Thm.~\ref{thm:univ:factorize}.
The uniqueness of $m$ and $e$ follows because no other classification would yield the same kernel as $\phi$ and no other inclusion the same image as $\phi$.
\end{proof}