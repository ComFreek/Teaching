In this section, we consider a binary relation $\#$ on a set $A$, i.e., a subset $\#\sq A\times A$.

\subsection{Classification}

\begin{definition}[Properties of Binary Relations]\label{def:math:binrel}
We say that $\#$ is \ldots if the following holds:
 \begin{compactitem}
 \item reflexive:  for all $x$, $x\# x$
 \item irreflexive\footnotemark: for no $x$, $x\# x$
 \item symmetric: for all $x,y$, if $x\# y$, then $y\# x$
 \item anti-symmetric\footnotemark: for all $x,y$, if $x\# y$ and $y\# x$, then $x=y$ (= if $x\# y$ then not $y\# x$ (unless $x=y$))
 \item transitive: for all $x,y,z$, if $x\# y$ and $y\# z$, then $x\# z$
 \end{compactitem}
 
Moreover, we call $\#$ a \ldots if it is:
 \begin{compactitem}
 \item strict order: irreflexive and transitive
 \item preorder: reflexive and transitive
 \item order\footnotemark: preorder and anti-symmetric (= reflexive, transitive, and anti-symmetric)
 \item total\footnotemark order: order and for all $x,y$, $x\# y$ or $y\# x$
 \item partial equivalence: symmetric and transitive (= not necessarily reflexive equivalence)
 \item equivalence: preorder and symmetric (= reflexive, transitive, and symmetric)
 \end{compactitem}

An element $a\in A$ is called \ldots of $\#$ if the following holds:
 \begin{compactitem}
  \item least element:  for all $x$, $a\# x$
  \item greatest element: for all $x$, $x\# a$
  \item least upper bound of $x,y$: $x\# a$ and $y\# a$ and for all $z$, if $x\# z$ and $y\# z$, then $a\# z$
  \item greatest lower bound of $x,y$: $a\# x$ and $a\# y$ and for all $z$, if $z\# x$ and $z\# y$, then $z\# a$
 \end{compactitem}
\end{definition}
\begin{multfootnotetext}{4}
\footnotetext{That's not the same as not being reflexive.}
\footnotetext{That's not the same as not being symmetric.}
\footnotetext{Orders are also called \emph{partial order}, \emph{poset} (for partially ordered set), or \emph{ordering}.}
\footnotetext{This notion of \emph{total} has nothing to do with the one from Def.~\ref{def:math:rel} of the same name.}
\end{multfootnotetext}

\begin{theorem}[Dual Relation]
If a relation is reflexive/irreflexive/symmetric/antisymmetric/transitive/total, then so is its dual.

If $a$ is a least/greatest element for a relation, then it is a greatest/least element for its dual.
If $a$ is a least upper/greatest lower bound of $x,y$ for some relation, then it is a greatest lower/least upper bound of $x,y$ for the dual.
\end{theorem}

\subsection{Equivalence Relations}

\paragraph{Symbols}
Equivalence relations are usually written using infix symbols whose shape is reminiscent of horizontal lines, such as $=$, $\sim$, or $\Equiv$.
Often vertically symmetric symbols are used to emphasize the symmetry property.

\paragraph{Equivalence Classes and Quotients}
Equivalence relations allow grouping related elements into classes and collecting all the classes in what is called a quotient.

\begin{definition}[Quotient]
Consider a relation $\Equiv$ on $A$.
Then
\begin{compactitem}
 \item For $x\in A$, the set $\{y\in A\,\|\,x\Equiv y\}$ is called the (equivalence) \textbf{class} of $x$.
  It is often written as $[x]_\Equiv$.
 \item $A/\Equiv$ is the set of all classes.
  It is called the \textbf{quotient} of $A$ by $\Equiv$.
\end{compactitem}
\end{definition}

\begin{definition}[Kernel]
Consider a function $f:A\to B$. The \textbf{kernel} of $f$, written $\mathrm{ker}\, f$, is the binary relation on $A$ defined by $x\,(\mathrm{ker}\, f)\,y$ iff $f(x)=f(y)$.
\end{definition}

\begin{definition}[Partition]
A \textbf{partition} $P$ on a set $A$ is a set of non-empty, pairwise disjoint subsets of $A$ whose overall union is $A$.
\end{definition}

\begin{theorem}
For a relation $\Equiv$ on $A$, the following are equivalent\footnotemark:
\begin{compactitem}
 \item $\Equiv$ is an equivalence.
 \item $\Equiv$ is the kernel of some function, i.e., there is a set $B$ and a function $f:A\to B$ such that $x\Equiv y$ iff $f(x)=f(y)$.
 \item $A/\Equiv$ is a partition on $A$, i.e., every element of $A$ is in exactly one class in $A/\Equiv$.
\end{compactitem}
\end{theorem}
\footnotetext{Logical equivalence is itself an equivalence relation.}

\paragraph{Partial Equivalence Relations}
Consider a partial equivalence relation $\Equiv$ on $A$.
$\Equiv$ is not an equivalence because it is not reflexive.
However, we can easily prove: if $x \Equiv y$, then $x\Equiv x$ and $y\Equiv y$.
Thus, the only elements for which $x\Equiv x$ does not hold are the ones that are in relation to no element at all.

Thus, we have:
\begin{theorem}\label{thm:math:per}
A partial equivalence relation $\Equiv$ on $A$ is the same as an equivalence relation on a subset of $A$.
\end{theorem}

\paragraph{Normal and Canonical Forms}
Instead of working with equivalence classes, we usually prefer working with representatives, i.e., designated elements of the classes that we use instead of the entire class.

\begin{definition}[System of Representatives]
Consider an equivalence relation $\Equiv$ on $A$.
A subset $R$ of $A$ is a system of \textbf{representatives} for $\Equiv$ if it contains exactly one element from every $\Equiv$-class.
\end{definition}

Normal forms are used to choose representatives for each element:

\begin{definition}[Normal Forms]
Consider an equivalence relation $\Equiv$ on $A$.
A function $N:A\to A$ is called a \ldots if
\begin{compactitem}
 \item \textbf{normal form}: $N(x)\Equiv x$ and $N(N(x))=N(x)$ for all $x\in A$
 \item \textbf{canonical form}: $N$ is a normal form and $N(x)=N(y)$ whenever $x\Equiv y$
\end{compactitem}
\end{definition}

We also call $N(x)$ the normal/canonical form of $x$.
The process of mapping $x$ to $N(x)$ is called \textbf{normalization}.

The main application of canonical forms is that we can check $x\Equiv y$ by comparing $N(x)$ and $N(y)$.

\begin{theorem}\label{thm:math:normalform}
Consider an equivalence relation $\Equiv$ on $A$ and a normal form $N: A\to A$.
The following are equivalent:
\begin{compactitem}
 \item $N$ is a canonical form.
 \item The image of $N$ is a system of representatives.
 \item $\Equiv$ is the kernel of $N$.
\end{compactitem}
\end{theorem}

\subsection{Orders}

\begin{theorem}[Strict Order vs. Order]
For every strict order $<$ on $A$, the relation ``$x<y$ or $x=y$'' is an order.

For every order $\leq$ on $A$, the relation ``$x\leq y$ and $x\neq y$'' is a strict order.
\end{theorem}

Thus, strict orders and orders come in pairs that carry the same information.

Strict orders are usually written using infix symbols whose shape is reminiscent of a semi-circle that is open to the right, such as $<$, $\subset$, or $\prec$.
This emphasizes the anti-symmetry ($x< y$ is very different from $y<x$.) and the transitivity ($< \ldots <$ is still $<$.)
The corresponding order is written with an additional horizontal bar at the bottom, i.e., $\leq$, $\sq$, or $\preceq$.
In both cases, the mirrored symbol is used for the dual relation, i.e., $>$, $\supset$, or $\succ$, and $\geq$, $\supseteq$, and $\succeq$. 

\begin{theorem}\label{thm:math:binrel}
If $\leq$ is an order, then least element, greatest element, least upper bound of $x,y$, and greatest lower bound of $x,y$ are unique whenever they exist.
\end{theorem}

\begin{theorem}[Preorder vs. Order]
For every preorder $\leq$ on $A$, the relation ``$x\leq y$ and $y\leq x$'' is an equivalence.

For equivalence classes $X$ and $Y$ of the resulting quotient, $x\leq y$ holds for either all pairs or no pairs $(x,y)\in X\times Y$.
If it holds for all pairs, we write $X\leq Y$.

The relation $\leq$ on the quotient is an order.
\end{theorem}

\begin{remark}[Order vs. Total Order]
If $\leq$ is a preorder, then for all elements $x,y$, there are four mutually exclusive options:

\begin{ctabular}{|l|c|c|c|}
\hline
& $x\leq y$ & $x\geq y$ & $x=y$ \\
\hline
$x$ strictly smaller than $y$, i.e., $x<y$ & true  & false & false\\
$x$ strictly greater than $y$, i.e., $x>y$ & false & true  & false \\
$x$ and $y$ incomparable      & false & false & false \\
$x$ and $y$ similar           & true  & true  & maybe \\
\hline
\end{ctabular}
Now anti-symmetry excludes the option of similarity (except when $x=y$ in which case trivially $x\leq y$ and $x\geq y$).
And totality excludes the option of incomparability.

Combining the two exclusions, a total order only allows for $x>y$, $y<x$, and $x=y$.
\end{remark}


