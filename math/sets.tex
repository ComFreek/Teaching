The meaning and purpose of a data structure is to describe a set in the sense of mathematics.
Similarly, the meaning and purpose of an algorithm is to describe a function between two sets.

Thus, it is helpful to collect some sets and functions as examples.
These are typically among the first data structures and algorithms implemented in any programming language and they serve as test cases for evaluating our languages.

\subsection{Base Sets}\label{sec:math:sets:prim}

When building sets, we have to start somewhere with some sets that are assumed to exist.
These are called the \emph{bases sets} or the \emph{primitive sets}.

The following table gives an overview of commonly used base sets, where we also list the size of each set according to Def.~\ref{def:setsize}:

\begin{ctabular}{|l|l|l|}
\hline
set & description/definition & size \\
\hline
\hline
\multicolumn{3}{|c|}{typical base sets of mathematics\footnotemark} \\
\hline
$\es$ & empty set & $0$ \\
$\N$  & natural numbers & $C$ \\
$\Z$  & integers & $C$ \\
$\Z_m$ for $m>0$ & integers modulo $m$, $\{0,\ldots,m-1\}$ \footnotemark & $m$\\
$\Q$ & rational numbers & $C$\\
$\R$ & real numbers & $U$\\
\hline
\hline
\multicolumn{3}{|c|}{additional or alternative base sets used in computer science} \\
\hline
$\Void$ & alternative name for $\es$ & $0$\\
$\Unit$     & unit type, $\{()\}$, equivalent to $\Z_1$ & $1$ \\
$\B$  & booleans, $\{\false,\true\}$, equivalent to $\Z_2$ & $2$ \\
$\Int $ & primitive integers, $-2^{n-1},\ldots,2^{n-1}-1$ for machine-dependent $n$, equivalent to $\Z_{2^n}$\footnotemark & $2^n$ \\
$\Float$ & IEEE floating point approximations of real numbers & $C$ \\
$\Char$ & characters & finite\footnotemark \\
$\String$ & lists of characters & $C$ \\
\hline
\end{ctabular}
\begin{multfootnotetext}{4}
\footnotetext{All of mathematics can be built by using $\es$ as the only base set because the others are definable. But it is common to assume at least the number sets as primitives.}
\footnotetext{$\Z_0$ also exists but is trivial: $\Z_0=\Z$.}
\footnotetext{Primitive integers are the $2^n$ possible values for a sequence of $n$ bits.
Old machines used $n=8$ (and the integers were called ``bytes''), later machines used $n=16$ (called ``words'').
Modern machines typically use $32$-bit or $64$-bit integers.
Modern programmers usually---but dangerously---assume that $2^n$ is much bigger than any number that comes up in practice so that essentially (but not actually) $\Int =\Z$.
Some programming languages (e.g., Python) correctly implement $\Int=\Z$.}
\footnotetext{The ASCII standard defined $2^7$ or $2^8$ characters. Nowadays, we use Unicode characters, which is a constantly growing set containing the characters of virtually any writing system, many scientific symbols, emojis, etc.
Many programming languages assume that there is one character for every primitive integers, e.g., typically $2^{32}$ characters.}
\end{multfootnotetext}

\subsection{Functions on the Base Sets}\label{sec:math:sets:primfun}

For every base set, we can define some basic operations.
These are usually built-in features of programming languages whenever the respective base set is built-in.

We only list a few examples here.

\subsubsection{Numbers}
For all number sets, we can define addition, subtraction, multiplication, and division in the usual way.

Some care must be taken when subtracting or dividing because the result may be in a different set.
For example, the difference of two natural numbers is not in general a natural number but only an integer (e.g., $3-5\nin\N$).
Moreover, division by $0$ is always forbidden.

\subsubsection{Quotients of the Integers}
The function $\modulus_m$ (see Sect.~\ref{sec:math:moduloarith}) for $m\in \N$ maps $x\in\Z$ to $x\modop m\in\Z_m$.

In programming languages, the set $\Z_m$ is usually not provided.
Instead, $x\modop y$ is built-in as a function on $\Int$.
\footnote{Some care must be taken if $x$ is negative because not all programming languages agree.}

\subsubsection{Booleans}
On booleans, we can define the usual boolean operations conjunction (usually written $\&$ or $\&\&$), disjunction (usually written $|$ or $||$), and negation (usually written $!$).

Moreover, we have the equality and inequality functions, which take two objects $x,y$ and return a boolean.
These are usually written $x==y$ and $x\,!\!=\,y$ in text files and $x=y$ and $x\neq y$ on paper.

\subsection{Set Constructors}\label{sec:math:sets:deriv}

From the base sets, we build all other sets by applying set constructors.
Those are operations that take sets and return new sets.

The following table gives an overview of commonly used set constructors, where we also list the size of each set according to Def.~\ref{def:setsizecomp}:

\begin{ctabular}{|l|l|l|}
\hline
set & description/definition & size \\
\hline
\hline
\multicolumn{3}{|c|}{typical constructors in mathematics} \\
\hline
$A\uplus B$ & disjoint union & $|A|+|B|$ \\
$A\times B$ & (Cartesian) product & $|A|*|B|$ \\
$A^n$ for $n\in\N$ & $n$-dimensional vectors over $A$ & $|A|^n$ \\
$B^A$ or $A\to B$ & functions from $A$ to $B$ & $|B|^{|A|}$ \\
$\pwr(A)$ & power set, equivalent to $\B^A$ & $2^{|A|}=\cas{2^n \mifc |A|=n \\ U \mothw}$\\
$\{x\in A|P(x)\}$ & subset of $A$ given by property $P$ & $\leq |A|$ \\
$\{f(x):x\in A\}$ & image of operation $f$ when applied to elements of $A$ & $\leq |A|$ \\
$A/r$& quotient set for an equivalence relation $r$ on $A$ & $\leq|A|$ \\
\hline
\hline
\multicolumn{3}{|c|}{selected additional constructors often used in computer science} \\
\hline
$A^*$       & lists over $A$ & $\cas{1 \mifc |A|=0\\ U \mifc |A|=U \\ C \mothw}$ \\
$A^?$       & optional element{\footnotemark} of $A$ & $1 + |A|$ \\
\hline
\multicolumn{3}{|c|}{for new names $l_1,\ldots,l_n$} \\
$\Enum\{l_1,\ldots,l_n\}$  & enumeration: like $\Z_n$ but also introduces &$n$ \\
                           & \tb named elements $l_i$ of the enumeration &  \\ 
$l_1(A_1)|\ldots|l_n(A_n)$ & labeled union: like $A_1\uplus \ldots \uplus A_n$ but also introduces & $|A_1|+\ldots+|A_n|$\\
                           & \tb named injections $l_i$ from $A_i$ into the union & \\ 
$\{l_1:A_1,\ldots,l_n:A_n\}$ & record: like $A_1\times\ldots\times A_n$ but also introduces  & $|A_1|*\ldots*|A_n|$\\
                           & \tb named projections $l_i$ from the record into $A_i$ & \\ 
\hline
\multicolumn{2}{|l|}{inductive data types\footnotemark} & $C$ \\
\multicolumn{2}{|l|}{classes\footnotemark}              & $U$ \\
\hline
\end{ctabular}

\begin{multfootnotetext}{3}
\footnotetext{An optional element of $A$ is either absent or an element of $A$.}
\footnotetext{These are too complex to define at this point. They are a key feature of functional programming languages like SML.}
\footnotetext{These are too complex to define at this point. They are a key feature of object-oriented programming languages like Java.}
\end{multfootnotetext}

\subsection{Characteristic Functions of the Set Constructors}\label{sec:math:sets:derivfun}

Every set constructor comes systematically with characteristic functions into and out of the constructed sets $S$.
These functions allow building elements of $S$ or using elements of $S$ for other computations.

For some sets, these functions do not have standard notations in mathematics.
In those cases, different programming languages may use slightly different notations.

The following table gives an overview:

\begin{ctabular}{|l|l|l|}
\hline
set $C$ & build an element of $C$ & use an element $x$ of $C$ \\
\hline
\hline
$A_1\uplus A_2$ & $\inj_1(a_1)$ or $\inj_2(a_2)$ for $a_i\in A_i$ & pattern-matching \\
$A_1\times A_2$ & $(a_1,a_2)$ for $a_i\in A_i$ & $x.i \in A_i$ for $i=1,2$ \\
$A^n$           & $(a_1,\ldots,a_n)$ for $a_i\in A$ & $x.i \in A$ for $i=1,\ldots,n$ \\
$B^A$ &  $(a\in A) \mapsto b(a)$ & $x(a)$ for $a\in A$ \\
\hline
\hline
$A^*$       & $[a_0,\ldots,a_{l-1}]${\footnotemark} for $a_i\in A$ & pattern-matching \\
$A^?$       & $None$ or $Some(a)$ for $a\in A$ & pattern-matching \\
$\Enum\{l_1,\ldots,l_n\}$  & $l_1$ or \ldots or $l_n$ & switch statement or pattern-matching \\
$l_1(A_1)|\ldots|l_n(A_n)$ & $l_1(a_1)$ or \ldots or $l_n(a_n)$ for $a_i\in A_i$ & pattern-matching \\ 
$\{l_1:A_1,\ldots,l_n:A_n\}$ & $\{l_1=a_1,\ldots,l_n=a_n\}$ for $a_i\in A_i$ & $x.l_i \in A_i$\\
inductive data type $A$ & $l(u_1,\ldots,u_n)$ for a constructor $l$ of $A$ & pattern-matching \\
class $A$               & $\mathbf{new}\;A$    & $x.l(u_1,\ldots,u_n)$ for a field $l$ of $A$ \\
\hline
\end{ctabular}
\footnotetext{Mathematicians start counting at $1$ and would usually write a list of length $n$ as $[a_1,\ldots,a_n]$.
However, computer scientists always start counting at $0$ and therefore write it as $[a_0,\ldots,a_{n-1}]$.
We use the computer science numbering here.}